@preamble{ "\providecommand{\KEYWORDS} " }

\begin{abstract}
    本文针对同济大学会议日程和花费,
    分别解决了满足题干约束情况下的最小化参会费用的问题,
    在五万元经费内尽可能展现学科影响力的问题和在满足题干约束情况下安排教师参会,
    最优化被选为大会报告的期望的问题,
    并且对每位教师制定了科学的出行日程和并给出经费预算.
    总的来说, 本文为同济大学教师参会给出了详细合理的安排. 
    
    针对问题一, 本文建立了多维状态的动态规划模型,
    本文通过检索获得各个会议地点火车公里数和地球大圆距离分别作为火车和飞机路费的依据,
    计算出参加各个会议的费用,
    将参会总人数和十三个会议分别的参会人数设置了十四维度的状态,
    在满足空间和时间复杂度的条件下进行状态转移, 递推求得参加会议的最低费用. 

    针对问题二, 本文建立了分组背包模型, 本文引用教师H指数,
    用教师H指数和大会星级综合评定会议行程所产生的影响力,
    将十八位教师看作十八组,
    把某位教师所有可能的参会组合看作组内的物品,
    利用分组背包模型, 求得每位教师选择至多一种参会行程,
    在总费用资小于等于五万的条件下, 递推求得展现学科影响力的最大值. 
	
    针对问题三, 本文建立了贪心算法的模型,
    用尽可能的让教授和教授参加同一场会议,
    副教授和副教授参加同一场会议的贪心策略, 并且从教授开始考虑,
    不断选择所有参会组合当前对教师报告被选为大会报告期望贡献最大的组合,
    贪心求得最优的方案. 
	
    针对问题四, 本文综合考虑参会费用,
    展示学科影响力和被选为参会报告的会议星级和期望,
    详细制定出科学合理的参会安排. 
	
    本文在问题一和二中, 通过巧妙地设置状态, 合理地求得状态转移方程,
    有效降低求解释时间复杂度, 严格地规划出费用和影响力的最优值.
    但是在问题三中, 本文采用地贪心算法,
    不能保证规划得解的最优性, 只能逼近最优解, 作者会继续探讨更优的算法,
    来更好地解决此问题.
    
    \textbf{关键词}:
        \KEYWORDS
\end{abstract}
