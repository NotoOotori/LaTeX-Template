% !Mode::"TeX:UTF-8"

% -------------------- Information --------------------

\newcommand{\TITLE}{基于动态规划和贪心的参会安排最优化模型}
\newcommand{\AUTHOR}{}
\newcommand{\SUBJECT}{校赛论文}

% -------------------- Packages --------------------

\documentclass[a4paper,12pt]{ctexart}
\usepackage{amsmath}
%\usepackage{amsthm} % 定理格式 由ntheorem代替.
\usepackage{amssymb}
\usepackage{authblk} % 作者 (见校赛论文).
\usepackage{array}
\usepackage{bigfoot} % to allow verbatim in footnote.
\usepackage{boldline} % 长表格表格线加粗.
\usepackage{caption} % 题注.
\usepackage{commath} % abs, norm
\usepackage{enumerate}
%\usepackage{enumitem} 用enumerate包代替.
\usepackage{fancyhdr} % 脚注.
\usepackage{filecontents}
\usepackage{flafter} % 不让float出现在定义之前的地方.
\usepackage{float} % 你们这帮float给我乖乖听话 HHHHHHHHHHH.
\usepackage[T1]{fontenc} % Bera Mono Font
\usepackage{fontspec} % 字体.
\usepackage{graphicx}
\usepackage{hyperref}
\usepackage{lastpage}
\usepackage{listings} % 排版程序语言.
\usepackage{longtable} % 长表格.
\usepackage{makecell} % 表格线加粗 \Xhline{1.2pt}.
\usepackage{mathtools} % \xleftrightarrow.
\usepackage{multirow} % 合并单元格.
\usepackage[square, numbers, sort&compress]{natbib} % 引用.
\usepackage[thmmarks, amsmath, thref]{ntheorem} % 定理格式.
\usepackage[section]{placeins} % 使图像不会显示在别的部分 若过于严格则换成[below].
\usepackage{stackrel} % 上下写 见校赛论文.
\usepackage{SUBSubsubsection}
\usepackage{titlesec} % Section标题格式.
\usepackage{varioref} % For Cross References.
\usepackage[dvipsnames]{xcolor} % 颜色声明.
\usepackage[all, cmtip]{xy} % Commutive diagram.

% Require `ntheorem'

\usepackage{DefaultTheoremStyle}
\usepackage[mathlines, edtable]{lineno} % Line numbers.
    %\begin{edtable}{tabular}[<args>] <entries> \end{edtable}

% Require `xcolor'

\usepackage[numbered, framed]{matlab-prettifier} 
\usepackage{pgfplots}

% -------------------- Settings --------------------

% Title

\title{\TITLE}
\author{\AUTHOR}
\date{\today}

% Package: caption

\captionsetup{
    margin    =   6pt,
    font      =   small,
    labelfont =   bf
}

% Package: fancyhdr

\setlength{\headheight}{15pt}
\lhead{Copyright \copyright\ \AUTHOR}
\rhead{Page \thepage\ of \pageref{LastPage}}

% Package: graphicx

\graphicspath{{resources/}} %图像文件目录
    
% Package: hyperref

\hypersetup{
    linktoc             =   all,
    colorlinks          =   true,
    linkcolor           =   cyan,
    anchorcolor         =   black,
    citecolor           =   blue,
    filecolor           =   cyan,
    menucolor           =   red,
    runcolor            =   filecolor,
    urlcolor            =   magenta,
    pdfinfo             =   {
        Title           =   {\TITLE},
        Author          =   {\AUTHOR},
        Subject         =   {\SUBJECT}},
    bookmarksnumbered   =   true,
    pdfstartview        =   FitH,
    pdfpagelayout       =   OneColumn
}

% Package: lineno

\renewcommand{\linenumberfont}{\normalfont\scriptsize\sffamily}

\let\oldlstinputlisting\lstinputlisting
\renewcommand{\lstinputlisting}[2][\empty]{
    \par\nolinenumbers\oldlstinputlisting[#1]{#2}\linenumbers\par
}

\let\oldlstlisting\lstlisting
\let\oldendlstlisting\endlstlisting
\renewenvironment{lstlisting}
    {\par\nolinenumbers\oldlstlisting}
    {\oldendlstlisting\endnolinenumbers\par}

\let\oldtable\table
\let\oldendtable\endtable
\renewenvironment{table}
    {\par\nolinenumbers\oldtable}
    {\oldendtable\endnolinenumbers\par}

% Package: listings

\lstMakeShortInline[style=MATLAB-editor, basicstyle=\mlttfamily]|

\lstset{
    breaklines=true,
    backgroundcolor=\color{lightgray},
    basicstyle=\scriptsize,
    numbers=left,
    numberstyle={\color{black!33}\scriptsize\sffamily},
    xleftmargin=2em,
    xrightmargin=2em
}

% Package: pgfplot

\pgfplotsset{width=7cm, compat=1.16}

% Package: varioref

\renewcommand{\reftextbefore}
    {on the \reftextvario{preceding page}{page before}}
\renewcommand{\reftextafter}
    {on the \reftextvario{following}{next} page}
\renewcommand{\reftextfacebefore}
    {on the \reftextvario{facing}{preceding} page}
\renewcommand{\reftextfaceafter}
    {on the \reftextvario{facing}{next}{page}}
\renewcommand{\reftextfaraway}[1]
    {on page \pageref{#1}}

% -------------------- General new commands --------------------

\DeclareMathAlphabet{\mathsfsl}{OT1}{cmss}{m}{sl}

\newcommand{\diff}{\mathop{}\!\mathrm{d}}
\newcommand{\matr}[1]{\ensuremath{\mathsfsl{#1}}} % italic sans serif
\newcommand{\me}{\mathrm{e}}
\newcommand{\mi}{\mathrm{i}}
\newcommand{\restrict[1]}{\raisebox{-.5ex}{$|$}_{#1}}

% -------------------- Specific new commands --------------------

\newcommand{\CityI}   {\stackrel[7.25]{7.20}{\text{北京}}}
\newcommand{\CityII}  {\stackrel[7.26]{7.21}{\text{上海}}}
\newcommand{\CityIII} {\stackrel[7.26]{7.22}{\text{广州}}}
\newcommand{\CityIV}  {\stackrel[7.28]{7.26}{\text{兰州}}}
\newcommand{\CityV}   {\stackrel[7.28]{7.26}{\text{成都}}}
\newcommand{\CityVI}  {\stackrel[7.31]{7.29}{\text{昆明}}}
\newcommand{\CityVII} {\stackrel[8.03]{8.01}{\text{南京}}}
\newcommand{\CityVIII}{\stackrel[8.04]{8.02}{\text{厦门}}}
\newcommand{\CityIX}  {\stackrel[8.06]{8.03}{\text{杭州}}}
\newcommand{\CityX}   {\stackrel[8.08]{8.06}{\text{济南}}}
\newcommand{\CityXI}  {\stackrel[8.09]{8.07}{\text{天津}}}
\newcommand{\CityXII} {\stackrel[8.10]{8.07}{\text{咸阳}}}
\newcommand{\CityXIII}{\stackrel[8.10]{8.08}{\text{大连}}}

% -------------------- Document --------------------

\begin{document}

    % -------------------- Title Page --------------------

    \maketitle
    \thispagestyle{empty}
    \pagenumbering{roman}

    % -------------------- Abstract Page --------------------

    \newpage
    % !TeX root = ../main.tex

% 中英文摘要和关键字

\begin{abstract}{代数几何, 交换代数, 准素分解, 维数理论}
  这篇文章主要面向没有接触过但想要了解代数几何的本科生, 旨在以尽可能少的前置知识向读者自洽地展示代数几何的基础.

  代数几何的理论根基在于代数. 本文从基本定义开始建立了以环论为主模论为辅的交换代数理论, 研究了商环与分式环的基本性质, 证明了Noether环上准素分解的存在性及其满足的唯一性, 以域论为基础利用Noether正规化引理证明了域的有限生成整环上的维数定理和Hilbert零点定理.

  代数几何的研究对象在于几何. 本文研究了固定代数闭域上的仿射与射影空间中的代数集, 建立了根式理想与代数集之间的对应, 一次将准素分解理论与几何相联系. 本文还讨论了代数簇上的函数结构, 以此将维数理论应用到几何中, 并证明了两组代数范畴与几何范畴的等价. 最后本文简要介绍了概形的概念, 其相比于代数簇能更完整地体现代数所提供的信息. 读完本文, 读者可以初步掌握代数几何基础, 并做好进一步学习重要技术的准备.
\end{abstract}

\begin{abstract*}{algebraic geometry, commutative algebra, primary decomposition, dimension theory}
  \lipsum[1-2]
\end{abstract*}


    % -------------------- Contents --------------------

    \newpage
    \tableofcontents

    % -------------------- Body --------------------

    \newpage
    \pagestyle{fancy}
    \pagenumbering{arabic}
    \linenumbers
    
    \section{背景介绍} %重述

    写一本代数几何的入门书籍的困难之处在于如何在提供几何直观和例子的同时推导现代技术的语言. 对于代数几何来说, 作为学科起源的直观想法与现代研究中使用的技术方法有一道巨大的鸿沟.

首当其冲的问题就是语言. 代数几何这门学科经历了很多轮发展, 每一次都有独特的语言和看问题的观点. 十九世纪晚期代数几何有Riemann的函数论方法, 有Brill和Noether的更加几何的方法, 还有Kronecker, Dedekind和Weber的纯代数方法. 同时以Castelnuovo, Enriques和Severi为代表的意大利学派致力于代数曲面的分类. 随后二十世纪以Chow, Weil和Zariski为首的``美国"学派给意大利学派的直观提供了坚实的代数基础. 最近, Serre和Grothendieck创立了法国学派, 他们用概型和上同调的语言重写了代数几何, 并且用心的方法解决了非常多的就问题. 每一个学派都引入了很多新的概念和方法. 在写一本入门书的时候, 是用旧的语言写来贴近几何直观比较好, 还是直接从现代研究中所用的技术语言开始写比较好呢?

第二个问题是一个理念上的问题. 现代数学家倾向于抹去历史的总计: 每一个新的学派都用自己的语言重写这门学科的根基, 这样做有利于严谨性但是不利于教学. 如果一个人知道了概型的定义, 但是却没有意识到一个代数数域的整数环, 一条代数曲线和一个紧黎曼面都是一个``一维正则概型"的例子的话, 那又有什么用呢? 那么这样一本入门书籍的作者应该如何既讲明白代数几何来源于数论, 交换代数和复分析, 又给读者介绍这门学科的主要内容, 即仿射或射影空间上的代数簇, 同时推导概型和上同调这样的现代语言你呢? 有什么样的话题, 可以做到既传达代数几何的意义, 又能作为将来学习和研究的坚实基础呢?

\bigskip

我个人偏向于古典几何这一边. 我相信代数几何中最重要的问题就是那些从老派的仿射空间或射影空间的簇中引出的问题. 他们提供了激发所有后来发展的几何直观. 我以关于簇的一章开始本书, 以最简单的形式建立了一些例子和基本想法, 将它们从技术细节里解放出来. 只有在这些内容都介绍完之后, 我才能系统地建立概型, 凝聚层(coherent sheaves)以及上同调. 这些第二第三章的内容是这本书的技术核心. 在其中我试图陈述一些最重要的结论, 不过不追求一般性. 因此, 比如说上同调理论是针对N\"otherian概型上的拟凝聚层建立的, 因为这比较简单并且对于大多数应用来说已经足够强了; "顺像层(direct image sheaves)的凝聚性(coherence)"定理只证明了射影态射的情况, 并没有对一般的固有态射(proper morphism)进行证明. 因为相同的原因, 我没有引入可表函子(representable functors), 代数空间(algebraic spaces), 平展上同调(\'etale cohomology), sites以及拓扑斯(topoi)这些最抽象的概念.

第四第五章处理了古典的内容, 即非奇异的射影曲线和曲面, 但是运用了概型和上同调的技术. 我希望这些应用可以证明为了发展前两章中的技术所花费的努力是值得的.

关于代数几何的基本语言和逻辑根基, 我才用了交换代数. 它有个好处就是很精确. 并且, 通过在任意特征的域上进行研究, 我们可以获取一些基域是复数域这种古典情形下的洞见. 几年之前, 当Zariski试图编写代数几何的丛书时, 他还需要在书中自己推导需要用到的代数知识. 这项工作占据了全部工作的如此大一部分, 以致于他专门出版了一本只讲交换代数的书. 现在我们十分幸运已经有了很多出色的关于交换代数的书籍. 我关于代数的对策是在需要时引用纯代数的结论, 并给出证明的参考资料. 在书的最后列出了所有用到的代数结论.

原本我计划了完整的一系列附录 - 关于一些当今研究方向的简短介绍, 为了建立这本书的主要内容与研究的桥梁. 因为时间和篇幅有限只有三篇附录得以呈现在成书中. 我十分遗憾这本书中没能包含其余附录, 读者可以去阅读the Arcata volume, 其中有一些针对非专家的由专家所写的关于他们研究领域的文章. 此外, 关于代数几何的历史发展, 可以参考Dieudonn\'e的书. 因为没有足够多的篇幅去像我所希望的那样探索代数几何与相邻领域之间的关系, 可以参考Cassels的关于与数论关系的综述文章, 也可以参考Shafarevich的关于与复流形和拓扑的综述文章.

因为我相信主动学习是一种好的学习方法, 书中有分厂多的习题. 有一些习题包含了正文中没有介绍的重要结论. 其余习题包含了一些能阐释一般理论的具体例子. 我相信对于例子的研究与发展一般理论之间有着不可分割的关系. 认真的学生应该尝试尽可能多地做这些习题, 但是不应该觉得能立即解出他们. 有不少习题需要一些真正有创造性的努力才能够理解. 一个星号表示这道习题是困难的, 两个星号表示这道习题是一个未解决的问题.

(I, \S 8)中有关于代数几何和这本书的进一步介绍.

\subsection*{术语}

大部分情况, 书中的属于与广泛接受的用法是相同的, 不过还是有一些值得注意的例外. \emph{簇}一直是不可约的, 并且一直是在代数闭域上的. 在第一张中所有的簇都是拟仿射的. 在(II, \S 4)中簇的定义被拓展为包括\emph{抽象簇}, 即为代数闭域上的integral separated schemes of finite type, 词语\emph{曲线}, \emph{曲面}和\emph{3-fold}分别用来表示1维, 2维和3维的簇. 但是在第四章中, 词语\emph{曲线}只用来表示非奇异的射影曲线; 在第五章中\emph{曲线}表示任何非奇异射影曲面上的有效除子(effective divisor). 第五章中\emph{曲面}表示非奇异射影曲面.

书中的\emph{概型}在第一版的EGA中被称为预概型(prescheme), 不过在新版EGA中被称为概型.

书中\emph{射影态射}和\emph{very ample invertible sheaf}的定义与EGA中的定义并不等价. 他们在技术上比较简单, 但是有一个缺点就是它并不是基上的局部概念.

词语\emph{非奇异}只对簇使用, 对于一般的概型来说, 我们采用\emph{正则}和\emph{光滑}.

\subsection*{代数的结论}

我假设读者熟悉环, 理想, 模, N\"otherian环, 整相关(integral dependence)的基础知识, 并且乐意接受或者查询其它属于交换代数或者同调代数结论, 这些结论如果需要的话会在书中进行陈述, 伴有相关文本的引用. 这些结论会被标注一个A, 比如说定理3.9A, 为了和书中证明的结论进行区分.

基本的约定有这些: 所有的环都是交换幺环, 单位元记作1. 所有的环同态都将1映到1. 在整环或者域中, $0\neq 1$. 一个\emph{素理想}(或者极大理想)是环$A$的一个理想$\ideal{p}$, 满足商环$A/\ideal{p}$是一个整环(或者域). 因此环本身不被认为是一个素理想或者是极大理想.

环$A$中的一个\emph{乘性系统}(multiplicative system)是一个包含1的子集$S$, 满足关于乘法封闭. \emph{局部化}$S^{-1}A$定义为分式$a/s, a\in A, s\in S$在等价关系下的商, 其中$a/s$与$a'/s'$\emph{等价}仅当存在$s''\in S$使得$s''(s'a-sa')=0$成立. 有两种一直用的局部化列举如下. 设$\ideal{p}$是$A$中的素理想, 那么$S = A - \ideal{p}$是一个乘性系统, 相对应的局部化被记为$A_{\ideal{p}}$. 如果$f$是$A$的元素, 那么$S = \{1\}\cup \{f^n\vert n\geq 1\}$是一个乘性系统, 相对应的局部化被记为$A_f$. (注意在$f$是幂零元的情况下, $A_f$是零环.)

\subsection*{引用}

关于定理, 命题, 引理的交叉引用用圆括号以及数字, 例如(3.5). 对习题的引用例如(习题3.5). 对另一章节的结论的引用以章节数字打头, 例如(II, 3.5)或(II, 习题3.5)


    \section{模型假设} %分析

    \begin{itemize}
        \item 仅在老师连续不间隔地参加两个不同的会时考虑交通方式的时间因素,
                  由于此时不存在两地距离很近的情况, 故一律乘坐飞机.
        \item 会场方在会议开始之前及开始以后的一段时间内也提供住宿,
                  即允许提前到达会场城市, 或者延后离开会场城市.
        \item 会议开始前的一晚上必须住在会场城市.
        \item 计算过程中向下取整.
        \item 将老师按职称及职务大小由大到小排序.
    \end{itemize}

    \section{符号介绍}

    \begin{table}[H]
        \begin{center}
            \begin{edtable}{tabular}{cc}
                \Xhline{1.2pt}
                符号                  &   含义\\
                \hline
                粗斜体字母            &   向量\\
                $t_{i}$               &   第$i$位老师参加会议数的最低要求\\
                $c_{i,1}$             &   第$i$个会议中参会总人数的最低要求\\
                $c_{i,2}$             &   第$i$个会议中参会教授与副教授人数和的最低要求\\
                $c_{i,3}$             &   第$i$个会议中参会教授人数的最低要求\\
                $s_{i}$               &   第$i$个会议的星级\\
                $\boldsymbol{e_{i}}$  &   单位矩阵第$i$列的列向量\\
                \Xhline{1.2pt}
            \end{edtable}
        \end{center}
    \end{table}

    \section{基于动态规划的费用最优化模型}

    \subsection{模型建立}
    任务一要求我们在满足各会议参会人数最低要求(约束一)和各老师参会数量最低要求(约束二)的情况下,
    给出总花费最低的参会安排.
    我们发现了如下几条性质:
    \begin{enumerate}
        \item 任何人以相同的交通方式前往相同的(多个)会议, 花费均相同.
        \item 满足约束一的最低老师参会人次$(n_{1,i})$
                均少于该类老师满足约束二的最低老师参会人次$(n_{2,i}$),
                $i=1,2,3$\\
                因为
                
                \begin{gather*}
                    n_{1,1}=\sum_{i=1}^{13} c_{i,1}=28<36=\sum_{i=1}^{18} t_{i}=n_{2,1}\\
                    n_{1,2}=\sum_{i=1}^{13} c_{i,2}=14<26=\sum_{i=1}^{13} t_{i}=n_{2,2}\\
                    n_{1,3}=\sum_{i=1}^{13} c_{i,3}=5 <10=\sum_{i=1}^{ 5} t_{i}=n_{2,3}.\\
                \end{gather*}
        \item 所有的老师参加且仅参加两场会议.\\
                {\itshape 注: 该性质为第二条性质的直接推论}.
    \end{enumerate}

    因为得到了每个老师参会的确切数目, 并且以对老师进行了排序,
    我们将已经确定所参加会议的老师数量作为阶段, 共分为18个阶段.

    由于最终要求解的是花费的最小值, 我们定义\textbf{状态}为$\lambda(k,\boldsymbol{\alpha})$,
    其中
    $\boldsymbol{\alpha}=(a_{1}, a_{2}, a_{3}, a_{4}, a_{5}, a_{6},
    a_{7}, a_{8}, a_{9}, a_{10}, a_{11}, a_{12}, a_{13}),$
    表示前$k$位老师参加会议$1, 2, \dotsc, 13$分别达到人数$a_{1}, a_{2}, \dotsc, a_{13}$的要求时
    的最低费用.

    说明: 由于当$a_{i}$达到约束条件之后, 值的增加不会影响规划的结果,
    因此在以下的计算中, 一旦参加第$i$个会议的人数大于$c_{i, 1}$, 我们都令$a_{i}=c_{i, 1}$.

    设在阶段$i$的所有状态为$\lambda(i, \boldsymbol{\alpha_{1}})$,
    在阶段$i+1$的所有状态为$\lambda(i+1, \boldsymbol{\alpha_{2}})$,
    则第$i+1$位老师的\textbf{决策}可以是在13个会议中任选不冲突两场,
    并且第$i+1$位老师的所有决策, 就是从阶段$i$到阶段$i+1$之间的\textbf{策略}。

    由上, 设第$i+1$位老师选择了前往会议$k$和$l$的决策($k<l$),
    那么$\boldsymbol{\alpha_{1}}$与$\boldsymbol{\alpha_{2}}$满足
    \[\boldsymbol{\alpha_{2}}=\boldsymbol{\alpha_{1}}+\boldsymbol{e_{k}^T}+\boldsymbol{e_{l}^T}.\]
    设$W(k,l)$表示一个人通过各种交通方式从同济前往会议$k$,
    再前往会议$l$(或留宿在城市$k$, 或提前到达城市$l$, 或先回到同济),
    最后再回到同济的整个过程所花费的最小费用.
    于是我们可以得到\textbf{状态转移方程}:
    \[\lambda(i+1, \boldsymbol{\alpha_{2}})=\min\{\lambda(i, \boldsymbol{\alpha_{1}})+W(k,l)\}.\]
    
\subsection{模型求解}
    我们先定义初始状态$\lambda(0, \boldsymbol{0})=0$,
    表示没有人参会, 并且会议参加数均为0人的最小花费为0元.
    
    接着从初始状态开始, 枚举出所有这一阶段所能采取的决策$(k,l)$,
    通过状态转移方程
    \[\lambda(i+1, \boldsymbol{\alpha_{2}})=\min\{\lambda(i, \boldsymbol{\alpha_{1}})+W(k,l)\}\]
    递推出下一个阶段, 状态为$\boldsymbol{\alpha_{2}}$时的最低花费,
    最终得出目标阶段的目标状态$\lambda(13, \boldsymbol{c_{1}})$时的最低花费
    (其中$\boldsymbol{c_{1}}$为各会议参会总人数最低要求向量).
    
    此时我们没有把会议对不同职称老师的限制考虑进来.
    在实际计算中, 我们先对教授进行状态转移, 即阶段$1-5$, 在阶段5结束之后,
    检查各状态是否满足教授的参会人数最低要求, 删去其中不满足要求的状态
    ($a_{1}<2\vee a_{2}<1\vee a_{12}<1\vee a_{13}<1$).
    
    同样的, 在阶段13结束之后, 我们也检查各状态是否满足教授与副教授的和人数的最低要求,
    并删去不满足要求的状态($\exists i(a_{i}<1\wedge i\in ([3, 11]\cap \mathcal{Z})$).
    
    这样, 我们便可以用如上方法, 通过递推来求解使得总费用最低的参会安排.
    结果算得最低总花费为89232元, 具体参会安排如下:
    
    \begin{scriptsize}
        \begin{align*}
            \text{主任:}&\ \textit{同济}
                \xrightarrow[7.19]{\text{高铁}}\ \CityI\
                \xrightarrow[7.25]{\text{飞机}}\ \CityIV\
                \xrightarrow[7.29]{\text{高铁}} \textit{同济}
                &\quad\text{花费8446元}\\
            \text{副主任:}&\ \textit{同济}
                \xrightarrow[7.20]{\text{高铁}}\ \CityI\
                \xrightarrow[7.25]{\text{飞机}}\ \CityIV\
                \xrightarrow[7.29]{\text{高铁}} \textit{同济}
                &\quad\text{花费8446元}\\
            \text{教授A:}&\ \textit{同济}
                \xrightarrow[7.19]{\text{高铁}}\ \CityI\
                \xrightarrow[7.26]{\text{高铁}} \textit{同济}
                \xrightarrow[7.31]{\text{高铁}}\ \CityVII\
                \xrightarrow[8.04]{\text{高铁}} \textit{同济}
                &\quad\text{花费7592元}\\
            \text{教授B:}&\ \textit{同济}
                \xrightarrow{\hspace{13pt}}\ \CityII\
                \xrightarrow[8.06]{\text{高铁}}\ \CityXII\
                \xrightarrow[8.11]{\text{高铁}} \textit{同济}
                &\quad\text{花费4388元}\\
            \text{教授C:}&\ \textit{同济}
                \xrightarrow{\hspace{13pt}}\ \CityII\
                \xrightarrow[8.07]{\text{飞机}}\ \CityXIII\
                \xrightarrow[8.11]{\text{高铁}} \textit{同济}
                &\quad\text{花费3634元}\\
            \text{副教授A:}&\ \textit{同济}
                \xrightarrow{\hspace{13pt}}\ \CityII\
                \xrightarrow[7.31]{\text{高铁}}\ \CityVII\
                \xrightarrow[8.04]{\text{高铁}} \textit{同济}
                &\quad\text{花费2804元}\\
            \text{副教授B:}&\ \textit{同济}
                \xrightarrow[7.21]{\text{高铁}}\ \CityIII\
                \xrightarrow[7.27]{\text{高铁}} \textit{同济}
                \xrightarrow[8.01]{\text{高铁}}\ \CityVIII\
                \xrightarrow[8.05]{\text{高铁}} \textit{同济}
                &\quad\text{花费7274元}\\
            \text{副教授C:}&\ \textit{同济}
                \xrightarrow[7.21]{\text{高铁}}\ \CityIII\
                \xrightarrow[7.27]{\text{高铁}} \textit{同济}
                \xrightarrow[8.01]{\text{高铁}}\ \CityVIII\
                \xrightarrow[8.05]{\text{高铁}} \textit{同济}
                &\quad\text{花费7274元}\\
            \text{副教授D:}&\ \textit{同济}
                \xrightarrow[7.25]{\text{高铁}}\ \CityV\
                \xrightarrow[7.28]{\text{高铁}}\ \CityVI\
                \xrightarrow[8.01]{\text{高铁}} \textit{同济}
                &\quad\text{花费5510元}\\
            \text{副教授E:}&\ \textit{同济}
                \xrightarrow[7.25]{\text{高铁}}\ \CityV\
                \xrightarrow[7.28]{\text{高铁}}\ \CityVI\
                \xrightarrow[8.01]{\text{高铁}} \textit{同济}
                &\quad\text{花费5510元}\\
            \text{副教授F:}&\ \textit{同济}
                \xrightarrow{\hspace{13pt}}\ \CityII\
                \xrightarrow[8.02]{\text{高铁}}\ \CityIX\
                \xrightarrow[8.07]{\text{高铁}} \textit{同济}
                &\quad\text{花费3238元}\\
            \text{副教授G:}&\ \textit{同济}
                \xrightarrow{\hspace{13pt}}\ \CityII\
                \xrightarrow[8.05]{\text{高铁}}\ \CityX\
                \xrightarrow[8.09]{\text{高铁}} \textit{同济}
                &\quad\text{花费3292元}\\
            \text{副教授H:}&\ \textit{同济}
                \xrightarrow{\hspace{13pt}}\ \CityII\
                \xrightarrow[8.06]{\text{高铁}}\ \CityXI\
                \xrightarrow[8.10]{\text{高铁}} \textit{同济}
                &\quad\text{花费3636元}\\
            \text{讲师A:}&\ \textit{同济}
                \xrightarrow{\hspace{13pt}}\ \CityII\
                \xrightarrow[8.02]{\text{高铁}}\ \CityIX\
                \xrightarrow[8.07]{\text{高铁}} \textit{同济}
                &\quad\text{花费3238元}\\
            \text{讲师B:}&\ \textit{同济}
                \xrightarrow{\hspace{13pt}}\ \CityII\
                \xrightarrow[8.05]{\text{高铁}}\ \CityX\
                \xrightarrow[8.09]{\text{高铁}} \textit{同济}
                &\quad\text{花费3292元}\\
            \text{讲师C:}&\ \textit{同济}
                \xrightarrow{\hspace{13pt}}\ \CityII\
                \xrightarrow[8.06]{\text{高铁}}\ \CityXI\
                \xrightarrow[8.10]{\text{高铁}} \textit{同济}
                &\quad\text{花费3636元}\\
            \text{讲师D:}&\ \textit{同济}
                \xrightarrow{\hspace{13pt}}\ \CityII\
                \xrightarrow[8.06]{\text{高铁}}\ \CityXII\
                \xrightarrow[8.11]{\text{高铁}} \textit{同济}
                &\quad\text{花费4388元}\\
            \text{讲师E:}&\ \textit{同济}
                \xrightarrow{\hspace{13pt}}\ \CityII\
                \xrightarrow[8.06]{\text{飞机}}\ \CityXIII\
                \xrightarrow[8.11]{\text{高铁}} \textit{同济}
                &\quad\text{花费3634元}
        \end{align*}
    \end{scriptsize}

    参加各会议的不同职称老师的人数如下表, 可以看出结果满足题目的约束条件.
    
    \begin{table}[htb]\scriptsize
        \begin{center}
            \caption{任务一中参加各会议的老师情况}
            \begin{tabular}{cccc}
                \Xhline{1.2pt}
                城市 & 实际总人数/要求总人数 & 实际(副)教授人数/要求(副)教授人数 &
                    实际教授人数/要求教授人数\\
                \hline
                北京 & 3/3 & 3/2 & 3/2\\
                上海 & 11/3 & 6/1 & 2/1\\
                广州 & 2/2 & 2/1 & 0/0\\
                兰州 & 2/2 & 2/1 & 2/0\\
                成都 & 2/2 & 2/1 & 0/0\\
                昆明 & 2/2 & 2/1 & 0/0\\
                南京 & 2/2 & 2/1 & 1/0\\
                厦门 & 2/2 & 2/1 & 0/0\\
                杭州 & 2/2 & 1/1 & 0/0\\
                济南 & 2/2 & 1/1 & 0/0\\
                天津 & 2/2 & 1/1 & 0/0\\
                咸阳 & 2/2 & 1/1 & 1/1\\
                大连 & 2/2 & 1/1 & 1/1\\
                \Xhline{1.2pt}
            \end{tabular}
        \end{center}
    \end{table}
    \clearpage
\subsection{模型复杂度计算}
    用C++实现本模型时所占用的内存为\[18\times 4^2\times3^{11}\times4=204073344\text{Byte},\]
    约为194.62MB. 运算次数约为$18\times 4^2\times 3^{11}\times C_{13}^2)\approx 3.98\times 10^9$,
    远远优于暴力枚举算法, 程序运行时间在50s左右, 在可接受范围内.
\subsection{模型检验与评价}
    \subsubsection{模型检验}
        从求解结果可以看出, 一旦各会议参会人数最低要求被满足之后,
        未达到参会数量最低要求的老师均选择参加上海的会议以凑满要求, 这是符合我们的预期的.
        因为所要规划的是最低总花费, 而因为在上海开会无需住宿费和长途交通费,
        所以花费比去其它城市开会要明显低很多.
    \subsubsection{模型评价}
        \subsubsubsection{优点}
            \begin{itemize}
                \item 模型中, 我们考虑到了老师在某一城市开完会, 准备前往下一个城市开会时,
                        有三种可能的决策.
                        \begin{enumerate}
                            \item 不做停留, 直接前往下一个城市.
                            \item 在当前城市留宿若干天, 直到下一场会议开会前一天, 再前往下一个城市.
                            \item 先回到上海, 之后再前往下一个城市开会.
                        \end{enumerate}
                        
                        因此模型得出的具体出行安排会很明确, 能确定老师出行的每一个时间点.
                        
                \item 我们通过推理得出了题目中老师参会数量的重要性质, 为固定值, 大幅简化了模型.
            \end{itemize}
        \subsubsubsection{缺点}
            \begin{itemize}
                \item 我们没有对会议结束的当天, 老师具体的住宿地点进行详细的讨论, 会带来一定的误差.
                \item 本模型仅适用于每位老师参加会议数量为固定值的情况, 如果该值可在小范围内浮动,
                        都会大大增加算法所需运算量, 使得难以算出结果.
            \end{itemize}

    \section{基于分组背包的动态规划影响力最优化模型}

    \subsection{问题分析}
    任务二中, 我们需要得出预算在5万元的情况下, 老师们总影响力最大时的参会安排.
    针对每一位老师, 令他参加的所有会议(不包括不参加会议)为这位老师的决策, 记为$x$,
    其中决策的个数是有限的.
    不同职称的老师做出相同的决策, 花费是相同的, 但是他们所产生的影响力, 即价值是不同的.
    于是问题可以转化为分组背包模型.
    
    所谓分组背包模型,即在若干组中, 每组至多只取一件物品, 在一定的费用内使得物品价值最优化的模型.
    
\subsection{物品组与物品}
    每个老师都可以当作一个背包, 每一个决策都相当于他背包中的一个物品,
    该决策所花费的注册费, 报名费以及路费是该物品的花费,
    而每一个决策贡献的影响力看作该物品的价值.

    那么在18个背包内取至多一个物品, 使得花费在50000元以内, 并且物品价值得到最优化,
    就是我们规划的目标.

    通过计算机预处理, 在会议日程不冲突的前提下, 对于每一个人, 其参加会议所有可能的组合均只有303种, 对应着每个组内的303件物品(见附录).
    
\subsection{物品的费用}
    定义第$i$个背包中的第$j$件物品(表示第$i$个人的第$j$种参会组合)的费用
    \[V(i, j):=W(x_{j}),\]
    其中, $W(x_{j})$为前往第$j$中参会组合中会议的所有费用的最小值, 并且$V(i, j)$的取值与$i$无关.

    例如$V(1 ,2)$表示第1个人, 前往第2种参会组合中的会议时的费用,
    即主任前往参会组合(1, 4), 代表北京$\rightarrow$兰州,
    则$V(1, 2)=W(1, 4).$
    
\subsection{物品的价值}
    当参会组合相同时, 不同职称的老师贡献的影响力却不同, 因此我们需要参数来描述职称对价值的影响,
    为此我们引入了h指数(h-index)的概念.
    
    h指数由物理学家Jorge E. Hirsch在2005年提出, 旨在用以比较理论物理学家的学术能力.\citep{J.E.Hirsch_2005}
    如果$f$为从发表的文章到引用次数的函数, 那么我们先将函数值按照从大到小排序, 那么h指数可由如下公式计算
    \citep{wiki_h-index}
    \[\text{\textit{h} -- index} (f) = {\displaystyle \max _{i}\min(f(i),i)}.\]
    
    Hirsch认为对于科学家, h指数达到12的可以在美国的主要研究型大学担任副教授职务,
    h指数在15-20的可以成为美国物理学会的成员, 
    h指数在45以上的可以成为美国国家科学院的主要成员.\citep{Peterson_2005}
    
    我们以此为基本, 定义了影响力指数$H$, 记$H_{i}$为第$i$个人的$H$值,
    并令主任和副主任的$H$值为45,
    普通教授的$H$值为$\lfloor(45+20)/2\rfloor=32$,
    副教授的$H$值为$(20+12)/2=16$,
    讲师的$H$值为$(12+0)/2=6$.
    
    再令$s_{j}$为第$j$场会议的星级,
    并定义第$i$个人参加第$k$场会议的价值为$H_{i}\times s_{k}$,
    一个背包内一件物品的价值即为参加该组合每场会议的影响力之和,
    记为$\iota_{i, k}$.
    
    例如$\iota_{1, 2}$表示第1个人, 前往第2种参会组合中的会议时的费用, 即前文的组合(1, 4),
    则\[\iota_{1,2}=s_{1}\times H_{1}+s_{4}\times H_{1}=5\times 45+3\times 45=315.\]
    
\subsection{状态及状态转移方程}
    定义状态$\lambda(k, \xi)$表示: 只有前$k$组的人员参加会议,
    每组只能选择至多一件物品(每人选择一种参会组合或选择不参会),
    且费用小于等于$\xi$所产生的最大影响力.

    假设当$k\le n-1$时的状态均已算出, 那么对于第$n$组, 有以下几种策略:
    \begin{enumerate}
        \item 在第$n$组中不选择任何物品(第$n$个人不参加任何会议)
        \item 在第$n$组中选择任意选择一件物品(第$n$个人任意选择一种参会组合)
    \end{enumerate}
    
    于是计算状态$\lambda(n, \xi)$可分解成一下几个子问题:
    \begin{itemize}
        \item 对应于策略1, 此时的影响力就等于上一个状态$k=n-1$时的影响力,
                即$\lambda(n, \xi)=\lambda(n-1, \xi).$
        \item 对应于策略2, 假设从第$n$组的背包选择了物品$j$(第$j$组参会方案),
                产生了$V(n, j)$的花费,
                因此只给前$n-1$组的背包留下了$\xi -V(n, j)$的花费,
                于是此时的最大影响力等于前$n-1$组在花费为$\xi -V(n, j)$以内时的最大影响力
                加上从第$n$组背包选择物品$j$所产生的影响力, 即
                \[\lambda(n, \xi)=\lambda(n-1, \xi -V(n, j))+\iota_{n, j}.\]
    \end{itemize}
    
     由此我们便可以得出状态转移方程:
     \[\lambda(n, \xi)=\max \{\max_{1\le i\le 303} \{\lambda(n-1, \xi -V(n, j))+\iota_{n, j}\}, 
        \lambda(n-1, \xi)\}.\]

    目标状态即为$\lambda(18, 50000)$, 
    表示前18组的人员参加会议, 且每组只能选择至多一中参会组合, 
    花费小于等于50000所产生的最大影响力.
\subsection{求解结果}
    对模型进行运算求解, 得出总影响力的最优值为3438, 此时的总花费为49754元, 具体的参会安排如下:
    
    \begin{scriptsize}
        \begin{align*}
            \text{主任:}&\ \textit{同济}
                \xrightarrow{\hspace{13pt}}\ \CityII\
                \xrightarrow[7.28]{\text{高铁}}\ \CityVI\
                \xrightarrow[8.01]{\text{高铁}}\ \CityVIII\
                \xrightarrow[8.05]{\text{高铁}} \textit{同济}
                \xrightarrow[8.05]{\text{飞机}}\ \CityX\
                \xrightarrow[8.09]{\text{高铁}} \textit{同济}
                &\quad\text{花费9618元}\\
            \text{副主任:}&\ \textit{同济}
                \xrightarrow{\hspace{13pt}}\ \CityII\
                \xrightarrow[7.28]{\text{高铁}}\ \CityVI\
                \xrightarrow[8.01]{\text{高铁}}\ \CityVIII\
                \xrightarrow[8.05]{\text{高铁}} \textit{同济}
                \xrightarrow[8.05]{\text{飞机}}\ \CityX\
                \xrightarrow[8.09]{\text{高铁}} \textit{同济}
                &\quad\text{花费9618元}\\
            \text{教授A:}&\ \textit{同济}
                \xrightarrow{\hspace{13pt}}\ \CityII\
                \xrightarrow[8.01]{\text{高铁}}\ \CityVIII\
                \xrightarrow[8.05]{\text{高铁}} \textit{同济}
                \xrightarrow[8.05]{\text{飞机}}\ \CityX\
                \xrightarrow[8.09]{\text{高铁}} \textit{同济}
                &\quad\text{花费5926元}\\
            \text{教授B:}&\ \textit{同济}
                \xrightarrow{\hspace{13pt}}\ \CityII\
                \xrightarrow[8.01]{\text{高铁}}\ \CityVIII\
                \xrightarrow[8.05]{\text{高铁}} \textit{同济}
                \xrightarrow[8.05]{\text{飞机}}\ \CityX\
                \xrightarrow[8.09]{\text{高铁}} \textit{同济}
                &\quad\text{花费5926元}\\
            \text{教授C:}&\ \textit{同济}
                \xrightarrow{\hspace{13pt}}\ \CityII\
                \xrightarrow[8.01]{\text{高铁}}\ \CityVIII\
                \xrightarrow[8.05]{\text{高铁}} \textit{同济}
                \xrightarrow[8.05]{\text{飞机}}\ \CityX\
                \xrightarrow[8.09]{\text{高铁}} \textit{同济}
                &\quad\text{花费5926元}\\
            \text{副教授A:}&\ \textit{同济}
                \xrightarrow{\hspace{13pt}}\ \CityII\
                \xrightarrow{\hspace{13pt}}\textit{同济}
                &\quad\text{花费980元}\\
            \text{副教授B:}&\ \textit{同济}
                \xrightarrow{\hspace{13pt}}\ \CityII\
                \xrightarrow{\hspace{13pt}}\textit{同济}
                &\quad\text{花费980元}\\
            \text{副教授C:}&\ \textit{同济}
                \xrightarrow{\hspace{13pt}}\ \CityII\
                \xrightarrow{\hspace{13pt}}\textit{同济}
                &\quad\text{花费980元}\\
            \text{副教授D:}&\ \textit{同济}
                \xrightarrow{\hspace{13pt}}\ \CityII\
                \xrightarrow{\hspace{13pt}}\textit{同济}
                &\quad\text{花费980元}\\
            \text{副教授E:}&\ \textit{同济}
                \xrightarrow{\hspace{13pt}}\ \CityII\
                \xrightarrow{\hspace{13pt}}\textit{同济}
                &\quad\text{花费980元}\\
            \text{副教授F:}&\ \textit{同济}
                \xrightarrow{\hspace{13pt}}\ \CityII\
                \xrightarrow{\hspace{13pt}}\textit{同济}
                &\quad\text{花费980元}\\
            \text{副教授G:}&\ \textit{同济}
                \xrightarrow{\hspace{13pt}}\ \CityII\
                \xrightarrow{\hspace{13pt}}\textit{同济}
                &\quad\text{花费980元}\\
            \text{副教授H:}&\ \textit{同济}
                \xrightarrow{\hspace{13pt}}\ \CityII\
                \xrightarrow{\hspace{13pt}}\textit{同济}
                &\quad\text{花费980元}\\
            \text{讲师A:}&\ \textit{同济}
                \xrightarrow{\hspace{13pt}}\ \CityII\
                \xrightarrow{\hspace{13pt}}\textit{同济}
                &\quad\text{花费980元}\\
            \text{讲师B:}&\ \textit{同济}
                \xrightarrow{\hspace{13pt}}\ \CityII\
                \xrightarrow{\hspace{13pt}}\textit{同济}
                &\quad\text{花费980元}\\
            \text{讲师C:}&\ \textit{同济}
                \xrightarrow{\hspace{13pt}}\ \CityII\
                \xrightarrow{\hspace{13pt}}\textit{同济}
                &\quad\text{花费980元}\\
            \text{讲师D:}&\ \textit{同济}
                \xrightarrow{\hspace{13pt}}\ \CityII\
                \xrightarrow{\hspace{13pt}}\textit{同济}
                &\quad\text{花费980元}\\
            \text{讲师E:}&\ \textit{同济}
                \xrightarrow{\hspace{13pt}}\ \CityII\
                \xrightarrow{\hspace{13pt}}\textit{同济}
                &\quad\text{花费980元}
        \end{align*}
    \end{scriptsize}
    
\subsection{模型复杂度计算}    
    用C++实现本模型时所占用的内存为$18\times 50000\times 4=3600000\text{Byte}$,
    约为3.43MB, 是比较优秀的.
    运算次数约为$18\times 50000\times 303= 2.7\times 10^8$,
    程序运行时间在5s左右, 也是比较优秀的.

\subsection{模型检验与评价}
    \subsubsection{模型检验}
        进行模型求解之前,我们比对了前往会议时间相似的若干组城市的花费与影响力,
        发现上海会议的影响力大于广州会议, 且费用远比其便宜,
        故所得结果中应有较多的老师前往上海会议, 较少的老师前往广州会议.
        同理, 比较南京会议与杭州会议, 济南会议与大连会议, 天津会议与咸阳会议, 
        还可以得出应有较少老师前往杭州, 大连与天津会议的结论.
        
        观察模型的计算结果, 确实与我们之前的判断相符, 故该模型不存在大的谬误.
        
        \subsubsubsection{误差分析}
            模型的误差主要来自影响力计算, 由于没有不同职称老师的学科影响力的定量数据,
            我们简单地以推荐h指数作为影响力指数, 由于没有会议等级与彰显同济大学学科影响力的定量关系,
            我们简单地以影响力指数与会议星级相乘在相加的结果作为所要规划的影响力,
            而没有对会议的星级指数进行调整.
        \subsubsubsection{敏感度分析}
            我们分别修改了会议的星级指数和老师的影响力指数, 重新进行求解.
            
            若固定影响力指数, 只将不同星级会议的差距拉大,
            即取星级指数为$2^s$, 其中$s$为星级,
            则运行结果没有发生任何区别,
            可见本模型对会议星级指数的变化不敏感.
            
            若固定会议的星级指数, 只改变影响力指数,
            略微提高副教授的影响力指数至18, 模型解的变化为有几位讲师放弃参加任何会议,
            而省出经费使得一位副教授可以出席南京的会议, 结果变化较小.
            若略微降低(副)主任或普通教授的影响力指数, 或者略微提高讲师的影响力指数,
            均不会对模型的结果造成影响, 故本模型对影响力指数的变化也不敏感.
    \subsubsection{模型评价}
        \subsubsubsection{优点}
            \begin{itemize}
                \item 本模型将问题抽象为分组背包的动态规划模型, 使得时间复杂度大幅降低,
                        可以在十秒钟左右计算出结果.
            \end{itemize}
        \subsubsubsection{缺点}
            \begin{itemize}
                \item 本模型只适用于每一位老师的影响力相互独立的情况, 在任务三便失效了.
                \item 未考虑第一题中(副)主任参会的上限限制, 使得模型结果可能不符合实际情况.
            \end{itemize}

    \section{基于贪心算法的影响力期望最优化模型}

    \subsection{问题分析}
    \begin{enumerate}
        \item 题目给出了``若参加同一地点会议的(至少)两人中,有一人的学术报告选为大会报告的概率'',
                我们理解为至少有一人的学术报告选为大会报告的概率.
                对于参会安排最优的评价, 我们只有这个概率和会议星级的数据,
                因此需以此建立一个参数来评判参会安排的优劣性.
        \item 于是, 如果参会人数多于两人, 多出的职称最低的一人不能对选为大会报告的期望作出贡献,
                并且每个会议的参会人数最低要求均不小于两人,
                因此, 参加各个会议的人数应恰好为最低要求.
        \item 经初步观察, 如果两位教授和两位副教授分别组队, 参加两个不同的会议,
                那么有一人的学术报告被选为大会报告的概率分别为0.75和0.35, 相加得1.10;
                如果两队均为一位教授和一位副教授的组合, 那么概率之和为1.00,
                有$1.10>1.00$.
                同理, 我们可得出副教授与讲师的组合中, 副教授和讲师分别组队优于交叉组队.
                当然, 这样的推导只是直观上的感觉, 是不严谨的, 我们将在下文给出具体的证明.
    \end{enumerate}

\subsection{模型建立}
    \subsubsection{影响力期望}
        与上一个模型相同, 我们依旧直接用会议的星级$s_{i}$表征会议的影响力,
        报告被选为大会报告的概率$p_{i}$即为产生影响力的概率, 
        因此影响力总和的期望$E=\sum(s_{i})\times p_{i}.$
    \subsubsection{贪心优化策略}
        根据问题分析中的第二点和第三点, 我们得出了两个基本的贪心优化策略:
        \begin{enumerate}[(i)]
            \item 每场会议的参加人数恰好为会议要求最低总人数
            \item 尽量使教授同教授一起组队, 副教授与副教授一起组队
        \end{enumerate}
        
        在这里我们给出(ii)的证明, 以两位教授及两位副教授两两之间组队参会为例.
        
        \begin{proof}[证明:]
            设两个会议的影响力分别为$\sigma_{1}$, $\sigma_{2}$, 其中$\sigma_{1}\ge \sigma_{2}>0$.
            设教授与副教授分别组队的影响力期望为$E_{1}$,
            教授与副教授交叉组队的影响力期望为$E_{2}$, 则
            \begin{align}
                E_{1}&=\frac{3}{4}\sigma_{1}+\frac{7}{20}\sigma_{2}\\
                E_{2}&=\frac{1}{2}(\sigma_{1}+\sigma_{2}).
            \end{align}
            因此
            \begin{equation}
                E_{1}-E_{2}=\frac{1}{4}\sigma_{1}-\frac{3}{20}\sigma_{2}
                =\frac{3}{20}(\sigma_{1}-\sigma_{2})+\frac{1}{10}\sigma_{1}>0
            \end{equation}
        \end{proof}
    \subsubsection{基本贪心策略}
        依据贪心优化策略(ii), 我们尽量避免不同职称之间老师交叉组队的情况, 列出下表.
        \begin{table}[htb]\footnotesize
            \begin{center}
            \caption{贪心模型老师组合表}
                \begin{tabular}{ccc}
                    \Xhline{1.2pt}
                    序号    &    老师1    &    老师2\\
                    \hline
                    1       &    主任     &    副主任\\
                    2       &    教授A    &    教授B\\
                    3       &    教授C    &    副教授A\\
                    4       &    副教授B  &    副教授C\\
                    5       &    副教授D  &    副教授E\\
                    6       &    副教授F  &    副教授G\\
                    7       &    副教授H  &    讲师A\\
                    8       &    讲师B    &    讲师C\\
                    9       &    讲师D    &    讲师E\\
                    \Xhline{1.2pt}
                \end{tabular}
            \end{center}
        \end{table}
        
\subsection{模型求解与评价}
    \subsubsection{算法实现}
        \begin{enumerate}[Step 1.]
            \item 定义所有参会组合的有序集合$\Omega$, $\Omega$共有303个元素, 代表着附录中的参会组合,
                    每个元素均为代表着所前往会议城市的编号.
                    \[\Omega=(\{1\}, \{1, 4, 6\}, \{1, 4, 6, 7\}, \dotsc, \{13\}).\]
            \item 定义$S_{i}$为参会组合$i$中所有会议的影响力总和
            \item 对于9种老师组合, 定义$\boldsymbol P$为9种组合被选为大会报告的概率向量,
                    \[{\boldsymbol P}=(0.75, 0.75, 0.5, 0.35, 0.35, 0.35, 0.10, 0.10, 0.10).\]
            \item 令$i \leftarrow 1$, $E \leftarrow 0.$
            \item 选出$\Omega$所有元素中影响力总和$S_{t}$最大的元素$t$作为第$i$组的参会安排,
                    并且$E \leftarrow E + S_{t}\times P_{i}.$
            \item 对任意一个$\Omega$中的元素$\Omega_{\zeta}$,
                    若$\exists \chi (\chi \in \Omega_{\zeta} \wedge \chi \in \Omega_{t})$,
                    则令$\Omega_{\zeta} \leftarrow \emptyset.$
            \item $i \leftarrow i+1.$
            \item 如果$i\le 9$返回步骤v.
            \item 增加某几位讲师的参会数量, 以满足参会最小人数的约束.
        \end{enumerate}
    \subsubsection{运行结果}
        求解得最大影响力期望为32.0, 具体参会安排如下:
        
        \begin{scriptsize}
            \begin{align*}
                \text{主任:}&\ \textit{同济}
                    \xrightarrow[7.19]{\text{高铁}}\ \CityI\
                    \xrightarrow[7.25]{\text{飞机}}\ \CityV\
                    \xrightarrow[7.28]{\text{飞机}}\ \CityVI\
                    \xrightarrow[8.01]{\text{高铁}}\ \CityVIII\
                    \xrightarrow[8.05]{\text{高铁}} \textit{同济}
                    \xrightarrow[8.05]{\text{飞机}}\ \CityX\
                    \xrightarrow[8.09]{\text{高铁}} \textit{同济}\\
                \text{副主任:}&\ \textit{同济}
                    \xrightarrow[7.19]{\text{高铁}}\ \CityI\
                    \xrightarrow[7.25]{\text{飞机}}\ \CityV\
                    \xrightarrow[7.28]{\text{飞机}}\ \CityVI\
                    \xrightarrow[8.01]{\text{高铁}}\ \CityVIII\
                    \xrightarrow[8.05]{\text{高铁}} \textit{同济}
                    \xrightarrow[8.05]{\text{飞机}}\ \CityX\
                    \xrightarrow[8.09]{\text{高铁}} \textit{同济}\\
                \text{教授A:}&\ \textit{同济}
                    \xrightarrow{\hspace{13pt}}\ \CityII\
                    \xrightarrow[7.31]{\text{高铁}}\ \CityVII\
                    \xrightarrow[8.04]{\text{高铁}} \textit{同济}
                    \xrightarrow[8.07]{\text{飞机}}\ \CityXIII\
                    \xrightarrow[8.11]{\text{高铁}} \textit{同济}\\
                \text{教授B:}&\ \textit{同济}
                    \xrightarrow{\hspace{13pt}}\ \CityII\
                    \xrightarrow[7.31]{\text{高铁}}\ \CityVII\
                    \xrightarrow[8.04]{\text{高铁}} \textit{同济}
                    \xrightarrow[8.07]{\text{飞机}}\ \CityXIII\
                    \xrightarrow[8.11]{\text{高铁}} \textit{同济}\\
                \text{教授C:}&\ \textit{同济}
                    \xrightarrow[7.21]{\text{高铁}}\ \CityIII\
                    \xrightarrow[8.02]{\text{高铁}}\ \CityIX\
                    \xrightarrow[8.06]{\text{飞机}}\ \CityXII\
                    \xrightarrow[8.10]{\text{高铁}} \textit{同济}\\
                \text{副教授A:}&\ \textit{同济}
                    \xrightarrow[7.21]{\text{高铁}}\ \CityIII\
                    \xrightarrow[8.02]{\text{高铁}}\ \CityIX\
                    \xrightarrow[8.06]{\text{飞机}}\ \CityXII\
                    \xrightarrow[8.10]{\text{高铁}} \textit{同济}\\
                \text{副教授B:}&\ \textit{同济}
                    \xrightarrow[7.25]{\text{高铁}}\ \CityIV\
                    \xrightarrow[7.28]{\text{高铁}}\ \CityXI\
                    \xrightarrow[7.29]{\text{高铁}} \textit{同济}\\
                \text{副教授C:}&\ \textit{同济}
                    \xrightarrow[7.25]{\text{高铁}}\ \CityIV\
                    \xrightarrow[7.28]{\text{高铁}}\ \CityXI\
                    \xrightarrow[7.29]{\text{高铁}} \textit{同济}\\
                \text{副教授D:}&\ \textit{同济}
                    \xrightarrow{\text{休息}}\ \textit{同济}\\
                \text{副教授E:}&\ \textit{同济}
                    \xrightarrow{\text{休息}}\ \textit{同济}\\
                \text{副教授F:}&\ \textit{同济}
                    \xrightarrow{\text{休息}}\ \textit{同济}\\
                \text{副教授G:}&\ \textit{同济}
                    \xrightarrow{\text{休息}}\ \textit{同济}\\
                \text{副教授H:}&\ \textit{同济}
                    \xrightarrow{\text{休息}}\ \textit{同济}\\
                \text{讲师A:}&\ \textit{同济}
                    \xrightarrow[7.19]{\text{高铁}}\ \CityI\
                    \xrightarrow[7.29]{\text{高铁}} \textit{同济}\\
                \text{讲师B:}&\ \textit{同济}
                    \xrightarrow{\hspace{13pt}}\ \CityII\
                    \xrightarrow{\hspace{13pt}}\textit{同济}\\
                \text{讲师C:}&\ \textit{同济}
                    \xrightarrow{\text{休息}} \textit{同济}\\
                \text{讲师D:}&\ \textit{同济}
                    \xrightarrow{\text{休息}}\ \textit{同济}\\
                \text{讲师E:}&\ \textit{同济}
                    \xrightarrow{\text{休息}}\ \textit{同济}
            \end{align*}
        \end{scriptsize}
    \subsubsection{算法评价}
        \subsubsubsection{合理性}
            以上算法, 优先考虑被选为会议报告概率最大的组别, 使其参会组合星级和最大,
            能极大概率的保证规划出的星级期望尽可能大, 可以做到接近或者等于最优值.
        \subsubsubsection{不合理性}
            从运行结果便可看出, 教授们参加了非常多的会议, 日程繁忙, 而有接近一半的教授无所事事,
            显然与真实情况有比较大的差距.


    \section{基于价格和影响力期望值综合考虑的日程安排}

    我们基于参会安排的费用和影响力以及被报告选为大会报告的概率, 做出日程安排.

    \subsection{安排的科学性}

    此日程安排在经费最优的基础上,进行调整,使得参会安排符合以下条件:
    \begin{enumerate}
        \item 每位教师都参加会议.
        \item 尽量使得教授参加级别较高的会议.
        \item 尽量使得教授和教授参加相同的会议.
        \item 尽量使得副教授和副教授参加相同的会议.
    \end{enumerate}
    其中, 条件1, 2是为了响应学院积极报名的要求, 同时尽可能提高同济大学影响力.
    条件2、3是为了尽量提高报告选为大会报告的影响力期望.
    因此,本安排具有经费少, 报告被选为大会报告的星级期望更大, 同时也能较高地提高同济大学影响力的优点.

    \subsection{具体安排}

    \begin{scriptsize}
        \begin{align*}
            \text{主任:}&\ \textit{同济}
                \xrightarrow[7.19]{\text{高铁}}\ \CityI\
                \xrightarrow[7.25]{\text{高铁}}\ \CityIV\
                \xrightarrow[8.09]{\text{高铁}} \textit{同济}\\
            \text{副主任:}&\ \textit{同济}
                \xrightarrow[7.19]{\text{高铁}}\ \CityI\
                \xrightarrow[7.25]{\text{高铁}}\ \CityIV\
                \xrightarrow[8.09]{\text{高铁}} \textit{同济}\\
            \text{教授A:}&\ \textit{同济}
                \xrightarrow{\hspace{13pt}}\ \CityII\
                \xrightarrow[7.31]{\text{高铁}}\ \CityVII\
                \xrightarrow[8.04]{\text{高铁}} \textit{同济}\\
            \text{教授B:}&\ \textit{同济}
                \xrightarrow[7.25]{\text{高铁}}\ \CityV\
                \xrightarrow[7.28]{\text{高铁}}\ \CityVI\
                \xrightarrow[8.01]{\text{高铁}} \textit{同济}\\
            \text{教授C:}&\ \textit{同济}
                \xrightarrow[7.21]{\text{高铁}}\ \CityIII\
                \xrightarrow[8.02]{\text{高铁}}\ \CityIX\
                \xrightarrow[8.06]{\text{飞机}}\ \CityXII\
                \xrightarrow[8.10]{\text{高铁}} \textit{同济}\\
            \text{副教授A:}&\ \textit{同济}
                \xrightarrow[7.21]{\text{高铁}}\ \CityIII\
                \xrightarrow[8.02]{\text{高铁}}\ \CityIX\
                \xrightarrow[8.06]{\text{飞机}}\ \CityXII\
                \xrightarrow[8.10]{\text{高铁}} \textit{同济}\\
            \text{副教授B:}&\ \textit{同济}
                \xrightarrow[7.25]{\text{高铁}}\ \CityIV\
                \xrightarrow[7.28]{\text{高铁}}\ \CityXI\
                \xrightarrow[7.29]{\text{高铁}} \textit{同济}\\
            \text{副教授C:}&\ \textit{同济}
                \xrightarrow[7.25]{\text{高铁}}\ \CityV\
                \xrightarrow[7.28]{\text{高铁}}\ \CityVI\
                \xrightarrow[8.01]{\text{高铁}} \textit{同济}\\
            \text{副教授D:}&\ \textit{同济}
                \xrightarrow[7.31]{\text{高铁}}\ \CityVII\
                \xrightarrow[8.05]{\text{高铁}}\ \CityX\
                \xrightarrow[8.09]{\text{高铁}} \textit{同济}\\
            \text{副教授E:}&\ \textit{同济}
                \xrightarrow[7.25]{\text{高铁}}\ \CityV\
                \xrightarrow[7.28]{\text{高铁}}\ \CityVI\
                \xrightarrow[8.01]{\text{高铁}} \textit{同济}\\
            \text{副教授F:}&\ \textit{同济}
                \xrightarrow{\hspace{13pt}}\ \CityII\
                \xrightarrow[8.02]{\text{高铁}}\ \CityIX\
                \xrightarrow[8.07]{\text{高铁}} \textit{同济}\\
            \text{副教授G:}&\ \textit{同济}
                \xrightarrow{\hspace{13pt}}\ \CityII\
                \xrightarrow[8.05]{\text{高铁}}\ \CityXII\
                \xrightarrow[8.09]{\text{高铁}} \textit{同济}\\
            \text{副教授H:}&\ \textit{同济}
                \xrightarrow[8.06]{\text{高铁}}\ \CityIII\
                \xrightarrow[8.06]{\text{高铁}}\ \CityX\
                \xrightarrow[8.10]{\text{高铁}} \textit{同济}\\
            \text{讲师A:}&\ \textit{同济}
                \xrightarrow[8.06]{\text{高铁}}\ \CityV\
                \xrightarrow[8.02]{\text{高铁}}\ \CityVI\
                \xrightarrow[8.07]{\text{高铁}} \textit{同济}\\
            \text{讲师B:}&\ \textit{同济}
                \xrightarrow[8.06]{\text{高铁}}\ \CityVII\
                \xrightarrow[8.05]{\text{高铁}}\ \CityX\
                \xrightarrow[8.09]{\text{高铁}} \textit{同济}\\
            \text{讲师C:}&\ \textit{同济}
                \xrightarrow[8.06]{\text{高铁}}\ \CityVIII\
                \xrightarrow[8.06]{\text{高铁}}\ \CityXI\
                \xrightarrow[8.10]{\text{高铁}} \textit{同济}\\
            \text{讲师D:}&\ \textit{同济}
                \xrightarrow[8.06]{\text{高铁}}\ \CityIX\
                \xrightarrow[8.06]{\text{高铁}}\ \CityXII\
                \xrightarrow[8.11]{\text{高铁}} \textit{同济}\\
            \text{讲师E:}&\ \textit{同济}
                \xrightarrow[8.06]{\text{高铁}}\ \CityIX\
                \xrightarrow[8.06]{\text{飞机}}\ \CityXIII\
                \xrightarrow[8.11]{\text{高铁}} \textit{同济}\\
        \end{align*}
    \end{scriptsize}

    \section{结论}

    我们通过对不同约束条件进行分析,
    分别建立了动态规划模型、动态规划模型的变种分组背包模型与贪心算法模型三种不同的模型,
    较为准确, 合理地给出了满足不同要求的最优化人员参会安排,
    并列出了具体的实施方案,
    较好地解决了本题中一系列关于参会安排的问题.
    并且在进行了误差分析与敏感度分析后,
    发现模型对其中参数的变化并不敏感, 故具有一定的稳定性.

    % -------------------- Bibliography --------------------

    \newpage
    \bibliography{tju_competition}
    \bibliographystyle{plain}

    % -------------------- Appendix --------------------

    \newpage
    \appendix

    \section{参会组合表}

    \begin{footnotesize}
    \begin{longtable}{cccccc}
        \multicolumn{6}{c}%
        {\textbf{参会组合表}}\\[5pt]
        \hlineB{2}
        \multicolumn{1}{c}{\sffamily 序号}
        &\multicolumn{1}{c}{\sffamily 会议城市1}
        &\multicolumn{1}{c}{\sffamily 会议城市2}
        &\multicolumn{1}{c}{\sffamily 会议城市3}
        &\multicolumn{1}{c}{\sffamily 会议城市4}
        &\multicolumn{1}{c}{\sffamily 会议城市5}\\
        \hlineB{1}
        \endfirsthead
        \multicolumn{6}{c}%
        {\textbf{参会组合表}
        (\textit{接上页})}\\[5pt]
        \hlineB{2}
        \multicolumn{1}{c}{\sffamily 序号}
        &\multicolumn{1}{c}{\sffamily 会议城市1}
        &\multicolumn{1}{c}{\sffamily 会议城市2}
        &\multicolumn{1}{c}{\sffamily 会议城市3}
        &\multicolumn{1}{c}{\sffamily 会议城市4}
        &\multicolumn{1}{c}{\sffamily 会议城市5}\\
        \hlineB{1}
        \endhead
        \hlineB{2}
        \multicolumn{6}{r}{\small\itshape 后接下页}\\
        \endfoot
        \hlineB{2}
        \endlastfoot
        1	& 北京 \\
        2	& 北京  & 兰州 \\
        3	& 北京  & 兰州  & 昆明 \\
        4	& 北京  & 兰州  & 昆明  & 南京 \\
        5	& 北京  & 兰州  & 昆明  & 南京  & 济南 \\
        6	& 北京  & 兰州  & 昆明  & 南京  & 天津 \\
        7	& 北京  & 兰州  & 昆明  & 南京  & 咸阳 \\
        8	& 北京  & 兰州  & 昆明  & 南京  & 大连 \\
        9	& 北京  & 兰州  & 昆明  & 厦门 \\
        10	& 北京  & 兰州  & 昆明  & 厦门  & 济南 \\
        11	& 北京  & 兰州  & 昆明  & 厦门  & 天津 \\
        12	& 北京  & 兰州  & 昆明  & 厦门  & 咸阳 \\
        13	& 北京  & 兰州  & 昆明  & 厦门  & 大连 \\
        14	& 北京  & 兰州  & 昆明  & 杭州 \\
        15	& 北京  & 兰州  & 昆明  & 杭州  & 天津 \\
        16	& 北京  & 兰州  & 昆明  & 杭州  & 咸阳 \\
        17	& 北京  & 兰州  & 昆明  & 杭州  & 大连 \\
        18	& 北京  & 兰州  & 昆明  & 济南 \\
        19	& 北京  & 兰州  & 昆明  & 天津 \\
        20	& 北京  & 兰州  & 昆明  & 咸阳 \\
        21	& 北京  & 兰州  & 昆明  & 大连 \\
        22	& 北京  & 兰州  & 南京 \\
        23	& 北京  & 兰州  & 南京  & 济南 \\
        24	& 北京  & 兰州  & 南京  & 天津 \\
        25	& 北京  & 兰州  & 南京  & 咸阳 \\
        26	& 北京  & 兰州  & 南京  & 大连 \\
        27	& 北京  & 兰州  & 厦门 \\
        28	& 北京  & 兰州  & 厦门  & 济南 \\
        29	& 北京  & 兰州  & 厦门  & 天津 \\
        30	& 北京  & 兰州  & 厦门  & 咸阳 \\
        31	& 北京  & 兰州  & 厦门  & 大连 \\
        32	& 北京  & 兰州  & 杭州 \\
        33	& 北京  & 兰州  & 杭州  & 天津 \\
        34	& 北京  & 兰州  & 杭州  & 咸阳 \\
        35	& 北京  & 兰州  & 杭州  & 大连 \\
        36	& 北京  & 兰州  & 济南 \\
        37	& 北京  & 兰州  & 天津 \\
        38	& 北京  & 兰州  & 咸阳 \\
        39	& 北京  & 兰州  & 大连 \\
        40	& 北京  & 成都 \\
        41	& 北京  & 成都  & 昆明 \\
        42	& 北京  & 成都  & 昆明  & 南京 \\
        43	& 北京  & 成都  & 昆明  & 南京  & 济南 \\
        44	& 北京  & 成都  & 昆明  & 南京  & 天津 \\
        45	& 北京  & 成都  & 昆明  & 南京  & 咸阳 \\
        46	& 北京  & 成都  & 昆明  & 南京  & 大连 \\
        47	& 北京  & 成都  & 昆明  & 厦门 \\
        48	& 北京  & 成都  & 昆明  & 厦门  & 济南 \\
        49	& 北京  & 成都  & 昆明  & 厦门  & 天津 \\
        50	& 北京  & 成都  & 昆明  & 厦门  & 咸阳 \\
        51	& 北京  & 成都  & 昆明  & 厦门  & 大连 \\
        52	& 北京  & 成都  & 昆明  & 杭州 \\
        53	& 北京  & 成都  & 昆明  & 杭州  & 天津 \\
        54	& 北京  & 成都  & 昆明  & 杭州  & 咸阳 \\
        55	& 北京  & 成都  & 昆明  & 杭州  & 大连 \\
        56	& 北京  & 成都  & 昆明  & 济南 \\
        57	& 北京  & 成都  & 昆明  & 天津 \\
        58	& 北京  & 成都  & 昆明  & 咸阳 \\
        59	& 北京  & 成都  & 昆明  & 大连 \\
        60	& 北京  & 成都  & 南京 \\
        61	& 北京  & 成都  & 南京  & 济南 \\
        62	& 北京  & 成都  & 南京  & 天津 \\
        63	& 北京  & 成都  & 南京  & 咸阳 \\
        64	& 北京  & 成都  & 南京  & 大连 \\
        65	& 北京  & 成都  & 厦门 \\
        66	& 北京  & 成都  & 厦门  & 济南 \\
        67	& 北京  & 成都  & 厦门  & 天津 \\
        68	& 北京  & 成都  & 厦门  & 咸阳 \\
        69	& 北京  & 成都  & 厦门  & 大连 \\
        70	& 北京  & 成都  & 杭州 \\
        71	& 北京  & 成都  & 杭州  & 天津 \\
        72	& 北京  & 成都  & 杭州  & 咸阳 \\
        73	& 北京  & 成都  & 杭州  & 大连 \\
        74	& 北京  & 成都  & 济南 \\
        75	& 北京  & 成都  & 天津 \\
        76	& 北京  & 成都  & 咸阳 \\
        77	& 北京  & 成都  & 大连 \\
        78	& 北京  & 昆明 \\
        79	& 北京  & 昆明  & 南京 \\
        80	& 北京  & 昆明  & 南京  & 济南 \\
        81	& 北京  & 昆明  & 南京  & 天津 \\
        82	& 北京  & 昆明  & 南京  & 咸阳 \\
        83	& 北京  & 昆明  & 南京  & 大连 \\
        84	& 北京  & 昆明  & 厦门 \\
        85	& 北京  & 昆明  & 厦门  & 济南 \\
        86	& 北京  & 昆明  & 厦门  & 天津 \\
        87	& 北京  & 昆明  & 厦门  & 咸阳 \\
        88	& 北京  & 昆明  & 厦门  & 大连 \\
        89	& 北京  & 昆明  & 杭州 \\
        90	& 北京  & 昆明  & 杭州  & 天津 \\
        91	& 北京  & 昆明  & 杭州  & 咸阳 \\
        92	& 北京  & 昆明  & 杭州  & 大连 \\
        93	& 北京  & 昆明  & 济南 \\
        94	& 北京  & 昆明  & 天津 \\
        95	& 北京  & 昆明  & 咸阳 \\
        96	& 北京  & 昆明  & 大连 \\
        97	& 北京  & 南京 \\
        98	& 北京  & 南京  & 济南 \\
        99	& 北京  & 南京  & 天津 \\
        100	& 北京  & 南京  & 咸阳 \\
        101	& 北京  & 南京  & 大连 \\
        102	& 北京  & 厦门 \\
        103	& 北京  & 厦门  & 济南 \\
        104	& 北京  & 厦门  & 天津 \\
        105	& 北京  & 厦门  & 咸阳 \\
        106	& 北京  & 厦门  & 大连 \\
        107	& 北京  & 杭州 \\
        108	& 北京  & 杭州  & 天津 \\
        109	& 北京  & 杭州  & 咸阳 \\
        110	& 北京  & 杭州  & 大连 \\
        111	& 北京  & 济南 \\
        112	& 北京  & 天津 \\
        113	& 北京  & 咸阳 \\
        114	& 北京  & 大连 \\
        115	& 上海 \\
        116	& 上海  & 昆明 \\
        117	& 上海  & 昆明  & 南京 \\
        118	& 上海  & 昆明  & 南京  & 济南 \\
        119	& 上海  & 昆明  & 南京  & 天津 \\
        120	& 上海  & 昆明  & 南京  & 咸阳 \\
        121	& 上海  & 昆明  & 南京  & 大连 \\
        122	& 上海  & 昆明  & 厦门 \\
        123	& 上海  & 昆明  & 厦门  & 济南 \\
        124	& 上海  & 昆明  & 厦门  & 天津 \\
        125	& 上海  & 昆明  & 厦门  & 咸阳 \\
        126	& 上海  & 昆明  & 厦门  & 大连 \\
        127	& 上海  & 昆明  & 杭州 \\
        128	& 上海  & 昆明  & 杭州  & 天津 \\
        129	& 上海  & 昆明  & 杭州  & 咸阳 \\
        130	& 上海  & 昆明  & 杭州  & 大连 \\
        131	& 上海  & 昆明  & 济南 \\
        132	& 上海  & 昆明  & 天津 \\
        133	& 上海  & 昆明  & 咸阳 \\
        134	& 上海  & 昆明  & 大连 \\
        135	& 上海  & 南京 \\
        136	& 上海  & 南京  & 济南 \\
        137	& 上海  & 南京  & 天津 \\
        138	& 上海  & 南京  & 咸阳 \\
        139	& 上海  & 南京  & 大连 \\
        140	& 上海  & 厦门 \\
        141	& 上海  & 厦门  & 济南 \\
        142	& 上海  & 厦门  & 天津 \\
        143	& 上海  & 厦门  & 咸阳 \\
        144	& 上海  & 厦门  & 大连 \\
        145	& 上海  & 杭州 \\
        146	& 上海  & 杭州  & 天津 \\
        147	& 上海  & 杭州  & 咸阳 \\
        148	& 上海  & 杭州  & 大连 \\
        149	& 上海  & 济南 \\
        150	& 上海  & 天津 \\
        151	& 上海  & 咸阳 \\
        152	& 上海  & 大连 \\
        153	& 广州 \\
        154	& 广州  & 昆明 \\
        155	& 广州  & 昆明  & 南京 \\
        156	& 广州  & 昆明  & 南京  & 济南 \\
        157	& 广州  & 昆明  & 南京  & 天津 \\
        158	& 广州  & 昆明  & 南京  & 咸阳 \\
        159	& 广州  & 昆明  & 南京  & 大连 \\
        160	& 广州  & 昆明  & 厦门 \\
        161	& 广州  & 昆明  & 厦门  & 济南 \\
        162	& 广州  & 昆明  & 厦门  & 天津 \\
        163	& 广州  & 昆明  & 厦门  & 咸阳 \\
        164	& 广州  & 昆明  & 厦门  & 大连 \\
        165	& 广州  & 昆明  & 杭州 \\
        166	& 广州  & 昆明  & 杭州  & 天津 \\
        167	& 广州  & 昆明  & 杭州  & 咸阳 \\
        168	& 广州  & 昆明  & 杭州  & 大连 \\
        169	& 广州  & 昆明  & 济南 \\
        170	& 广州  & 昆明  & 天津 \\
        171	& 广州  & 昆明  & 咸阳 \\
        172	& 广州  & 昆明  & 大连 \\
        173	& 广州  & 南京 \\
        174	& 广州  & 南京  & 济南 \\
        175	& 广州  & 南京  & 天津 \\
        176	& 广州  & 南京  & 咸阳 \\
        177	& 广州  & 南京  & 大连 \\
        178	& 广州  & 厦门 \\
        179	& 广州  & 厦门  & 济南 \\
        180	& 广州  & 厦门  & 天津 \\
        181	& 广州  & 厦门  & 咸阳 \\
        182	& 广州  & 厦门  & 大连 \\
        183	& 广州  & 杭州 \\
        184	& 广州  & 杭州  & 天津 \\
        185	& 广州  & 杭州  & 咸阳 \\
        186	& 广州  & 杭州  & 大连 \\
        187	& 广州  & 济南 \\
        188	& 广州  & 天津 \\
        189	& 广州  & 咸阳 \\
        190	& 广州  & 大连 \\
        191	& 兰州 \\
        192	& 兰州  & 昆明 \\
        193	& 兰州  & 昆明  & 南京 \\
        194	& 兰州  & 昆明  & 南京  & 济南 \\
        195	& 兰州  & 昆明  & 南京  & 天津 \\
        196	& 兰州  & 昆明  & 南京  & 咸阳 \\
        197	& 兰州  & 昆明  & 南京  & 大连 \\
        198	& 兰州  & 昆明  & 厦门 \\
        199	& 兰州  & 昆明  & 厦门  & 济南 \\
        200	& 兰州  & 昆明  & 厦门  & 天津 \\
        201	& 兰州  & 昆明  & 厦门  & 咸阳 \\
        202	& 兰州  & 昆明  & 厦门  & 大连 \\
        203	& 兰州  & 昆明  & 杭州 \\
        204	& 兰州  & 昆明  & 杭州  & 天津 \\
        205	& 兰州  & 昆明  & 杭州  & 咸阳 \\
        206	& 兰州  & 昆明  & 杭州  & 大连 \\
        207	& 兰州  & 昆明  & 济南 \\
        208	& 兰州  & 昆明  & 天津 \\
        209	& 兰州  & 昆明  & 咸阳 \\
        210	& 兰州  & 昆明  & 大连 \\
        211	& 兰州  & 南京 \\
        212	& 兰州  & 南京  & 济南 \\
        213	& 兰州  & 南京  & 天津 \\
        214	& 兰州  & 南京  & 咸阳 \\
        215	& 兰州  & 南京  & 大连 \\
        216	& 兰州  & 厦门 \\
        217	& 兰州  & 厦门  & 济南 \\
        218	& 兰州  & 厦门  & 天津 \\
        219	& 兰州  & 厦门  & 咸阳 \\
        220	& 兰州  & 厦门  & 大连 \\
        221	& 兰州  & 杭州 \\
        222	& 兰州  & 杭州  & 天津 \\
        223	& 兰州  & 杭州  & 咸阳 \\
        224	& 兰州  & 杭州  & 大连 \\
        225	& 兰州  & 济南 \\
        226	& 兰州  & 天津 \\
        227	& 兰州  & 咸阳 \\
        228	& 兰州  & 大连 \\
        229	& 成都 \\
        230	& 成都  & 昆明 \\
        231	& 成都  & 昆明  & 南京 \\
        232	& 成都  & 昆明  & 南京  & 济南 \\
        233	& 成都  & 昆明  & 南京  & 天津 \\
        234	& 成都  & 昆明  & 南京  & 咸阳 \\
        235	& 成都  & 昆明  & 南京  & 大连 \\
        236	& 成都  & 昆明  & 厦门 \\
        237	& 成都  & 昆明  & 厦门  & 济南 \\
        238	& 成都  & 昆明  & 厦门  & 天津 \\
        239	& 成都  & 昆明  & 厦门  & 咸阳 \\
        240	& 成都  & 昆明  & 厦门  & 大连 \\
        241	& 成都  & 昆明  & 杭州 \\
        242	& 成都  & 昆明  & 杭州  & 天津 \\
        243	& 成都  & 昆明  & 杭州  & 咸阳 \\
        244	& 成都  & 昆明  & 杭州  & 大连 \\
        245	& 成都  & 昆明  & 济南 \\
        246	& 成都  & 昆明  & 天津 \\
        247	& 成都  & 昆明  & 咸阳 \\
        248	& 成都  & 昆明  & 大连 \\
        249	& 成都  & 南京 \\
        250	& 成都  & 南京  & 济南 \\
        251	& 成都  & 南京  & 天津 \\
        252	& 成都  & 南京  & 咸阳 \\
        253	& 成都  & 南京  & 大连 \\
        254	& 成都  & 厦门 \\
        255	& 成都  & 厦门  & 济南 \\
        256	& 成都  & 厦门  & 天津 \\
        257	& 成都  & 厦门  & 咸阳 \\
        258	& 成都  & 厦门  & 大连 \\
        259	& 成都  & 杭州 \\
        260	& 成都  & 杭州  & 天津 \\
        261	& 成都  & 杭州  & 咸阳 \\
        262	& 成都  & 杭州  & 大连 \\
        263	& 成都  & 济南 \\
        264	& 成都  & 天津 \\
        265	& 成都  & 咸阳 \\
        266	& 成都  & 大连 \\
        267	& 昆明 \\
        268	& 昆明  & 南京 \\
        269	& 昆明  & 南京  & 济南 \\
        270	& 昆明  & 南京  & 天津 \\
        271	& 昆明  & 南京  & 咸阳 \\
        272	& 昆明  & 南京  & 大连 \\
        273	& 昆明  & 厦门 \\
        274	& 昆明  & 厦门  & 济南 \\
        275	& 昆明  & 厦门  & 天津 \\
        276	& 昆明  & 厦门  & 咸阳 \\
        277	& 昆明  & 厦门  & 大连 \\
        278	& 昆明  & 杭州 \\
        279	& 昆明  & 杭州  & 天津 \\
        280	& 昆明  & 杭州  & 咸阳 \\
        281	& 昆明  & 杭州  & 大连 \\
        282	& 昆明  & 济南 \\
        283	& 昆明  & 天津 \\
        284	& 昆明  & 咸阳 \\
        285	& 昆明  & 大连 \\
        286	& 南京 \\
        287	& 南京  & 济南 \\
        288	& 南京  & 天津 \\
        289	& 南京  & 咸阳 \\
        290	& 南京  & 大连 \\
        291	& 厦门 \\
        292	& 厦门  & 济南 \\
        293	& 厦门  & 天津 \\
        294	& 厦门  & 咸阳 \\
        295	& 厦门  & 大连 \\
        296	& 杭州 \\
        297	& 杭州  & 天津 \\
        298	& 杭州  & 咸阳 \\
        299	& 杭州  & 大连 \\
        300	& 济南 \\
        301	& 天津 \\
        302	& 咸阳 \\
        303	& 大连 \\
    \end{longtable}
\end{footnotesize}

    \section{基于动态规划的费用最优化模型求解代码}

    \begin{lstlisting}[language=C++, numberstyle={\color{black!33}\tiny\sffamily}, basicstyle=\tiny]
#include<cstdio>
#include<cstring>
#include<iostream>
#include<ctime>
#define pathen (int)Pas[cur][a1][a2][a3][a4][a5][a6][a7][a8][a9][a10][a11][a12][a13]
using namespace std;
const int maxn = 15;
const int inf = 0x3fffffff;
int INF = 0x3fffffff;
int Paths[20][5];
int Dom[maxn] = { 0,650,0,550,400,460,480,490,500,500,450,480,490,320 };
int Money[maxn] = { 0,1200,980,650,500,500,500,550,550,600,500,500,400,650 };
int dist1[maxn][maxn] = {//火车表
	{ 0 },
	{ 0,0,0,0,0,0,0,0,0,0,0,0,0,0 },
	{ 0,1318,0,0,0,0,0,0,0,0,0,0,0,0 },
	{ 0,2298,1790,0,0,0,0,0,0,0,0,0,0,0 },
	{ 0,1784,2185,2687,0,0,0,0,0,0,0,0,0,0 },
	{ 0,1874,1976,inf,inf,0,0,0,0,0,0,0,0,0 },
	{ 0,2760,2252,1330,inf,1112,0,0,0,0,0,0,0,0 },
	{ 0,1023,295,1581,1782,1872,2349,0,0,0,0,0,0,0 },
	{ 0,2053,1085,inf,inf,inf,2337,1182,0,0,0,0,0,0 },
	{ 0,1279,159,1631,2038,inf,2093,256,993,0,0,0,0,0 },
	{ 0,406,912,2251,1737,1830,2713,608,1647,873,0,0,0,0 },
	{ 0,120,1196,2309,inf,1885,inf,901,1931,1157,284,0,0,0 },
	{ 0,1246,1539,inf,538,inf,inf,1244,inf,inf,1199,inf,0,0 },
	{ 0,963,2054,inf,inf,inf,inf,1759,inf,inf,1142,836,inf,0 }
};
int dist2[maxn][maxn] =
{
	{ 0 },
	{ 0,0,0,0,0,0,0,0,0,0,0,0,0,0 },
	{ 0,1070,0,0,0,0,0,0,0,0,0,0,0,0 },
	{ 0,1893,1216,0,0,0,0,0,0,0,0,0,0,0 },
	{ 0,1182,1716,1697,0,0,0,0,0,0,0,0,0,0 },
	{ 0,1528,1669,1232,620,0,0,0,0,0,0,0,0,0 },
	{ 0,2106,1955,1068,1245,645,0,0,0,0,0,0,0,0 },
	{ 0,898,270,1130,1451,1402,1746,0,0,0,0,0,0,0 },
	{ 0,1726,828,513,1878,1536,1545,845,0,0,0,0,0,0 },
	{ 0,1121,174,1043,1650,1540,1807,235,678,0,0,0,0,0 },
	{ 0,360,733,1550,1196,1380,1889,535,1367,762,0,0,0,0 },
	{ 0,107,963,1811,1222,1518,2080,793,1629,1014,271,0,0,0 },
	{ 0,921,1250,1321,481,598,1191,968,1417,1167,801,922,0,0 },
	{ 0,461,853,1928,1598,1842,2346,799,1613,967,472,380,1259,0 }
};
double Speed[2] = { 290,900 };
struct Node {
	int id;
	int Beg, End;
	Node(int id = 0, int Beg = 0, int End = 0) :id(id), Beg(Beg), End(End) {}
}Nodes[maxn];
struct Arc {
	int u, v, w;
};
inline int Min(int a, int b) {
	return (a < b) ? a : b;
}

int Cost(int u, int v, int &flag, int &Kind) {
	if (Nodes[v].Beg == Nodes[u].End + 1) {
		Kind = 1;
		return (int)(dist2[u][v] * 0.8);
	}
	else if (Nodes[v].Beg - Nodes[u].End >= 1 && dist1[u][v] <= 1000000) {
		Dom[v]>Dom[u] ? Kind = 2 : Kind = 3;
		return (int)(dist1[u][v] * 0.5 + Min(Dom[v], Dom[u])*(Nodes[v].Beg - Nodes[u].End - 1));
	}
	else if (Nodes[v].Beg - Nodes[u].End >= 1) {
		Dom[v]>Dom[u] ? Kind = 4 : Kind = 5;
		return (int)(dist2[u][v] * 0.8 + Min(Dom[v], Dom[u])*(Nodes[v].Beg - Nodes[u].End - 1));
	}
	else {
		flag = -1;
		return -1;
	}
}
int Cost1(int u, int &p) {
	dist1[2][u] * 0.5<dist2[2][u] * 0.8 ? p = 1 : p = 0;
	return (int)(Min(dist1[2][u] * 0.5, dist2[2][u] * 0.8));
}


int dp[20][4][4][3][3][3][3][3][3][3][3][3][3][3];

struct Pa {
	char c1, c2;
	char k;
	short w;
	struct Pa* last;
	Pa(char c1=0,char c2=0,char k=0,short w=0,struct Pa* last=NULL)
	:c1(c1),c2(c2),k(k),w(w),last(last){}
}Pas[20][4][4][3][3][3][3][3][3][3][3][3][3][3];
void dfs(int cur,struct Pa* p) {
	if (cur==0) {
		return;
	}
	Paths[cur][0] = (int)(p->c1);
	Paths[cur][1] = (int)(p->c2);
	Paths[cur][2] = (int)(p->w);
	Paths[cur][3] = (int)(p->k);
	dfs(cur-1,p->last);
	return;
}
char *Paing(int Kind){
	if(Kind==1){
		return (char *)"直接坐飞机到";
	}
	if(Kind==2){
		return (char *)"先住宿,再坐火车到";
	}
	if(Kind==3){
		return (char *)"先坐火车到,再住宿";
	}
	if(Kind==4){
		return (char *)"先住宿,再坐飞机到";
	}
	if(Kind==5){
		return (char *)"先坐飞机到,再住宿";
	}
	if(Kind==6){
        return (char *)"做火车回上海,再坐火车去";
	}
	if(Kind==7){
		return (char *)"做火车回上海,再坐飞机去";
	}
	if(Kind==8){
		return (char *)"坐飞机回上海,再做火车去";
	}
	if(Kind==9){
		return (char *)"坐飞机回上海,再做飞机去";
	}
}
int main() {
	freopen("a.txt", "w", stdout);
	time_t a, b;
	int Chs[15];
	int p1,p2,q1,q2,w,s,t,g,flag,Kind;
	a = time(NULL);
	for (int i = 1; i <= 13; i++) {
		for (int j = i + 1; j <= 13; j++) {
			dist1[i][j] = dist1[j][i];
		}
	}
	for (int i = 1; i <= 13; i++) {
		for (int j = i + 1; j <= 13; j++) {
			dist2[i][j] = dist2[j][i];
		}
	}
	Nodes[1] = Node(1, 0, 5);
	Nodes[2] = Node(2, 1, 6);
	Nodes[3] = Node(3, 2, 6);
	Nodes[4] = Node(4, 6, 8);
	Nodes[5] = Node(5, 6, 8);
	Nodes[6] = Node(6, 9, 11);
	Nodes[7] = Node(7, 12, 14);
	Nodes[8] = Node(8, 13, 15);
	Nodes[9] = Node(9, 14, 17);
	Nodes[10] = Node(10, 17, 19);
	Nodes[11] = Node(11, 18, 20);
	Nodes[12] = Node(12, 18, 21);
	Nodes[13] = Node(13, 19, 21);
	memset(dp, 63, sizeof(dp));
	INF = dp[0][0][0][0][0][0][0][0][0][0][0][0][0][0];
	dp[0][0][0][0][0][0][0][0][0][0][0][0][0][0] = 0;
	cout << "程序运行开始!" << endl;
	for (int i = 0; i <= 4; i++) {
		for (int a1 = 0; a1 <= 3; a1++) {
			for (int a2 = 0; a2 <= 3; a2++) {
				for (int a3 = 0; a3 <= 2; a3++) {
					for (int a4 = 0; a4 <= 2; a4++) {
						for (int a5 = 0; a5 <= 2; a5++) {
							for (int a6 = 0; a6 <= 2; a6++) {
								for (int a7 = 0; a7 <= 2; a7++) {
									for (int a8 = 0; a8 <= 2; a8++) {
										for (int a9 = 0; a9 <= 2; a9++) {
											for (int a10 = 0; a10 <= 2; a10++) {
												for (int a11 = 0; a11 <= 2; a11++) {
													for (int a12 = 0; a12 <= 2; a12++) {
														for (int a13 = 0; a13 <= 2; a13++) {
	if(dp[i][a1][a2][a3][a4][a5][a6][a7][a8][a9][a10][a11][a12][a13]>1000000){
       continue;
    }
	for (int k = 1; k <= 13; k++) {
		for (int l = k + 1; l <= 13; l++) {
			memset(Chs,0,sizeof(Chs));
			Chs[k] = Chs[l] = 1;
			flag = 1;
			p1 = 0;
			p2 = 0;
		    Kind = 0;
			w = Cost(k, l, flag, Kind);
			if (flag == -1) {
				continue;
			}
			
			s = dp[i + 1][Min(a1 + Chs[1], 3)][Min(a2 + Chs[2], 3)][Min(a3 + Chs[3], 2)][Min(a4 + Chs[4], 2)][Min(a5 + Chs[5], 2)][Min(a6 + Chs[6], 2)]
			[Min(a7 + Chs[7], 2)][Min(a8 + Chs[8], 2)][Min(a9 + Chs[9], 2)][Min(a10 + Chs[10], 2)][Min(a11 + Chs[11], 2)][Min(a12 + Chs[12], 2)][Min(a13 + Chs[13], 2)];
			q1 = Cost1(k, p1);
			q2 = Cost1(l, p2);
			if (q1 + q2<w) {
				if (p1&&p2) Kind = 6;
				if (p1 && !p2) Kind = 7;
				if (!p1&&p2) Kind = 8;
				if (!p1 && !p2) Kind = 9;
			}
			g=Min(w, q1 + q2) + q1 + q2 + Money[k] + Money[l] + Dom[k] * (Nodes[k].End - Nodes[k].Beg) + Dom[l] * (Nodes[l].End - Nodes[l].Beg);
			t = dp[i][a1][a2][a3][a4][a5][a6][a7][a8][a9][a10][a11][a12][a13] + g;
			if (s>=t) {

				Pas[i + 1][Min(a1 + Chs[1], 3)][Min(a2 + Chs[2], 3)][Min(a3 + Chs[3], 2)][Min(a4 + Chs[4], 2)][Min(a5 + Chs[5], 2)][Min(a6 + Chs[6], 2)]
				[Min(a7 + Chs[7], 2)][Min(a8 + Chs[8], 2)][Min(a9 + Chs[9], 2)][Min(a10 + Chs[10], 2)][Min(a11 + Chs[11], 2)][Min(a12 + Chs[12], 2)][Min(a13 + Chs[13], 2)]
					= Pa((char)k,(char)l,(char)Kind,(short)g,&Pas[i][a1][a2][a3][a4][a5][a6][a7][a8][a9][a10][a11][a12][a13]);
				dp[i+1][Min(a1 + Chs[1], 3)][Min(a2 + Chs[2], 3)][Min(a3 + Chs[3], 2)][Min(a4 + Chs[4], 2)][Min(a5 + Chs[5], 2)][Min(a6 + Chs[6], 2)]
				[Min(a7 + Chs[7], 2)][Min(a8 + Chs[8], 2)][Min(a9 + Chs[9], 2)][Min(a10 + Chs[10], 2)][Min(a11 + Chs[11], 2)][Min(a12 + Chs[12], 2)][Min(a13 + Chs[13], 2)]
					= t;
			}
		}
	}
														}
													}
												}
											}
										}
									}
								}
							}
						}
					}
				}
			}
		}
		cout << dp[i+1][3][3][2][2][2][2][2][2][2][2][2][2][2] << endl;
		cout << "程序已经运行完成" << i + 1 << "/18" << endl;
	}
	for (int a1 = 0; a1 <= 3; a1++) {
		for (int a2 = 0; a2 <= 3; a2++) {
			for (int a3 = 0; a3 <= 2; a3++) {
				for (int a4 = 0; a4 <= 2; a4++) {
					for (int a5 = 0; a5 <= 2; a5++) {
						for (int a6 = 0; a6 <= 2; a6++) {
							for (int a7 = 0; a7 <= 2; a7++) {
								for (int a8 = 0; a8 <= 2; a8++) {
									for (int a9 = 0; a9 <= 2; a9++) {
										for (int a10 = 0; a10 <= 2; a10++) {
											for (int a11 = 0; a11 <= 2; a11++) {
												for (int a12 = 0; a12 <= 2; a12++) {
													for (int a13 = 0; a13 <= 2; a13++) {
														if (a1<2 || a2<1 || a12<1 || a13<1) {
															dp[5][a1][a2][a3][a4][a5][a6][a7][a8][a9][a10][a11][a12][a13] = INF;
														}
													}
												}
											}
										}
									}
								}
							}
						}
					}
				}
			}
		}
	}
	for (int i = 5; i <= 12; i++) {
		for (int a1 = 0; a1 <= 3; a1++) {
			for (int a2 = 0; a2 <= 3; a2++) {
				for (int a3 = 0; a3 <= 2; a3++) {
					for (int a4 = 0; a4 <= 2; a4++) {
						for (int a5 = 0; a5 <= 2; a5++) {
							for (int a6 = 0; a6 <= 2; a6++) {
								for (int a7 = 0; a7 <= 2; a7++) {
									for (int a8 = 0; a8 <= 2; a8++) {
										for (int a9 = 0; a9 <= 2; a9++) {
											for (int a10 = 0; a10 <= 2; a10++) {
												for (int a11 = 0; a11 <= 2; a11++) {
													for (int a12 = 0; a12 <= 2; a12++) {
														for (int a13 = 0; a13 <= 2; a13++) {
	if(dp[i][a1][a2][a3][a4][a5][a6][a7][a8][a9][a10][a11][a12][a13]>1000000){
	 	continue;
    }

	for (int k = 1; k <= 13; k++) {
		for (int l = k + 1; l <= 13; l++) {
			memset(Chs,0,sizeof(Chs));
			Chs[k] = Chs[l] = 1;
			flag = 1;
			p1 = 0;
			p2 = 0;
			Kind = 0;
			w = Cost(k, l, flag, Kind);
			if (flag == -1) {
				continue;
			}
			s = dp[i + 1][Min(a1 + Chs[1], 3)][Min(a2 + Chs[2], 3)][Min(a3 + Chs[3], 2)][Min(a4 + Chs[4], 2)][Min(a5 + Chs[5], 2)][Min(a6 + Chs[6], 2)]
			[Min(a7 + Chs[7], 2)][Min(a8 + Chs[8], 2)][Min(a9 + Chs[9], 2)][Min(a10 + Chs[10], 2)][Min(a11 + Chs[11], 2)][Min(a12 + Chs[12], 2)][Min(a13 + Chs[13], 2)];
			q1 = Cost1(k, p1), q2 = Cost1(l, p2);
			if (q1 + q2<w) {
				if (p1&&p2) Kind = 6;
				if (p1 && !p2) Kind = 7;
				if (!p1&&p2) Kind = 8;
				if (!p1 && !p2) Kind = 9;
			}
			g=Min(w, q1 + q2) + q1 + q2 + Money[k] + Money[l] + Dom[k] * (Nodes[k].End - Nodes[k].Beg) + Dom[l] * (Nodes[l].End - Nodes[l].Beg);
			t = dp[i][a1][a2][a3][a4][a5][a6][a7][a8][a9][a10][a11][a12][a13] + g;
			if (s>=t) {

				Pas[i + 1][Min(a1 + Chs[1], 3)][Min(a2 + Chs[2], 3)][Min(a3 + Chs[3], 2)][Min(a4 + Chs[4], 2)][Min(a5 + Chs[5], 2)][Min(a6 + Chs[6], 2)]
				[Min(a7 + Chs[7], 2)][Min(a8 + Chs[8], 2)][Min(a9 + Chs[9], 2)][Min(a10 + Chs[10], 2)][Min(a11 + Chs[11], 2)][Min(a12 + Chs[12], 2)][Min(a13 + Chs[13], 2)]
					= Pa((char)k,(char)l,(char)Kind,(short)g,&Pas[i][a1][a2][a3][a4][a5][a6][a7][a8][a9][a10][a11][a12][a13]);
				dp[i + 1][Min(a1 + Chs[1], 3)][Min(a2 + Chs[2], 3)][Min(a3 + Chs[3], 2)][Min(a4 + Chs[4], 2)][Min(a5 + Chs[5], 2)][Min(a6 + Chs[6], 2)]
				[Min(a7 + Chs[7], 2)][Min(a8 + Chs[8], 2)][Min(a9 + Chs[9], 2)][Min(a10 + Chs[10], 2)][Min(a11 + Chs[11], 2)][Min(a12 + Chs[12], 2)][Min(a13 + Chs[13], 2)]
					= t;
			}
		}
	}
														}
													}
												}
											}
										}
									}
								}
							}
						}
					}
				}
			}
		}
		cout << dp[i+1][3][3][2][2][2][2][2][2][2][2][2][2][2] << endl;
		cout << "程序已经运行完成" << i + 1 << "/18" << endl;
	}
	for (int a1 = 0; a1 <= 3; a1++) {
		for (int a2 = 0; a2 <= 3; a2++) {
			for (int a3 = 0; a3 <= 2; a3++) {
				for (int a4 = 0; a4 <= 2; a4++) {
					for (int a5 = 0; a5 <= 2; a5++) {
						for (int a6 = 0; a6 <= 2; a6++) {
							for (int a7 = 0; a7 <= 2; a7++) {
								for (int a8 = 0; a8 <= 2; a8++) {
									for (int a9 = 0; a9 <= 2; a9++) {
										for (int a10 = 0; a10 <= 2; a10++) {
											for (int a11 = 0; a11 <= 2; a11++) {
												for (int a12 = 0; a12 <= 2; a12++) {
													for (int a13 = 0; a13 <= 2; a13++) {
														if (a3<1 || a4<1 || a5<1 || a6<1 || a7<1 || a8<1 || a9<1 || a10<1 || a11<1) {
															dp[13][a1][a2][a3][a4][a5][a6][a7][a8][a9][a10][a11][a12][a13] = INF;
														}
													}
												}
											}
										}
									}
								}
							}
						}
					}
				}
			}
		}
	}
	for (int i = 13; i <= 17; i++) {
		for (int a1 = 0; a1 <= 3; a1++) {
			for (int a2 = 0; a2 <= 3; a2++) {
				for (int a3 = 0; a3 <= 2; a3++) {
					for (int a4 = 0; a4 <= 2; a4++) {
						for (int a5 = 0; a5 <= 2; a5++) {
							for (int a6 = 0; a6 <= 2; a6++) {
								for (int a7 = 0; a7 <= 2; a7++) {
									for (int a8 = 0; a8 <= 2; a8++) {
										for (int a9 = 0; a9 <= 2; a9++) {
											for (int a10 = 0; a10 <= 2; a10++) {
												for (int a11 = 0; a11 <= 2; a11++) {
													for (int a12 = 0; a12 <= 2; a12++) {
														for (int a13 = 0; a13 <= 2; a13++) {
	if(dp[i][a1][a2][a3][a4][a5][a6][a7][a8][a9][a10][a11][a12][a13]>1000000){
       continue;
    }

	for (int k = 1; k <= 13; k++) {
		for (int l = k + 1; l <= 13; l++) {
			memset(Chs,0,sizeof(Chs));
			Chs[k] = Chs[l] = 1;
			flag = 1;
			p1 = 0;
			p2 = 0;
			Kind = 0;
			w = Cost(k, l, flag, Kind);
			if (flag == -1) {
				continue;
			}
			s = dp[i + 1][Min(a1 + Chs[1], 3)][Min(a2 + Chs[2], 3)][Min(a3 + Chs[3], 2)][Min(a4 + Chs[4], 2)][Min(a5 + Chs[5], 2)][Min(a6 + Chs[6], 2)]
			[Min(a7 + Chs[7], 2)][Min(a8 + Chs[8], 2)][Min(a9 + Chs[9], 2)][Min(a10 + Chs[10], 2)][Min(a11 + Chs[11], 2)][Min(a12 + Chs[12], 2)][Min(a13 + Chs[13], 2)];
			q1 = Cost1(k, p1), q2 = Cost1(l, p2);
			if (q1 + q2<w) {
				if (p1&&p2) Kind = 6;
				if (p1 && !p2) Kind = 7;
				if (!p1&&p2) Kind = 8;
				if (!p1 && !p2) Kind = 9;
			}
			g=Min(w, q1 + q2) + q1 + q2 + Money[k] + Money[l] + Dom[k] * (Nodes[k].End - Nodes[k].Beg) + Dom[l] * (Nodes[l].End - Nodes[l].Beg);
			t = dp[i][a1][a2][a3][a4][a5][a6][a7][a8][a9][a10][a11][a12][a13] + g;
			if (s>=t) {

				Pas[i + 1][Min(a1 + Chs[1], 3)][Min(a2 + Chs[2], 3)][Min(a3 + Chs[3], 2)][Min(a4 + Chs[4], 2)][Min(a5 + Chs[5], 2)][Min(a6 + Chs[6], 2)]
				[Min(a7 + Chs[7], 2)][Min(a8 + Chs[8], 2)][Min(a9 + Chs[9], 2)][Min(a10 + Chs[10], 2)][Min(a11 + Chs[11], 2)][Min(a12 + Chs[12], 2)][Min(a13 + Chs[13], 2)]
					= Pa((char)k,(char)l,(char)Kind,(short)g,&Pas[i][a1][a2][a3][a4][a5][a6][a7][a8][a9][a10][a11][a12][a13]);
				dp[i + 1][Min(a1 + Chs[1], 3)][Min(a2 + Chs[2], 3)][Min(a3 + Chs[3], 2)][Min(a4 + Chs[4], 2)][Min(a5 + Chs[5], 2)][Min(a6 + Chs[6], 2)]
				[Min(a7 + Chs[7], 2)][Min(a8 + Chs[8], 2)][Min(a9 + Chs[9], 2)][Min(a10 + Chs[10], 2)][Min(a11 + Chs[11], 2)][Min(a12 + Chs[12], 2)][Min(a13 + Chs[13], 2)]
					= t;
			}
		}
	}
														}
													}
												}
											}
										}
									}
								}
							}
						}
					}
				}
			}
		}
		cout << dp[i+1][3][3][2][2][2][2][2][2][2][2][2][2][2] << endl;
		cout << "程序已经运行完成" << i + 1 << "/18" << endl;
	}
	b = time(NULL);
	cout << "参加所有会议的最小费用是:" << endl;
	cout << dp[18][3][3][2][2][2][2][2][2][2][2][2][2][2] << endl;
	cout << "程序允许的总时间是:" << b - a << " s" << endl;
	cout << "路径是:" << endl;
	dfs(18,&Pas[18][3][3][2][2][2][2][2][2][2][2][2][2][2]);
	for (int i = 1; i <= 18; i++) {
		cout << "第" << i << "个人:"<<endl;
		cout << "前往城市" << Paths[i][0] << "和" << Paths[i][1] << endl;
		cout << "坐火车到" << Paths[i][0] <<", "<< Paing(Paths[i][3])  << Paths[i][1] << "  ,花费" << Paths[i][2] << "元" << endl;
		cout << endl;
	}
	
	return 0;
}

\end{lstlisting}


    \section{基于分组背包的动态规划影响力最优化模型求解代码}

    \begin{lstlisting}[language=C++, numberstyle={\color{black!33}\tiny\sffamily}, basicstyle=\tiny]
#include<cstdio>
#include<cstring>
#include<iostream>
#include<ctime>
using namespace std;
const int maxn = 20;
const int inf = 0x3fffffff;
const int H[20]={0,45,45,32,32,32,15,15,15,15,15,15,15,15,6,6,6,6,6};
int Dom[maxn] = { 0,650,0,550,400,460,480,490,500,500,450,480,490,320 };
int Money[maxn] = { 0,1200,980,650,500,500,500,550,550,600,500,500,400,650 };
int Rank[maxn] = { 0,5,5,4,3,4,3,3,4,3,4,3,3,4};
//int Rank[maxn] = { 0,1<<5,1<<5,1<<4,1<<3,1<<4,1<<3,1<<3,1<<4,1<<3,1<<4,1<<3,1<<3,1<<4};
int dp[20][50005];
int dist1[maxn][maxn] = {//火车表
	{ 0 },
	{ 0,0,0,0,0,0,0,0,0,0,0,0,0,0 },
	{ 0,1318,0,0,0,0,0,0,0,0,0,0,0,0 },
	{ 0,2298,1790,0,0,0,0,0,0,0,0,0,0,0 },
	{ 0,1784,2185,2687,0,0,0,0,0,0,0,0,0,0 },
	{ 0,1874,1976,inf,inf,0,0,0,0,0,0,0,0,0 },
	{ 0,2760,2252,1330,inf,1112,0,0,0,0,0,0,0,0 },
	{ 0,1023,295,1581,1782,1872,2349,0,0,0,0,0,0,0 },
	{ 0,2053,1085,inf,inf,inf,2337,1182,0,0,0,0,0,0 },
	{ 0,1279,159,1631,2038,inf,2093,256,993,0,0,0,0,0 },
	{ 0,406,912,2251,1737,1830,2713,608,1647,873,0,0,0,0 },
	{ 0,120,1196,2309,inf,1885,inf,901,1931,1157,284,0,0,0 },
	{ 0,1246,1539,inf,538,inf,inf,1244,inf,inf,1199,inf,0,0 },
	{ 0,963,2054,inf,inf,inf,inf,1759,inf,inf,1142,836,inf,0 }
};
int dist2[maxn][maxn] =
{
	{ 0 },
	{ 0,0,0,0,0,0,0,0,0,0,0,0,0,0 },
	{ 0,1070,0,0,0,0,0,0,0,0,0,0,0,0 },
	{ 0,1893,1216,0,0,0,0,0,0,0,0,0,0,0 },
	{ 0,1182,1716,1697,0,0,0,0,0,0,0,0,0,0 },
	{ 0,1528,1669,1232,620,0,0,0,0,0,0,0,0,0 },
	{ 0,2106,1955,1068,1245,645,0,0,0,0,0,0,0,0 },
	{ 0,898,270,1130,1451,1402,1746,0,0,0,0,0,0,0 },
	{ 0,1726,828,513,1878,1536,1545,845,0,0,0,0,0,0 },
	{ 0,1121,174,1043,1650,1540,1807,235,678,0,0,0,0,0 },
	{ 0,360,733,1550,1196,1380,1889,535,1367,762,0,0,0,0 },
	{ 0,107,963,1811,1222,1518,2080,793,1629,1014,271,0,0,0 },
	{ 0,921,1250,1321,481,598,1191,968,1417,1167,801,922,0,0 },
	{ 0,461,853,1928,1598,1842,2346,799,1613,967,472,380,1259,0 }
};

struct Node {
	int id;
	int Beg, End;
	Node(int id = 0, int Beg = 0, int End = 0) :id(id), Beg(Beg), End(End) {}
}Nodes[maxn];

inline int Min(int a, int b) {
	return (a < b) ? a : b;
}

int Cost(int u, int v) {
	if (Nodes[v].Beg == Nodes[u].End + 1) {
		return (int)(dist2[u][v] * 0.8);
	}
	else if (Nodes[v].Beg - Nodes[u].End >= 1 && dist1[u][v] <= 1000000) {
		return (int)(dist1[u][v] * 0.5 + Min(Dom[v], Dom[u])*(Nodes[v].Beg - Nodes[u].End - 1));
	}
	else if (Nodes[v].Beg - Nodes[u].End >= 1) {
		return (int)(dist2[u][v] * 0.8 + Min(Dom[v], Dom[u])*(Nodes[v].Beg - Nodes[u].End - 1));
	}
	else {
		return -1;
	}
}

int Cost1(int u) {
	return (int)(Min(dist1[2][u] * 0.5, dist2[2][u] * 0.8));
}
struct Wa{
	int k,v;
	int val,id;
	Wa(int k=0,int v=0,int val=0,int id=0):k(k),v(v),val(val),id(id){}
}Ways[20][50005];
int Obj[20],tmp=0;
int W[20][4005],V[20][4005],Pos[20];
int All_obj[250*15][20],cnts[250*15],Op=0;
int pp=0;
void Object(int p){
	Op++;
	cnts[Op]=p;
	memcpy(All_obj[Op],Obj,sizeof(Obj));
	int Sw=0,Sv=0;
	Sv+=Cost1(Obj[1]);
	Sv+=Cost1(Obj[p]);
	Sv+=Money[Obj[1]];
	Sv+=(Nodes[Obj[1]].End-Nodes[Obj[1]].Beg)*Dom[Obj[1]];
	Sw+=Rank[Obj[1]];
	for(int i=2;i<=p;i++){
			Sv+=Money[Obj[i]];
			Sv+=(Nodes[Obj[i]].End-Nodes[Obj[i]].Beg)*Dom[Obj[i]];
			Sv+=min(Cost1(Obj[i-1])+Cost1(Obj[i]),Cost(Obj[i-1],Obj[i]));
			Sw+=Rank[Obj[i]];
	}
	for(int i=1;i<=18;i++){
		    pp++;
			W[i][++Pos[i]]=Sw*H[i];
			V[i][Pos[i]]=Sv;
	}
	return;
}
void Dfs_pre(int cur){
	bool ok=false;
	for(int i=cur+1;i<=13;i++){
		if(cur==0||Cost(cur,i)!=-1){
			Obj[++tmp]=i;
		    Object(tmp);
			Dfs_pre(i);
			tmp--;
		}
	}
	return;
}
int sq=0,val_all=0;
void Way_dfs(int k,int v){
	if(k==0){
		return;
	}
	Way_dfs(Ways[k][v].k,Ways[k][v].v);
	cout<<"第"<<++sq<<"个人:"<<endl;
 	if(Ways[k][v].id==-1){
 		cout<<"呆在上海啥都不干"<<endl;
		cout<<endl;
	}
	else{
		cout<<"前去的会场是:";
		for(int i=1;i<=cnts[Ways[k][v].id];i++){
			cout<<All_obj[Ways[k][v].id][i]<<" ";
		}
		cout<<"花费的费用是:"<<Ways[k][v].val<<endl;
		val_all+=Ways[k][v].val;
		cout<<endl;
	}
	
}
int main(){
	freopen("c.txt","w",stdout);
	for (int i = 1; i <= 13; i++) {
		for (int j = i + 1; j <= 13; j++) {
			dist1[i][j] = dist1[j][i];
		}
	}
	for (int i = 1; i <= 13; i++) {
		for (int j = i + 1; j <= 13; j++) {
			dist2[i][j] = dist2[j][i];
		}
	}
	Nodes[1] = Node(1, 0, 5);
	Nodes[2] = Node(2, 1, 6);
	Nodes[3] = Node(3, 2, 6);
	Nodes[4] = Node(4, 6, 8);
	Nodes[5] = Node(5, 6, 8);
	Nodes[6] = Node(6, 9, 11);
	Nodes[7] = Node(7, 12, 14);
	Nodes[8] = Node(8, 13, 15);
	Nodes[9] = Node(9, 14, 17);
	Nodes[10] = Node(10, 17, 19);
	Nodes[11] = Node(11, 18, 20);
	Nodes[12] = Node(12, 18, 21);
	Nodes[13] = Node(13, 19, 21);
	Dfs_pre(0);
	/*
	cout<<Op<<endl;
	
	for(int i=1;i<=Op;i++){
			for(int j=1;j<=cnts[i];j++){
			cout<<All_obj[i][j]<<" ";
		}
		cout<<endl;
	}
	
	*/
	for (int k=1;k<=18;k++){
		for(int v=0;v<=50000;v++){
				dp[k][v]=dp[k-1][v];
				Ways[k][v]=Wa(k-1,v,0,-1);
			for(int i=1;i<=Op;i++){
				if(v>=V[k][i]){
					if(dp[k][v]<dp[k-1][v-V[k][i]]+W[k][i]){
						Ways[k][v]=Wa(k-1,v-V[k][i],V[k][i],i);
					}
					dp[k][v]=max(dp[k][v],dp[k-1][v-V[k][i]]+W[k][i]);
				}
			}
		}
	}
	cout<<"dynymic programming result:"<<dp[18][50000]<<endl;
	Way_dfs(18,50000);
	cout<<"总花费:"<<val_all<<endl;
	cout<<"总影响力:"<<dp[18][50000]<<endl;
	
	return 0;
}
\end{lstlisting}


    \section{基于贪心算法的影响力期望最优化模型求解代码}

    \begin{lstlisting}[language=C++, numberstyle={\color{black!33}\tiny\sffamily}, basicstyle=\tiny]
#include<cstdio>
#include<cstring>
#include<algorithm>
#include<iostream>
#include<ctime>
using namespace std;
const int maxn = 20;
const int inf = 0x3fffffff;
const int H[20]={0,45,45,32,32,32,15,15,15,15,15,15,15,15,6,6,6,6,6};
int Dom[maxn] = { 0,650,0,550,400,460,480,490,500,500,450,480,490,320 };
int Money[maxn] = { 0,1200,980,650,500,500,500,550,550,600,500,500,400,650 };
int Rank[maxn] = { 0,5,5,4,3,4,3,3,4,3,4,3,3,4};
//int Rank[maxn] = { 0,1<<5,1<<5,1<<4,1<<3,1<<4,1<<3,1<<3,1<<4,1<<3,1<<4,1<<3,1<<3,1<<4};
int dp[20][50005];
int dist1[maxn][maxn] = {//火车表
	{ 0 },
	{ 0,0,0,0,0,0,0,0,0,0,0,0,0,0 },
	{ 0,1318,0,0,0,0,0,0,0,0,0,0,0,0 },
	{ 0,2298,1790,0,0,0,0,0,0,0,0,0,0,0 },
	{ 0,1784,2185,2687,0,0,0,0,0,0,0,0,0,0 },
	{ 0,1874,1976,inf,inf,0,0,0,0,0,0,0,0,0 },
	{ 0,2760,2252,1330,inf,1112,0,0,0,0,0,0,0,0 },
	{ 0,1023,295,1581,1782,1872,2349,0,0,0,0,0,0,0 },
	{ 0,2053,1085,inf,inf,inf,2337,1182,0,0,0,0,0,0 },
	{ 0,1279,159,1631,2038,inf,2093,256,993,0,0,0,0,0 },
	{ 0,406,912,2251,1737,1830,2713,608,1647,873,0,0,0,0 },
	{ 0,120,1196,2309,inf,1885,inf,901,1931,1157,284,0,0,0 },
	{ 0,1246,1539,inf,538,inf,inf,1244,inf,inf,1199,inf,0,0 },
	{ 0,963,2054,inf,inf,inf,inf,1759,inf,inf,1142,836,inf,0 }
};
int dist2[maxn][maxn] =
{
	{ 0 },
	{ 0,0,0,0,0,0,0,0,0,0,0,0,0,0 },
	{ 0,1070,0,0,0,0,0,0,0,0,0,0,0,0 },
	{ 0,1893,1216,0,0,0,0,0,0,0,0,0,0,0 },
	{ 0,1182,1716,1697,0,0,0,0,0,0,0,0,0,0 },
	{ 0,1528,1669,1232,620,0,0,0,0,0,0,0,0,0 },
	{ 0,2106,1955,1068,1245,645,0,0,0,0,0,0,0,0 },
	{ 0,898,270,1130,1451,1402,1746,0,0,0,0,0,0,0 },
	{ 0,1726,828,513,1878,1536,1545,845,0,0,0,0,0,0 },
	{ 0,1121,174,1043,1650,1540,1807,235,678,0,0,0,0,0 },
	{ 0,360,733,1550,1196,1380,1889,535,1367,762,0,0,0,0 },
	{ 0,107,963,1811,1222,1518,2080,793,1629,1014,271,0,0,0 },
	{ 0,921,1250,1321,481,598,1191,968,1417,1167,801,922,0,0 },
	{ 0,461,853,1928,1598,1842,2346,799,1613,967,472,380,1259,0 }
};

struct Node {
	int id;
	int Beg, End;
	Node(int id = 0, int Beg = 0, int End = 0) :id(id), Beg(Beg), End(End) {}
}Nodes[maxn];

inline int Min(int a, int b) {
	return (a < b) ? a : b;
}

int Cost(int u, int v) {
	if (Nodes[v].Beg == Nodes[u].End + 1) {
		return (int)(dist2[u][v] * 0.8);
	}
	else if (Nodes[v].Beg - Nodes[u].End >= 1 && dist1[u][v] <= 1000000) {
		return (int)(dist1[u][v] * 0.5 + Min(Dom[v], Dom[u])*(Nodes[v].Beg - Nodes[u].End - 1));
	}
	else if (Nodes[v].Beg - Nodes[u].End >= 1) {
		return (int)(dist2[u][v] * 0.8 + Min(Dom[v], Dom[u])*(Nodes[v].Beg - Nodes[u].End - 1));
	}
	else {
		return -1;
	}
}

int Cost1(int u) {
	return (int)(Min(dist1[2][u] * 0.5, dist2[2][u] * 0.8));
}
struct Wa{
	int k,v;
	int val,id;
	Wa(int k=0,int v=0,int val=0,int id=0):k(k),v(v),val(val),id(id){}
}Ways[20][50005];
int Obj[20],tmp=0;
int W[20][4005],V[20][4005],Pos[20];
int All_obj[250*15][20],cnts[250*15],Op=0;
int pp=0;

struct C{
	int a[20];
	int sum;
	void clear(){
		memset(a,0,sizeof(a));
		sum=0;
		return;
	}
	bool operator < (const C &b) const{
		 return sum>b.sum;
	}
}Cs[500];
void Object(int p){
	Op++;
	Cs[Op].clear();
	for(int i=1;i<=p;i++){
		Cs[Op].a[Obj[i]]=1;
		Cs[Op].sum+=Rank[Obj[i]];
	}
	return;
}
void Dfs_pre(int cur){
	bool ok=false;
	for(int i=cur+1;i<=13;i++){
		if(cur==0||Cost(cur,i)!=-1){
			Obj[++tmp]=i;
		    Object(tmp);
			Dfs_pre(i);
			tmp--;
		}
	}
	return;
}

struct Teacher{
	double P;
}Ts[15];
int main(){
	//freopen("f.txt","w",stdout);
	for (int i = 1; i <= 13; i++) {
		for (int j = i + 1; j <= 13; j++) {
			dist1[i][j] = dist1[j][i];
		}
	}
	for (int i = 1; i <= 13; i++) {
		for (int j = i + 1; j <= 13; j++) {
			dist2[i][j] = dist2[j][i];
		}
	}
	Nodes[1] = Node(1, 0, 5);
	Nodes[2] = Node(2, 1, 6);
	Nodes[3] = Node(3, 2, 6);
	Nodes[4] = Node(4, 6, 8);
	Nodes[5] = Node(5, 6, 8);
	Nodes[6] = Node(6, 9, 11);
	Nodes[7] = Node(7, 12, 14);
	Nodes[8] = Node(8, 13, 15);
	Nodes[9] = Node(9, 14, 17);
	Nodes[10] = Node(10, 17, 19);
	Nodes[11] = Node(11, 18, 20);
	Nodes[12] = Node(12, 18, 21);
	Nodes[13] = Node(13, 19, 21);
	Dfs_pre(0);
	for(int i=1;i<=9;i++){
		if(i<=2){
			Ts[i].P=0.75;
		}
		else if(i<=6){
			Ts[i].P=0.5;
		}
		else{
			Ts[i].P=0.1;
		}
	}
	double ans_sum=0;
	int si=0;
	int ans[200][15],pas=0;
	while(true){
		if(si==9){
			break;
		}
		si++;
		sort(Cs+1,Cs+303+1);
		for(int i=1;i<=13;i++){
			ans[si][i]=Cs[1].a[i];
		}
		ans_sum+=Ts[si].P*Cs[1].sum*1.0;
		for(int i=2;i<=Op;i++){
			for(int j=1;j<=13;j++){
				if(Cs[1].a[j]&&Cs[i].a[j]){
					Cs[i].a[j]=0;
					Cs[i].sum-=Rank[j];
				}
			}
		}
		for(int i=1;i<=13;i++){
			Cs[1].a[i]=0;
			Cs[1].sum=0;
		}
	}
	cout<<Op<<endl;
	cout<<"结果:"<<endl;
	cout<<ans_sum<<endl;
	for(int i=1;i<=9;i++){
		cout<<"第"<<i<<"组:"<<endl ;
		for(int j=1;j<=13;j++){
			if(ans[i][j]){
				cout<<j<<" ";
			}
		}
		cout<<endl;
	}

	return 0;
}

\end{lstlisting}


\end{document}
