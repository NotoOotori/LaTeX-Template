数学家一生中的大成果多出于青年时代. 习总书记在五四运动100周年大会上的讲话中提到牛顿和莱布尼茨发现微积分时分别是22岁和28岁. 我们也知道年仅26岁的阿贝尔提出了群的理论证明了5次代数方程没有求根公式, 结论很震撼, 所用方法也统领了新的数学分支的发展, 现在交换群就以阿贝尔的名字命名, 称为阿贝尔群. 年仅20岁的伽罗瓦提出了伽罗瓦理论, 将代数方程的解与循环群联系起来, 被推广到数学的其他领域.

中国近现代的一些著名数学家也是如此. 陈省身出生于1911年, 他最大的贡献有两个, 第一个是在1945年(34岁)推广了Gauss-Bornet定理到高维情形, 现称为Chern-Gauss-Bornet定理, 该定理陈省身的名字写在赫赫有名的高斯之前也足以表明他作出的贡献之大. 第二个是在1946年(35岁)提出了陈氏类, 该概念广泛应用在物理学, Calabi-Yau流形等领域. 陈景润出生于1933年, 他最大的贡献是在1966年(33岁)证明了"$1 + 2$", 是用筛法做哥德巴赫猜想的最好结果, 现称为陈氏定理.

数学大家们能在青年时代就做出大成果离不开他们早年立下坚定志向, 并努力学习积淀, 最后他们能捕捉到不易遇见的微光, 想出别人想不到的思路, 从而做出举世瞩目的成果.

诚然, 能成为数学大家的人是极少数人, 但是数学学科发展也需要默默付出, 给大牛们补充细节计算例子, 完善数学大厦的普通人. 中国需要有青年人立下坚定志向从事数学事业, 并且能在本科阶段就多学习学校课程以外的内容, 多接触现代数学, 这样能更早地进入研究状态, 从而能对新时代中国的数学学科作出更多的贡献.
