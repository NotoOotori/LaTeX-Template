% !Mode::"TeX:UTF-8"

% -------------------- Information --------------------

\newcommand{\TITLE}{论新时代下新青年与数学学科发展}
\newcommand{\AUTHOR}{陈旭阳}
\newcommand{\SUBJECT}{毛泽东思想和中国特色社会主义理论体系概论课程论文}
\newcommand{\KEYWORDS}{新时代, 新青年, 数学学科建设}

% -------------------- Packages --------------------

\documentclass[a4paper, 12pt]{ctexart}
\usepackage{authblk} % 作者 (见校赛论文).
\usepackage{array}
\usepackage{boldline} % 长表格表格线加粗.
\usepackage{caption} % 题注.
\usepackage{enumerate}
\usepackage{fancyhdr} % 脚注.
\usepackage{filecontents}
\usepackage{float} % 你们这帮float给我乖乖听话 HHHHHHHHHHH.
\usepackage[T1]{fontenc} % Bera Mono Font
\usepackage{fontspec} % 字体.
\usepackage{graphicx}
\usepackage{hyperref}
\usepackage{lastpage}
\usepackage{makecell} % 表格线加粗 \Xhline{1.2pt}.
\usepackage[square, numbers, sort&compress]{natbib} % 引用.
\usepackage{subcaption} % subcaption and subfigure
\usepackage{titlesec} % Section标题格式.
\usepackage{varioref} % For Cross References.
\usepackage[dvipsnames]{xcolor} % 颜色声明.

% -------------------- Settings --------------------

% Title

\title{\TITLE}
\author{\AUTHOR}
\date{\today}

% Package: caption

\captionsetup{
    margin    =   6pt,
    font      =   small,
    labelfont =   bf
}

% Package: ctex

\setCJKfamilyfont{fzstk}{FZShuTi} % 方正舒体
\newcommand{\fzstk}{\CJKfamily{fzstk}}
\ctexset{
    section/name = {第,节},
    section/number = \chinese{section},
    subsection/number = (\chinese{subsection}),
    subsubsection/number = \arabic{subsubsection}
}

% Package: fancyhdr

\setlength{\headheight}{15pt}
\lhead{\SUBJECT}
\rhead{第\thepage 页\ 共\ \pageref{LastPage}\ 页}

% Package: graphicx

\graphicspath{{resources/}} % 图像文件目录

% Package: hyperref

\hypersetup{
    linktoc             =   all,
    colorlinks          =   true,
    linkcolor           =   black,
    anchorcolor         =   black,
    citecolor           =   black,
    filecolor           =   black,
    menucolor           =   black,
    runcolor            =   black,
    urlcolor            =   black,
	pdftitle           	=   {\TITLE},
	pdfauthor          	=   {\AUTHOR},
	pdfsubject         	=   {\SUBJECT},
	pdfcreator			=	{Visual Studio Code},
	pdfproducer			=	{XeLaTeX with documentclass ctexart},
	pdfkeywords        	=   {\KEYWORDS},
    bookmarksnumbered   =   true,
    pdfstartview        =   FitH,
    pdfpagelayout       =   OneColumn
}

% Package: varioref

\renewcommand{\reftextbefore}
    {on the \reftextvario{preceding page}{page before}}
\renewcommand{\reftextafter}
    {on the \reftextvario{following}{next} page}
\renewcommand{\reftextfacebefore}
    {on the \reftextvario{facing}{preceding} page}
\renewcommand{\reftextfaceafter}
    {on the \reftextvario{facing}{next} page}
\renewcommand{\reftextfaraway}[1]
    {on page \pageref{#1}}

%% Label formats

\labelformat{lstlisting}{代码#1}
\labelformat{equation}{式(#1)}
\labelformat{figure}{图#1}
\labelformat{table}{表#1}

% -------------------- General new commands --------------------



% -------------------- Specific new commands --------------------



% -------------------- Document --------------------

\begin{document}

    % -------------------- Title Page --------------------

    % \maketitle
    % \thispagestyle{empty}
    \pagenumbering{roman}

    % -------------------- Abstract Page --------------------

    % \newpage

    \begin{abstract}
        % !TeX root = ../main.tex

% 中英文摘要和关键字

\begin{abstract}{代数几何, 交换代数, 准素分解, 维数理论}
  这篇文章主要面向没有接触过但想要了解代数几何的本科生, 旨在以尽可能少的前置知识向读者自洽地展示代数几何的基础.

  代数几何的理论根基在于代数. 本文从基本定义开始建立了以环论为主模论为辅的交换代数理论, 研究了商环与分式环的基本性质, 证明了Noether环上准素分解的存在性及其满足的唯一性, 以域论为基础利用Noether正规化引理证明了域的有限生成整环上的维数定理和Hilbert零点定理.

  代数几何的研究对象在于几何. 本文研究了固定代数闭域上的仿射与射影空间中的代数集, 建立了根式理想与代数集之间的对应, 一次将准素分解理论与几何相联系. 本文还讨论了代数簇上的函数结构, 以此将维数理论应用到几何中, 并证明了两组代数范畴与几何范畴的等价. 最后本文简要介绍了概形的概念, 其相比于代数簇能更完整地体现代数所提供的信息. 读完本文, 读者可以初步掌握代数几何基础, 并做好进一步学习重要技术的准备.
\end{abstract}

\begin{abstract*}{algebraic geometry, commutative algebra, primary decomposition, dimension theory}
  \lipsum[1-2]
\end{abstract*}

    \end{abstract}

    \textbf{关键词}:
        \KEYWORDS

    % -------------------- Contents --------------------

    % \newpage
    \tableofcontents

    % -------------------- Body --------------------

    \newpage
    \pagestyle{fancy}
    \pagenumbering{arabic}
    
    \section*{引言\quad 新时代下新青年当奋斗}
    \addcontentsline{toc}{section}{引言\ 新时代下新青年当奋斗}

    写一本代数几何的入门书籍的困难之处在于如何在提供几何直观和例子的同时推导现代技术的语言. 对于代数几何来说, 作为学科起源的直观想法与现代研究中使用的技术方法有一道巨大的鸿沟.

首当其冲的问题就是语言. 代数几何这门学科经历了很多轮发展, 每一次都有独特的语言和看问题的观点. 十九世纪晚期代数几何有Riemann的函数论方法, 有Brill和Noether的更加几何的方法, 还有Kronecker, Dedekind和Weber的纯代数方法. 同时以Castelnuovo, Enriques和Severi为代表的意大利学派致力于代数曲面的分类. 随后二十世纪以Chow, Weil和Zariski为首的``美国"学派给意大利学派的直观提供了坚实的代数基础. 最近, Serre和Grothendieck创立了法国学派, 他们用概型和上同调的语言重写了代数几何, 并且用心的方法解决了非常多的就问题. 每一个学派都引入了很多新的概念和方法. 在写一本入门书的时候, 是用旧的语言写来贴近几何直观比较好, 还是直接从现代研究中所用的技术语言开始写比较好呢?

第二个问题是一个理念上的问题. 现代数学家倾向于抹去历史的总计: 每一个新的学派都用自己的语言重写这门学科的根基, 这样做有利于严谨性但是不利于教学. 如果一个人知道了概型的定义, 但是却没有意识到一个代数数域的整数环, 一条代数曲线和一个紧黎曼面都是一个``一维正则概型"的例子的话, 那又有什么用呢? 那么这样一本入门书籍的作者应该如何既讲明白代数几何来源于数论, 交换代数和复分析, 又给读者介绍这门学科的主要内容, 即仿射或射影空间上的代数簇, 同时推导概型和上同调这样的现代语言你呢? 有什么样的话题, 可以做到既传达代数几何的意义, 又能作为将来学习和研究的坚实基础呢?

\bigskip

我个人偏向于古典几何这一边. 我相信代数几何中最重要的问题就是那些从老派的仿射空间或射影空间的簇中引出的问题. 他们提供了激发所有后来发展的几何直观. 我以关于簇的一章开始本书, 以最简单的形式建立了一些例子和基本想法, 将它们从技术细节里解放出来. 只有在这些内容都介绍完之后, 我才能系统地建立概型, 凝聚层(coherent sheaves)以及上同调. 这些第二第三章的内容是这本书的技术核心. 在其中我试图陈述一些最重要的结论, 不过不追求一般性. 因此, 比如说上同调理论是针对N\"otherian概型上的拟凝聚层建立的, 因为这比较简单并且对于大多数应用来说已经足够强了; "顺像层(direct image sheaves)的凝聚性(coherence)"定理只证明了射影态射的情况, 并没有对一般的固有态射(proper morphism)进行证明. 因为相同的原因, 我没有引入可表函子(representable functors), 代数空间(algebraic spaces), 平展上同调(\'etale cohomology), sites以及拓扑斯(topoi)这些最抽象的概念.

第四第五章处理了古典的内容, 即非奇异的射影曲线和曲面, 但是运用了概型和上同调的技术. 我希望这些应用可以证明为了发展前两章中的技术所花费的努力是值得的.

关于代数几何的基本语言和逻辑根基, 我才用了交换代数. 它有个好处就是很精确. 并且, 通过在任意特征的域上进行研究, 我们可以获取一些基域是复数域这种古典情形下的洞见. 几年之前, 当Zariski试图编写代数几何的丛书时, 他还需要在书中自己推导需要用到的代数知识. 这项工作占据了全部工作的如此大一部分, 以致于他专门出版了一本只讲交换代数的书. 现在我们十分幸运已经有了很多出色的关于交换代数的书籍. 我关于代数的对策是在需要时引用纯代数的结论, 并给出证明的参考资料. 在书的最后列出了所有用到的代数结论.

原本我计划了完整的一系列附录 - 关于一些当今研究方向的简短介绍, 为了建立这本书的主要内容与研究的桥梁. 因为时间和篇幅有限只有三篇附录得以呈现在成书中. 我十分遗憾这本书中没能包含其余附录, 读者可以去阅读the Arcata volume, 其中有一些针对非专家的由专家所写的关于他们研究领域的文章. 此外, 关于代数几何的历史发展, 可以参考Dieudonn\'e的书. 因为没有足够多的篇幅去像我所希望的那样探索代数几何与相邻领域之间的关系, 可以参考Cassels的关于与数论关系的综述文章, 也可以参考Shafarevich的关于与复流形和拓扑的综述文章.

因为我相信主动学习是一种好的学习方法, 书中有分厂多的习题. 有一些习题包含了正文中没有介绍的重要结论. 其余习题包含了一些能阐释一般理论的具体例子. 我相信对于例子的研究与发展一般理论之间有着不可分割的关系. 认真的学生应该尝试尽可能多地做这些习题, 但是不应该觉得能立即解出他们. 有不少习题需要一些真正有创造性的努力才能够理解. 一个星号表示这道习题是困难的, 两个星号表示这道习题是一个未解决的问题.

(I, \S 8)中有关于代数几何和这本书的进一步介绍.

\subsection*{术语}

大部分情况, 书中的属于与广泛接受的用法是相同的, 不过还是有一些值得注意的例外. \emph{簇}一直是不可约的, 并且一直是在代数闭域上的. 在第一张中所有的簇都是拟仿射的. 在(II, \S 4)中簇的定义被拓展为包括\emph{抽象簇}, 即为代数闭域上的integral separated schemes of finite type, 词语\emph{曲线}, \emph{曲面}和\emph{3-fold}分别用来表示1维, 2维和3维的簇. 但是在第四章中, 词语\emph{曲线}只用来表示非奇异的射影曲线; 在第五章中\emph{曲线}表示任何非奇异射影曲面上的有效除子(effective divisor). 第五章中\emph{曲面}表示非奇异射影曲面.

书中的\emph{概型}在第一版的EGA中被称为预概型(prescheme), 不过在新版EGA中被称为概型.

书中\emph{射影态射}和\emph{very ample invertible sheaf}的定义与EGA中的定义并不等价. 他们在技术上比较简单, 但是有一个缺点就是它并不是基上的局部概念.

词语\emph{非奇异}只对簇使用, 对于一般的概型来说, 我们采用\emph{正则}和\emph{光滑}.

\subsection*{代数的结论}

我假设读者熟悉环, 理想, 模, N\"otherian环, 整相关(integral dependence)的基础知识, 并且乐意接受或者查询其它属于交换代数或者同调代数结论, 这些结论如果需要的话会在书中进行陈述, 伴有相关文本的引用. 这些结论会被标注一个A, 比如说定理3.9A, 为了和书中证明的结论进行区分.

基本的约定有这些: 所有的环都是交换幺环, 单位元记作1. 所有的环同态都将1映到1. 在整环或者域中, $0\neq 1$. 一个\emph{素理想}(或者极大理想)是环$A$的一个理想$\ideal{p}$, 满足商环$A/\ideal{p}$是一个整环(或者域). 因此环本身不被认为是一个素理想或者是极大理想.

环$A$中的一个\emph{乘性系统}(multiplicative system)是一个包含1的子集$S$, 满足关于乘法封闭. \emph{局部化}$S^{-1}A$定义为分式$a/s, a\in A, s\in S$在等价关系下的商, 其中$a/s$与$a'/s'$\emph{等价}仅当存在$s''\in S$使得$s''(s'a-sa')=0$成立. 有两种一直用的局部化列举如下. 设$\ideal{p}$是$A$中的素理想, 那么$S = A - \ideal{p}$是一个乘性系统, 相对应的局部化被记为$A_{\ideal{p}}$. 如果$f$是$A$的元素, 那么$S = \{1\}\cup \{f^n\vert n\geq 1\}$是一个乘性系统, 相对应的局部化被记为$A_f$. (注意在$f$是幂零元的情况下, $A_f$是零环.)

\subsection*{引用}

关于定理, 命题, 引理的交叉引用用圆括号以及数字, 例如(3.5). 对习题的引用例如(习题3.5). 对另一章节的结论的引用以章节数字打头, 例如(II, 3.5)或(II, 习题3.5)


    \section{新时代下数学学科建设的必要性}

    \subsection{国家发展需要数学}

中国的现代化发展离不开数学, 各个产业的现代化都以数学作为基础. 经济, 金融, 大数据等方向与数学的关系比较明显, 在这里我们以农业现代化举例.

农业现代化首先是监测手段的现代化. 以前农民只能通过经验进行进行灌溉, 而现在有天气预报能较为准确地预测未来的下雨情况, 有摄像头能拍摄并自动判断作物状态. 这一切都离不开数学的支持. 我们知道大气的运动是按照著名的偏微分方程Navier-Stokes方程运作的, 而天气预报实质上就是对该方程的数值模拟. 通过从数学理论上研究方程解的性质和各种数值计算方法的效率和稳定性都可以提升天气预报的即时性和准确率. 而智能识别作物状态属于计算机视觉方面, 现在常用深度学习的方法来实现, 而到现在为止深度学习算法仍需要数学理论证明其稳定性. 因此数学上的研究能指导智能识别作物状态的实践.

农业现代化其次是农业设备的现代化, 其中包括自动化与信息化. 自动化离不开数学上的系统控制理论, 正如2019年5月10日下午中国科学院数学与系统科学研究所郭雷院士在同济大学高等讲堂第78讲中所展现的那样, 控制论可以很好地应用在农业产业中. 可是实际操作中, 许多参数的选取都是仅凭借经验得到的, 没有经过理论上的推导. 所以从理论上的进一步研究可以增强自动化的效果和稳定性.信息化涉及到电气与通信方面, 比如现在比较热门的5G概念就是一种信息化的解决方案. 通信的底层是编码, 而编码实际上也需要深刻的数学理论. 任正非曾说华为的技术是以土耳其科学家艾利坎的极化码编制理论的基础上发展起来的. 艾利坎根据信息传输的极化现象, 构建了这套编码理论, 在理论上这样编码信号传输可以达到香农上限. 相反Turbo编码以及高通的LPDC编码由于其理论基础没有极化码编制理论那么完美, 更多的是实践中的产物, 从而限制了未来的应用空间. 可见一项技术背后的数学理论背景有多么重要.

\subsection{中国需要建设好数学学科}

数学是一门重视传承的学科, 历史上常发生一整个学派在某个方向均有结出成果的情况. 比如近代俄罗斯的圣彼得堡数学学派, 包括切比雪夫, 马尔科夫和李雅普诺夫, 他们在概率论等方向均作出了杰出的贡献. 中国也有解析数论学派, 以华罗庚为首, 包括陈景润, 王元, 潘承洞等人, 他们在哥德巴赫猜想等问题上作出了杰出的成果.

并且当前中美展开了贸易战, 美国有可能通过拒绝中国数学学生乃至研究人员的签证来阻断学术交流, 从而打击中国的数学学科.

因此想要发展数学, 就需要中国自己建立优秀的数学系, 建立优秀的数学研究所, 招募优秀的教师, 吸引并培养优秀的毕业生, 以此良性循环, 使学科越来越强. 进行学科建设是有必要的, 这是因为现在中国的数学学科乃至基础学科存在以下问题.

第一, 毕业生待遇不高. 一般数学博士毕业后只能在大学获得博士后或者助理教授的工作, 工资收入相比起统计, 金融, 软件等热门行业差了许多, 而且工作也不稳定, 如果不能及时发表优秀的文章就可能丢掉工作. 所以目前数学学科的吸引力并不是很高, 只有真正热爱数学的人才会走上这一条道路. 哪怕是大学本科是数学专业的同学, 也只有很少的一部分会继续修读数学, 这之中又只有很少的一部分最后会从事数学相关的工作, 这其中有大量的人才流失.

第二, 教授的琐事很多, 难以专心学术. 实际上一个教授在大学数学系的整个数学生涯, 做出最好成果的时间段一般是刚进入大学获得教职的时候. 那时候他们可以专心搞学术, 不用把过多的精力放在评职称申请经费拉项目等等琐事上, 因此可以做出较好的成果.

我觉得提高数学从业人员的待遇, 减少数学工作以外过多的琐事是建设数学学科的好方法.


    \section{新青年应立下志向努力学习}

    数学家一生中的大成果多出于青年时代. 习总书记在五四运动100周年大会上的讲话中提到牛顿和莱布尼茨发现微积分时分别是22岁和28岁. 我们也知道年仅26岁的阿贝尔提出了群的理论证明了5次代数方程没有求根公式, 结论很震撼, 所用方法也统领了新的数学分支的发展, 现在交换群就以阿贝尔的名字命名, 称为阿贝尔群. 年仅20岁的伽罗瓦提出了伽罗瓦理论, 将代数方程的解与循环群联系起来, 被推广到数学的其他领域.

中国近现代的一些著名数学家也是如此. 陈省身出生于1911年, 他最大的贡献有两个, 第一个是在1945年(34岁)推广了Gauss-Bornet定理到高维情形, 现称为Chern-Gauss-Bornet定理, 该定理陈省身的名字写在赫赫有名的高斯之前也足以表明他作出的贡献之大. 第二个是在1946年(35岁)提出了陈氏类, 该概念广泛应用在物理学, Calabi-Yau流形等领域. 陈景润出生于1933年, 他最大的贡献是在1966年(33岁)证明了"$1 + 2$", 是用筛法做哥德巴赫猜想的最好结果, 现称为陈氏定理.

数学大家们能在青年时代就做出大成果离不开他们早年立下坚定志向, 并努力学习积淀, 最后他们能捕捉到不易遇见的微光, 想出别人想不到的思路, 从而做出举世瞩目的成果.

诚然, 能成为数学大家的人是极少数人, 但是数学学科发展也需要默默付出, 给大牛们补充细节计算例子, 完善数学大厦的普通人. 中国需要有青年人立下坚定志向从事数学事业, 并且能在本科阶段就多学习学校课程以外的内容, 多接触现代数学, 这样能更早地进入研究状态, 从而能对新时代中国的数学学科作出更多的贡献.


    \section*{结语}
    \addcontentsline{toc}{section}{结语}

    针对问题一, 本文设计了一个基于快速聚类算法的静态推荐模型,
实现了推送给电视频道与其相似度高的广告.

针对问题二, 本文通过分析设置保留价对竞价带来的变化, 退出保留价应与卖方估价相近,
从而得到合理保留价.

针对问题三, 本文设计了一个基于协同过滤和深度学习的动态推荐模型,
比传统算法能更有效地去刻画广告和用户的特征.

针对问题四, 本文从经济机制理论出发, 推广了格罗夫斯机制,
结合本问题的经济环境得到了具体的经济机制.


    % -------------------- Bibliography --------------------

    \newpage
    \bibliography{bibliography}
    \bibliographystyle{plain}

\end{document}
