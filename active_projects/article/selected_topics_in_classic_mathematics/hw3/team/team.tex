% !Mode::"TeX:UTF-8"
\newcommand{\TITLE}{小组作业3}
\newcommand{\AUTHOR}{Team 29}
\newcommand{\SUBJECT}{经典数学专题选讲}
\documentclass[a4paper,12pt]{ctexart}
\usepackage{amsmath}
%\usepackage{amsthm} %定理格式 由ntheorem代替
\usepackage{amssymb}
\usepackage[thmmarks, amsmath, thref]{ntheorem}
\usepackage{DefaultTheoremStyle}
\usepackage{lastpage}
\usepackage{makecell} %表格线加粗 \Xhline{1.2pt}
\usepackage{boldline} %长表格表格线加粗
\usepackage{multirow} %合并单元格
\usepackage{array}
\usepackage{longtable} %长表格
\usepackage[dvipsnames]{xcolor} %颜色声明
\usepackage{varioref} %For Cross References
\renewcommand{\reftextbefore}
    {on the \reftextvario{preceding page}{page before}}
\renewcommand{\reftextafter}
    {on the \reftextvario{following}{next} page}
\renewcommand{\reftextfacebefore}
    {on the \reftextvario{facing}{preceding} page}
\renewcommand{\reftextfaceafter}
    {on the \reftextvario{facing}{next}{page}}
\renewcommand{\reftextfaraway}[1]
    {on page \pageref{#1}}
\usepackage{caption} %题注
\captionsetup{margin    =   6pt,
              font      =   small,
              labelfont =   bf}
\usepackage{fancyhdr} %脚注
\setlength{\headheight}{15pt}
\lhead{Copyright \copyright\ \AUTHOR}
\rhead{Page \thepage\ of \pageref{LastPage}}
\usepackage[square, numbers, sort&compress]{natbib} %引用
%\renewcommand{\bibsection}{} %不显示"Reference"
\usepackage{hyperref}
\hypersetup{linktoc             =   all,
            colorlinks          =   true,
            linkcolor           =   TealBlue,
           %anchorcolor         =   Black,
            citecolor           =   Blue,
           %filecolor           =   Cyan,
           %menucolor           =   Red,
           %runcolor            =   filecolor,
            urlcolor            =   magenta,
            pdfinfo             =   {
                Title           =   {\TITLE},
                Author          =   {\AUTHOR},
                Subject         =   {\SUBJECT}},
            bookmarksnumbered   =   true,
            pdfstartview        =   FitH,
            pdfpagelayout       =   OneColumn}
\usepackage[section]{placeins} % 使图像不会显示在别的部分 若过于严格则换成[below]\
\usepackage{graphicx}
\graphicspath{{figures/}} %图像文件目录
\usepackage[section]{placeins} % 使图像不会显示在别的部分 若过于严格则换成[below]
%\usepackage{fontspec} % 字体
\usepackage{titlesec} %Section标题格式
\usepackage{SUBSubsubsection}
\usepackage{authblk} %作者
\usepackage{stackrel} %上下写
\usepackage{mathtools} %\xleftrightarrow
%\usepackage{enumitem} 用enumerate包代替
\usepackage{listings} %排版程序语言
\usepackage{enumerate}
\usepackage{flafter} %不让float出现在定义之前的地方
\usepackage{float} %你们这帮float给我乖乖听话 HHHHHHHHHHH
\usepackage{pgfplots}
\pgfplotsset{width=7cm}
\usepackage{bigfoot} % to allow verbatim in footnote
\usepackage[numbered, framed]{matlab-prettifier}
\usepackage{filecontents}
\usepackage[all,cmtip]{xy} % Commutive diagram.
\usepackage{lineno} % Line numbers.
\newcommand{\diff}{\mathop{}\!\mathrm{d}}
\newcommand{\ratio}{\frac{a_{n+1}}{a_n}}
\newcommand{\maxf}{\max_{x\in[a,b]}{f(x)}}
\newcommand{\E}{E_{\varepsilon}}

%===============TITLE===============
\title{\TITLE}
\author{\AUTHOR}
\date{\today}

\begin{document}
    \maketitle
    \thispagestyle{empty}
    \newpage

    \pagestyle{fancy}
    \pagenumbering{arabic}
    \linenumbers
    
    \begin{problem}
        设$f(x), g(x)$是$[a,b]$上的非负连续函数, 且$f,g$只有有限个零点.
        记$\displaystyle a_n=\int_a^b{(f(x))^ng(x)\diff x}$,
        证明:$\displaystyle \lim_{n\to\infty}\ratio=\maxf.$
    \end{problem}

    \begin{theorem}
        $a_n>0, \forall n.$
    \end{theorem}

    \begin{proof}
        首先$a_n\geq 0, \forall n$.
        假设如果存在$n$使得$a_n=0$, 那么根据非负连续函数积分的性质, 有
        \begin{equation}
            (f(x))^ng(x)=0, \forall x\in [a, b].
        \end{equation}
        与$f,g$只有有限个零点矛盾!
    \end{proof}

    \begin{theorem}
        \label{leq}
        对于任意的$n$, 均有
        \begin{equation}
            \ratio \leq \max_{x\in [a, b]}f(x).
        \end{equation}
    \end{theorem}

    \begin{proof}
        根据积分第一中值定理, 有
        \begin{equation}
            \ratio = \frac{f(\xi_n)a_n}{a_n} = f(\xi_n) \leq \maxf.
        \end{equation}
    \end{proof}

    \begin{theorem}
        \label{ext}
        极限$\lim_{n\to\infty}{(a_{n+1}/a_n)}$存在.
    \end{theorem}

    \begin{proof}
        根据Cauchy-Schwarz's inequality, 我们有
        \begin{equation}
        \begin{aligned}
            a_{n+2}\cdot a_{n}
            &= \int_a^b{(f(x))^{n+2}g(x)\diff x}\int_a^b{(f(x))^{n}g(x)\diff x}\\
            &\geq (\int_a^b{(f(x))^{n+1}g(x)\diff x})^2
            = a_{n+1}^2.
        \end{aligned}
        \end{equation}

        故$\displaystyle\ratio$单调递增有上界, 极限存在.
    \end{proof}

    \begin{lemma}
        设$f$在闭区间$[a,b]$上非负连续, 非空开集$O\subseteq [a,b]$,
        $b_n=\displaystyle \int_O (f(x))^n\diff x$, 则
        $\displaystyle \lim_{n\to\infty}\sqrt[n]{b_n}$存在, 并且
        \begin{equation}
            \lim_{n\to\infty}\sqrt[n]{b_n}=\sup_{x\in O}{f(x)}.
        \end{equation}
    \end{lemma}

    \begin{proof}
        将$O$写成至多可数个互不相交的开区间之并$O=\bigcup_j{I_j}$,
        由积分中值定理, 有
        \begin{equation}
        \begin{aligned}
            \sqrt[n]{b_n}&=\sqrt[n]{\sum_{j}\int_{I_j}{(f(x))^n\diff x}}\\
            &= \sqrt[n]{\sum_{j}m(I_j)(f(\xi_j))^n}\\
            &\leq \sqrt[n]{m(O)}\cdot\sup_{x\in O}{f(x)}.
        \end{aligned}
        \end{equation}
        故
        \begin{equation}
            \limsup_{n\to\infty}{\sqrt[n]{b_n}}\leq\sup_{x\in O}{f(x)}.
        \end{equation}

        设$\varepsilon >0$,
        记$\displaystyle\E=\{f>\sup_{x\in O}{f(x)}-\varepsilon\}$为非空开集,
        设开区间$J_{\varepsilon}\subseteq\E$.
        于是
        \begin{equation}
        \begin{aligned}
            \sqrt[n]{b_n}
            &\geq \sqrt[n]{\int_{J_{\varepsilon}}{(f(x))^n\diff x}}\\
            &\geq \sqrt[n]{m(J_{\varepsilon})}
                \cdot (\sup_{x\in O}{f(x)}-\varepsilon)
        \end{aligned}
        \end{equation}
        故
        \begin{equation}
            \liminf_{n\to\infty}{\sqrt[n]{b_n}}\geq\sup_{x\in O}{f(x)}-\varepsilon.
        \end{equation}
        由于$\varepsilon$的任意性, 所以
        \begin{equation}
            \liminf_{n\to\infty}{\sqrt[n]{b_n}}\geq\sup_{x\in O}{f(x)}.
        \end{equation}
        
        最后由夹逼原理, 有$\displaystyle \lim_{n\to\infty}\sqrt[n]{b_n}$存在, 并且
        \begin{equation}
            \lim_{n\to\infty}\sqrt[n]{b_n}=\sup_{x\in O}{f(x)}.
        \end{equation}
    \end{proof}

    \begin{lemma}
        \label{rudin}
        设$\{a_n\}$为正项数列, 那么
        \begin{equation}
            \liminf_{n\to\infty}{\frac{a_{n+1}}{a_n}}
            \leq\liminf_{n\to\infty}{\sqrt[n]{a_n}}
            \leq\limsup_{n\to\infty}{\sqrt[n]{a_n}}
            \leq\limsup_{n\to\infty}{\frac{a_{n+1}}{a_n}}
        \end{equation}
    \end{lemma}

    \begin{proof}
        见\citep{rudin_pma}[THEOREM 3.37].
    \end{proof}

    \begin{theorem}
        \label{sqrt_limit}
        极限$\displaystyle\lim_{n\to\infty}\sqrt[n]{a_n}$存在, 并且
        \begin{equation}
            \lim_{n\to\infty}\sqrt[n]{a_n}=\maxf.
        \end{equation}
    \end{theorem}

    \begin{proof}
        设$\E =\{g>\varepsilon\}, \varepsilon>0$. 因为$g$连续, 所以$\E$为开集,
        故可写成至多可数个互不相交的开区间之并, 即
        \begin{equation}
            \E = \bigcup_j{I_j^{\varepsilon}};
        \end{equation}
        并且因为$g$非负且仅有有限个零点,
        所以对于充分小的$\varepsilon$都有$\E$非空, 即$m(\E)>0$.

        于是我们有
        \begin{equation}
        \label{sqrt_geq}
        \begin{aligned}
            \sqrt[n]{a_n}
            &\geq \sqrt[n]{\int_{\E}{(f(x))^{n}g(x)\diff x}}\\
            &= \sqrt[n]{\sum_{j}
                {g(\xi_j)\int_{I_j^{\varepsilon}}{(f(x))^n\diff x}}}\\
            &\geq \sqrt[n]{\varepsilon}\cdot
                \sqrt[n]{\sum_{j}{\int_{I_j^{\varepsilon}}{(f(x))^n\diff x}}}\\
            &= \sqrt[n]{\varepsilon}\cdot\sqrt[n]{\int_{\E}{(f(x))^n \diff x}}.
        \end{aligned}
        \end{equation}

        因为$\displaystyle \bigcup_{\varepsilon >0}{\E} = \{g\geq 0\}=[a,b]$,
        所以
        \begin{equation}
            \sup_{\varepsilon >0}\sup_{x\in \E}{f(x)} = \maxf.
        \end{equation}

        于是对式(\ref{sqrt_geq})两边取下极限可得
        \begin{equation}
            \liminf_{n\to\infty}{\sqrt[n]{a_n}} \geq \maxf.
        \end{equation}

        根据引理\ref{rudin}和定理\ref{leq},
        有极限$\lim_{n\to\infty}\sqrt[n]{a_n}$存在, 并且
        \begin{equation}
            \lim_{n\to\infty}\sqrt[n]{a_n}=\maxf.
        \end{equation}
    \end{proof}

    \begin{corollary}
        \begin{equation}
            \lim_{n\to\infty}{a_n} = \maxf.
        \end{equation}
    \end{corollary}

    \begin{proof}
        定理\ref{ext}引理\ref{rudin}和定理\ref{sqrt_limit}的直接推论.
    \end{proof}

    \begin{thebibliography}{9}
        \bibitem{rudin_pma}
            W.Rudin. Principles of Mathematical Analysis[M].
            McGraw-Hill, 1976: P68.
    \end{thebibliography}

\end{document}
