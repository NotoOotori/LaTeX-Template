% !TeX root = ../main.tex

\section{基数}

\begin{proposition}[{{\parencite[4]{jacobson_basic_1989}}}]\label{prop:cardcartcor}
    如果$A$非空, $B$为无限集并且$\vert B\vert\geq \vert A\vert$, 则$\vert A\times B\vert=\vert B\vert$.
\end{proposition}

\section{域论}

\begin{proposition}\label{thm:fntfieldextdegree}% TODO: reference
    给定域扩张$F\subseteq E$, 如果有不同的元素$x, y\in E$, 则$[F(x, y):F]\leq [F(x):F][F(y):F]$.
\end{proposition}

\begin{theorem}\label{thm:fieldextdegreemulti}% TODO: reference
    域扩张的次数具有可乘性, 即若有三个域$F\subseteq E\subseteq K$, 则$[K:F]=[K:E][E:F]$.
\end{theorem}

\section{拓扑}

闭集其实是一个局部性质.

\begin{proposition}\label{prop:closedsetlocal}
    设$X$为拓扑空间, $Y\subseteq X$, 则$Y$是$X$的闭集当且仅当存在$X$中的开覆盖$\bigcup U_\lambda\supseteq Y$使得每个$U_\lambda\cap Y$都是$U_\lambda$的闭集.
\end{proposition}

\begin{proposition}[{{\parencite[104, Theorem 18.1]{munkres_topology_2000}}}]\label{prop:continuousimageclosure}
    如果$f\colon X\to Y$是拓扑空间之间的映射, 则$f$连续当且仅当对于任意的集合$A\subseteq X$都有$f(\overline{A})\subseteq\overline{f(A)}$.
\end{proposition}
