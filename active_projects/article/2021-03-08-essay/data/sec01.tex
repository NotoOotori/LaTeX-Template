% !TeX root = ../main.tex

\section{引言}

代数几何, 顾名思义, 主要研究的是代数方程或者代数方程组解集的几何性质. 这是一门历史悠久的学科, 经历了多轮语言和观点的蜕变. 关于代数几何的历史与发展, 主要参考\parencite[18-113, V-VIII]{sally_history_1985}与\parencite[xiii, Introduction]{hartshorne_algebraic_1977}.

代数几何这门学科的最大发展来源于Riemann在19世纪中期对代数函数与Abel积分的研究. Riemann在处理多值函数的积分时, 划时代地提出了Riemann面的概念, 并通过函数论的方法发现了Riemann面上的半纯函数满足一个多项式关系, 于是Riemann自然地考虑了Riemann面上被后人称为半纯(或者有理)函数域的结构, 并研究了两条代数曲线的双有理同构, 这开创了双有理几何这一分支, 统领了之后80年代数几何的研究.

在Riemann去世后, 因为Riemann的工作是如此的伟大, 以致于后人无法完全继承发扬他提出的纲领, 于是就衍生出了不同的学派. 他们分别专注于Riemann的在双有理几何中的某一点研究, 用自己的语言解释并试图重新证明Riemann在平面代数曲线中的某些结果, 并尝试推广到代数曲面乃至任意维数的代数簇上.

Kronecker, Dedekind和Weber的代数学派对于现代代数几何诞生的影响是最大的. Kronecker试图给数论和代数几何统一建立代数理论的根基, 他从代数角度给出了不可约簇和维数的定义. Dedekind与Weber试图对Riemann的曲线理论给出纯代数的证明. 他们的出发点非常高, 有一系列的想法成为了现代理论的根基. Riemann面的有理函数域是复数域$\CC$上的有理分式域$\CC(x)$的有限域扩张, 他们反过来认为黎曼面在同构下的不变量会恰好由某一个这样的域扩张的性质来刻画, 于是他们想要利用给定的有限域扩张, 来重构一个有理函数域恰为这个扩域的黎曼面.

Roch和Clebsch将Riemann的成果与平面代数曲线的射影几何联系到一起, 研究其双有理不变量以及在双有理同构意义下的分类, 吸引了许多几何学派到这方面的研究. 值得一提的是第二代Italy几何学家Castelnuovo, Enriques和Severi, 他们的研究, 用现代语言来说, 是有关于曲线的线性系统及其推广.

Cayley, Clebsch, M. Noether和Picard想要把Riemann对曲线上Abel积分的研究推广到曲面情况, 这被称为``超越"理论. 后来, 在H.Poincar\'e引入单纯复形之后, 后人发现对曲面上的积分的一些研究, 实际上就是对曲面的同调的研究.

代数几何在二十世纪初期的发展来源于``结构"这一思想的广泛普及, 使得理论得到重构和加深. 微分流形这一内蕴概念得到了精确定义, 微分流形的同调, 外微分形式及其关系得到了研究. Hodge推广了Riemann的方法, 研究了紧微分流形的同调的另一种刻画, 并将其应用到代数簇上. 首先Hodge严格证明了可定向紧Riemann流形的调和$p$形式全体$\symbb{H}^p$与上同调类$H^p(\Lambda)$的线性空间结构是同构. 后来, 随着E. K\"ahler发现了复射影空间上的一种特殊的Hermite度量, 带着这样的度量的全纯流形被称为K\"ahler流形. Hodge随后得出了在K\"ahler流形上的调和$p$形式的一系列好结果.

同时期, 抽象代数这一学科出现了, 将代数结构作为代数学的根本概念, 这也将代数几何所讨论的基域从复数域推广到了一般的域, 主要的研究对象即为域上多项式环的理想或者齐次理想. Hilbert零点定理建立了根基理想与代数集之间的一一对应, 于是代数集的分解就对应到了理想的分解, 而E. Noether最终得出了E. Noether环上准素分解的存在唯一性.

在二十世纪中叶及后半段, 随着代数拓扑, 微分几何等理论的发展, 代数几何所使用的语言和技术也走向了抽象, 已经很不容易看出其与本初的几何直观之间的关系. H. Cartan和Serre用层的语言简洁漂亮地重述了全纯流形理论, 并得出了一些新的结果. Serre利用H. Cartan提出的赋环空间的概念, 改善了A. Weil提出的赋有Zariski拓扑的抽象簇, 并在上面建立了层上同调理论. 最终, A. Grothendick整合了前人的理论, 划时代地给出了仿射概形与概形的定义, 这也成为了现代代数几何的主要研究对象.

本文的主要内容是研究在概型出现之前的古典代数几何与交换环论之前的联系. 几何的直观性可以帮助理解代数的理论, 而代数的刚性可以做出更精细的结论. 本文第一部分将建立所需要的代数理论, 主要包括环的基础概念, 素谱, 局部化, Noether环中的准素分解, 分次环的分次理想与局部化, 域上多项式环中的维数理论等. 第二部分将介绍几何概念, 主要研究对象为仿射空间和射影空间中的代数簇或拟代数簇, 将介绍不可约, 维数等重要概念. 第三部分将建立代数理论与几何直观的联系, 找到其中存在的对应关系, 并指出可能存在的区别.

本文中的定义大多采用如``如果\fbox{对象}满足\fbox{性质}, 那么称\fbox{对象}为\fbox{\emph{名称}}"的句式, 虽然逻辑上应该将关联词``如果\dots\dots 那么\dots\dots"替换为``仅当\dots\dots 才\dots\dots ", 不过为了方便以及符合平时的习惯, 我们仍然运用这个逻辑上存在问题的句式.

本文中采用的集合论公理系统为包含选择公理的Zermelo-Fraenkel公理系统, 简称为ZFC公理系统, 为了保证非零环中极大理想的存在性(\thref{thm:maxideal}).

% TODO: 记号 用X表示未定元
