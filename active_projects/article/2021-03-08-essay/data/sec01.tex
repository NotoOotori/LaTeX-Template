% !TeX root = ../main.tex

\section{引言}

代数几何是一门历史悠久而又充满活力的学科, 不过现代代数几何的语言和技术已经与其直观背景有高度的不同, 导致新人要进行非常长时间的学习才能入门. 本文旨在提供一个本科生和低年级研究生级别的代数几何入门读物, 只假设读者有一定的抽象代数基础并尽量做到自洽.

本文的第一部分将以域的有限生成整环为原型, 发展本文中所需要用到的代数基础. 第一节中我们将研究一般的交换幺环的基础理论, 包括素理想, 极大理想, 整环等基础概念, 以及商环, 分式环之类的基础的技术, 我们将证明环与其商环或分式环之间都存在某种理想之间的对应关系, 并将证明商环与分式环可交换, 以及累次分式环与做一次分式环是等价的. 在第一节的最后我们将简要提及齐次环与齐次理想的概念, 为了射影簇等概念做准备.

第二节中我们首先将引入模的概念以及技术, 研究有限生成模的性质并介绍了整元的概念. 接着我们将讨论模的升链条件, 并重点研究Noether环的性质, 先证明Hilbert基定理, 即域的有限生成整环都是Noether环, 再证明Lasker-Noether定理, 即Noether环的理想存在准素分解, 最后证明一般的环上的准素分解都满足伴随素理想以及孤立准素理想的唯一性.

第三节中我们将对域的有限生成整环的维数理论展开研究. 首先我们将讨论域扩张的超越基与超越次数, 将代数元的概念抽象为了集合中的对应关系, 并在抽象的情况下证明了基的基数是唯一的. 接着我们将定义域的有限生成整环的维数即为其分式域关于基域的域扩张的超越次数, 并定义素理想的高度及深度, 利用域扩张的理论和Noether正规化引理证明素理想的高度深度与域的有限生成整环的维数之间的关系, 并用此证明Hilbert零点定理.
% 主要包括局部化, 准素分解, 维数理论等内容.

\bigskip

本文的第二部分是几何中的基本概念. 前两节中我们将分别讨论仿射空间的代数集以及射影空间的代数集, 并将拓扑空间中的不可约, 维数等概念应用其中, 得到代数集的不可约分解, 以及代数簇的维数.

第三节与第四节我们将重点研究代数簇上的函数结构, 先定义代数簇到基域上的正则函数, 并由此定义代数簇之间的正则函数, 我们还可以定义代数簇上的整体函数环, 局部环以及有理函数域, 它们在一定程度上能反过来刻画代数簇的结构, 并由于Zariski拓扑的刚性, 这三个环之间还有嵌入关系. 在最后将我们定义并简单研究代数簇之间的有理函数.

\bigskip

本文的第三部分是代数与几何的联系, 这一部分我们将慢慢展现代数与几何之间的联系. 第一节中我们将利用Hilbert零点定理建立起仿射空间中的代数集与多项式环的根式理想之间的反序双射, 射影空间中也有代数集与除了无关极大理想之外的齐次根式理想之间的反序双射, 并且这个双射将素理想对应到代数簇, 我们就可以将代数中的准素分解理论与几何中的不可约分解联系起来.

第二节中我们将定义代数簇的坐标环, 并探究其与函数环之间的关系, 由此我们首先将发现分式环与局部化的几何意义, 即只考虑定义在某些点以外的函数, 其次我们还将代数中的维数理论与代数簇的维数联系起来, 于是我们可以探究超曲面的交的不可约分支的维数. 甚至我们还将证明两个范畴等价, 分别是固定代数闭域$\kk$的仿射代数簇关于代数簇的态射构成的范畴与$\kk$的有限生成整环关于代数同态构成的范畴之间的范畴等价, 以及固定代数闭域$\kk$的代数簇关于有理映射构成的范畴与$\kk$的有限生成代数关于代数同态构成的范畴之间的范畴等价. 在过程中, 我们将顺便证明每个代数簇都有由仿射开子集构成的拓扑基.

前两节中我们发现几何的结论没有代数的结论那么精细, 并且对于对象的要求也比代数上要多许多. 几何学家认为是以前的几何框架限制了他们的直觉, 因此要扩充几何的研究对象, 这就最终引出了概型的概念. 很自然地, 概型应该也是一个有函数结构的拓扑空间, 并且在局部上是仿射的, 而仿射的东西我们认为是一个环的素谱, 于是在第一小节我们将简要介绍素谱上的Zariski拓扑, 读者可以发现其与仿射空间的类似之处. 函数结构是由层来定义, 于是第二小节我们将介绍层与茎的概念以及层之间的态射. 由于代数簇在某一点处的局部环是局部环, 我们也希望概型保留类似的性质, 所以最后一小节我们将引入局部赋环空间的概念及其中的态射, 并对环的素谱定义局部赋环空间的结构, 初步研究其性质.

\bigskip

附录中我们将讲述一些放在正文中略显突兀的内容. 第一节中我们将不加证明地列举一些读者应该了解但可能不是很熟悉的命题. 第二节中我们将介绍范畴论的初步内容, 它能帮助我们更深刻的理解相关概念及代数与几何之间的联系: 第一小节简要叙述范畴论的基本概念, 包含范畴的定义, 范畴之间的函子, 函子之间的自然变换和范畴等价; 第二小节我们主要研究可表函子, 通过研究函子和态射函子之间的自然构造我们得到了Yoneda引理, 并最后介绍了如何利用Yoneda引理从可表函子中导出映射的泛性质; 第三小节我们主要研究范畴论的极限, 定义了滤子化上极限, 并在值域为集合范畴的情况下给出了滤子化上极限的构造, 最后论述了利用更深刻的范畴论知识可以证明在某些代数范畴上滤子化上极限的构造与集合范畴中的相同, 由此我们做好了为层论的建立做好了准备.


% 代数几何, 顾名思义, 主要研究的是代数方程或者代数方程组解集的几何性质. 这是一门历史悠久的学科, 经历了多轮语言和观点的蜕变. 关于代数几何的历史与发展, 主要参考\tjucite[18--113, V-VIII]{sally_history_1985}与\tjucite[xiii, Introduction]{hartshorne_algebraic_1977}.

% 代数几何这门学科的最大发展来源于Riemann在19世纪中期对代数函数与Abel积分的研究. Riemann在处理多值函数的积分时, 划时代地提出了Riemann面的概念, 并通过函数论的方法发现了Riemann面上的亚纯函数满足一个多项式关系, 于是Riemann自然地考虑了Riemann面上被后人称为亚纯(或者有理)函数域的结构, 并研究了两条代数曲线的双有理同构, 这开创了双有理几何这一分支, 统领了之后80年代数几何的研究.

% 在Riemann去世后, 因为Riemann的工作是如此的伟大, 以致于后人无法完全继承发扬他提出的纲领, 于是就衍生出了不同的学派. 他们分别专注于Riemann的在双有理几何中的某一点研究, 用自己的语言解释并试图重新证明Riemann在平面代数曲线中的某些结果, 并尝试推广到代数曲面乃至任意维数的代数簇上.

% Kronecker, Dedekind和Weber的代数学派对于现代代数几何诞生的影响是最大的. Kronecker试图给数论和代数几何统一建立代数理论的根基, 他从代数角度给出了不可约簇和维数的定义. Dedekind与Weber试图对Riemann的曲线理论给出纯代数的证明. 他们的出发点非常高, 有一系列的想法成为了现代理论的根基. Riemann面的有理函数域是复数域$\CC$上的有理分式域$\CC(x)$的有限域扩张, 他们反过来认为黎曼面在同构下的不变量会恰好由某一个这样的域扩张的性质来刻画, 于是他们想要利用给定的有限域扩张, 来重构一个有理函数域恰为这个扩域的黎曼面.

% Roch和Clebsch将Riemann的成果与平面代数曲线的射影几何联系到一起, 研究其双有理不变量以及在双有理同构意义下的分类, 吸引了许多几何学派到这方面的研究. 值得一提的是第二代Italy几何学家Castelnuovo, Enriques和Severi, 他们的研究, 用现代语言来说, 是有关于曲线的线性系统及其推广.

% Cayley, Clebsch, M. Noether和Picard想要把Riemann对曲线上Abel积分的研究推广到曲面情况, 这被称为``超越"理论. 后来, 在H.Poincar\'e引入单纯复形之后, 后人发现对曲面上的积分的一些研究, 实际上就是对曲面的同调的研究.

% 代数几何在二十世纪初期的发展来源于``结构"这一思想的广泛普及, 使得理论得到重构和加深. 微分流形这一内蕴概念得到了精确定义, 微分流形的同调, 外微分形式及其关系得到了研究. Hodge推广了Riemann的方法, 研究了紧微分流形的同调的另一种刻画, 并将其应用到代数簇上. 首先Hodge严格证明了可定向紧Riemann流形的调和$p$形式全体$\symbb{H}^p$与上同调类$H^p(\Lambda)$的线性空间结构是同构. 后来, 随着E. K\"ahler发现了复射影空间上的一种特殊的Hermite度量, 带着这样的度量的全纯流形被称为K\"ahler流形. Hodge随后得出了在K\"ahler流形上的调和$p$形式的一系列好结果.

% 同时期, 抽象代数这一学科出现了, 将代数结构作为代数学的根本概念, 这也将代数几何所讨论的基域从复数域推广到了一般的域, 主要的研究对象即为域上多项式环的理想或者齐次理想. Hilbert零点定理建立了根基理想与代数集之间的一一对应, 于是代数集的分解就对应到了理想的分解, 而E. Noether最终得出了E. Noether环上准素分解的存在唯一性.

% 在二十世纪中叶及后半段, 随着代数拓扑, 微分几何等理论的发展, 代数几何所使用的语言和技术也走向了抽象, 已经很不容易看出其与本初的几何直观之间的关系. H. Cartan和Serre用层的语言简洁漂亮地重述了全纯流形理论, 并得出了一些新的结果. Serre利用H. Cartan提出的赋环空间的概念, 改善了A. Weil提出的赋有Zariski拓扑的抽象簇, 并在上面建立了层上同调理论. 最终, A. Grothendick整合了前人的理论, 划时代地给出了仿射概型与概型的定义, 这也成为了现代代数几何的主要研究对象.

% 本文的主要内容是研究在概型出现之前的古典代数几何与交换环论之前的联系. 几何的直观性可以帮助理解代数的理论, 而代数的刚性可以做出更精细的结论. 本文第一部分将建立所需要的代数理论, 主要包括环的基础概念, 素谱, 局部化, Noether环中的准素分解, 分次环的分次理想与局部化, 域上多项式环中的维数理论等. 第二部分将介绍几何概念, 主要研究对象为仿射空间和射影空间中的代数簇或拟代数簇, 将介绍不可约, 维数等重要概念. 第三部分将建立代数理论与几何直观的联系, 找到其中存在的对应关系, 并指出可能存在的区别.

% 本文中的定义大多采用如``如果\framebox[\width]{对象}满足\framebox[\width]{性质}, 那么称\framebox[\width]{对象}为\framebox[\width]{\emph{名称}}"的句式, 虽然逻辑上应该将关联词``如果\dots\dots 那么\dots\dots"替换为``仅当\dots\dots 才\dots\dots ", 不过为了方便以及符合平时的习惯, 我们仍然运用这个逻辑上存在问题的句式.

% 本文中采用的集合论公理系统为包含选择公理的Zermelo-Fraenkel公理系统, 简称为\axiom{ZFC}公理系统, 为了保证非零环中极大理想的存在性(\thref{thm:maxideal}), 刻画理想的根基为包含其的素理想全体的并(\thref{prop:scheinnullstellensatz}), 偏序集的升链条件与极大条件的等价性(\thref{prop:chaincondition}).

% 关于本文的记号, 设集合$X\subset Y$, 我们将用$X-Y$表示集合$Y$在$X$中的余集, 即$\{x\in X\vert x\notin Y\}$; 我们将用大写字母$X$, $Y$, $X_n$来表示多项式环中的未定元, 故$A[X_1, \dotsc, X_n]$表示$A$的$n$元多项式环.% 而用小写的字母$x$, $y$, $x_n$表示元素, 故$A[x_1, \dotsc, x_n]$表示由$x_1, \dotsc, x_n$生成的$A$-代数.

% 本文中有部分与主题相关度不是很大的命题的叙述与证明会放入附录中, 附录的小节用大写英文字母编号, 附录中的命题编号形如\thref{thm:fieldextdegreemulti}, 与正文中的命题编号(如\thref{thm:tranbasis})相区别.

% 注意如果在不加说明的情况下, 一个对象若可以自然地赋有某种结构, 那么就默认赋有这种结构. 例如环$A$有自然的$A$-模结构, 那么我们在说$A$是一个$A$-模时, 会默认赋有自然的$A$-模结构.
