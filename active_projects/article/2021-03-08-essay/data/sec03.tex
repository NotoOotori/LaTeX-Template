% !TeX root = ../main.tex

\section{几何}

古典代数几何关心的主要内容就是仿射空间或者射影空间中, 由多项式方程或者多项式方程组的解集的点集的几何. 不同于分析中常见的针对某一个特定的多项式方程或方程组进行研究, 我们将把所有这样的解集放在一起定义一个拓扑空间, 并研究这个拓扑空间的结构, 包括不可约性, 维数以及函数结构.

由于基域非代数闭域的情况比较复杂, 所以这一部分中若未另加说明则只考虑基域为代数闭域的情况.

\subsection{仿射代数簇}

这一节我们将定义固定的代数闭域上的仿射代数簇, 定义以代数集为闭集的Zariski拓扑, 定义不可约闭集以及维数的概念.

设$\field{k}$为固定的代数闭域, 定义$\field{k}$上的$n$-\emph{仿射空间}$\AF^n$为$\field{k}^n$, 即$\field{k}$的$n$元组全体. 称一个$P\in \AF^n$为一个\emph{点}, 若$P=(x_1, \dotsc, x_n)$, 则称这些$x_j\in\field{k}$为$P$的\emph{坐标}.

设$A:=\field{k}[X_1, \dotsc, X_n]$为$\field{k}$的$n$元多项式环, 我们可以将$A$的元素看成从$\AF^n$到$k$的映射, 对于$f\in A$, $P\in \AF^n$, 记$f(P)=f(x_1, \dotsc, x_n)$. 于是我们可以讨论$f\in A$的\emph{零点}, 即为$\AF^n$的子集$Z(f):=\{P\in \AF^n\vert f(P)=0\}$. 更一般地, 我们还可以讨论$A$的子集$T$的\emph{公共零点集}$Z(T):=\{P\in \AF^n\vert f(P)=0, \forall f\in T\}$. 如果$\AF^n$的子集$Y$可以写成$Y=Z(T)$的形式, 则称$Y$为\emph{代数集}.

\begin{proposition}[{{\parencite[2, Proposition 1
    1]{hartshorne_algebraic_1977}}}]\label{prop:zariskitopology}
    $\AF^n$的代数集满足拓扑空间中闭集的公理, 即
    \begin{enumerate}
        \item\label{enum:prop-zariski-topology-1} 空集与全集是代数集;
        \item\label{enum:prop-zariski-topology-2} 有限个代数集的并是代数集;
        \item\label{enum:prop-zariski-topology-3} 任意代数集的交是代数集.
    \end{enumerate}
    如此可以在$\AF^n$定义以代数集全体为闭集全体的拓扑空间结构, 称为\emph{Zariski拓扑}.
\end{proposition}

\begin{proof}
    \ref{enum:prop-zariski-topology-1} 观察到若取$f=1, g=0$, 则有$Z(f)=\varnothing, Z(g)=\AF^n$.

    \ref{enum:prop-zariski-topology-2} 只需证两个代数集的并是代数集. 设$Y_1=Z(T_1), Y_2 = Z(T_2)$, 记$T_1T_2=\{g_1g_2\vert g_1\in T_1, g_2\in T_2\}$, 则我们断言$Y_1\cup Y_2 = Z(T_1T_2)$. 先证$\subseteq$. 如果$P\in Y_1\cup Y_2$, 则存在$g_1\in T_1, g_2\in T_2$使得$g_1(P)=g_2(P)=0$, 于是$(g_1g_2)(P)=0$. 再证$\supseteq$. 因为$(g_1g_2)(P)=0$蕴含$g_1(P)=0$或者$g_2(P)=0$, 于是有$Z(T_1T_2)\subseteq Z(T_1)\cup Z(T_2)=Y_1\cup Y_2$.

    \ref{enum:prop-zariski-topology-3} 根据定义即有$\bigcap Z(T_\lambda)=Z(\bigcup T_\lambda)$.
\end{proof}

回顾二维欧氏空间中二次曲线的分类, 多项式$XY\in \RR[X, Y]$虽然是二次的, 但是它所定义的解集可以写成两条直线的并, 被称为是退化的. 类似地, 我们想先对非退化的代数集展开研究, 这就引出了拓扑空间中不可约的概念.

如果非空拓扑空间$X$不能写成两个真闭子空间的并, 则称$X$是\emph{不可约的}, 这等价于$X$中任意两个非空开集都有非空交, 也等价于$X$中每个非空开集都是稠密的. 称$X$的子集$Y$不可约当且仅当$Y$的子拓扑结构是\emph{不可约的}. 为了叙述方便, 我们不认为空集是不可约的. $\AF^n$的不可约闭子集被称为\emph{仿射代数簇}, 仿射代数簇的开子集被称为\emph{拟仿射代数簇}.

\begin{remark}
    这里仿射代数簇与拟仿射代数簇的定义依赖于到$\AF^n$的嵌入, 等到()% TODO: where
    定义了簇的同构之后我们才可以给出仿射簇与拟仿射簇的内蕴定义.
\end{remark}

\begin{proposition}[{{\parencite[13, Exercise 20]{atiyah_introduction_1969}}}]
    拓扑空间$X$的不可约集全体在集合包含关系定义的偏序集结构有极大元, 这些极大元都是闭集, 并且它们的并为全空间. 称这些极大元为$X$的\emph{不可约分支}.
\end{proposition}

\begin{proof}
    先利用Zorn引理证明有极大元. 考虑不可约集的全序子链$\{Y_\lambda\}$, 则断言它们的并$Y$一定是不可约集. 这是因为对于每两个$Y$的真闭子集$Z_1, Z_2$, 都可以找到$\lambda$使得$Z_1\cap Y_\lambda, Z_2\cap Y_\lambda$是$Y_\lambda$的真闭子集.

    再证明不可约集的闭包仍然是不可约集, 于是极大元一定是闭集. 设$Y$为不可约集, 考虑$\overline{Y}$中的非空开子集$O_1, O_2$, 则$O_1\cap Y, O_2\cap Y$是$Y$的非空开子集, 有非空交, 故$O_1, O_2$也有非空交.

    最后根据定义知, 单点集一定是不可约集, 所以这些极大元的并一定是全空间.
\end{proof}

我们在这一部分中研究的拓扑空间都满足关于闭子集的降链条件, 即对于任意的闭集降链$Y_1\supseteq Y_2\supseteq\dotsb$都存在$n$使得$Y_n=Y_{n+1}=\dotsb$, 称这样的拓扑空间为\emph{Noether拓扑空间}.

\begin{proposition}[{{\parencite[5, Proposition 1.5]{hartshorne_algebraic_1977}}}]
    Noether拓扑空间$X$中的不可约分支仅有有限个.
\end{proposition}

\begin{proof}
    不会了, 参考Lasker-Noether定理的证明吧.
\end{proof}

拓扑空间$X$的\emph{维数}$\dim (X)$被定义为满足存在不可约闭子集升链$Z_0\subsetneqq Z_1\subsetneqq \dotsb \subsetneqq Z_n$的$n$的上确界. 定义仿射代数簇与拟仿射代数簇的\emph{维数}为它们作为拓扑空间的维数.

\begin{example}
    考虑$\AF^1$的Zariski拓扑. 由于$\field{k}$为代数闭域, 每个正次数的多项式都可以分解成一次多项式的乘积, 因此$\AF^1$的代数集全体为$\AF^1$的有限集全体并上空集以及全集. 故$\AF^1$中有且仅有全集和单点集是不可约闭集, 从而$\AF^1$的维数为1.
\end{example}
