% !TeX root = ../main.tex

\section{几何}\label{sec:geometry}

古典代数几何关心的主要内容就是仿射空间或者射影空间中由多项式方程或者多项式方程组的解集的点集的几何. 不同于分析中常见的针对某一个特定的多项式方程或方程组进行研究, 我们将把所有这样的解集放在一起定义一个拓扑空间, 并研究这个拓扑空间的结构, 包括不可约性, 维数以及函数结构.

由于基域非代数闭域的情况比较复杂, 所以这一部分中若未另加说明则只考虑基域为代数闭域的情况.

\subsection{仿射代数簇}\label{sec:varietyaffine}

这一节我们将定义固定的代数闭域上的仿射代数簇, 定义以代数集为闭集的Zariski拓扑, 定义不可约闭集以及维数的概念.

设$\kk$为固定的代数闭域, 定义$\kk$上的$n$-\emph{仿射空间}$\AF^n$为$\kk^n$, 即$\kk$的$n$元组全体. 称一个$P\in \AF^n$为一个\emph{点}, 若$P=(x_1, \dotsc, x_n)$, 则称这些$x_j\in\kk$为$P$的\emph{坐标}.

设$A\coloneq \kk[X_1, \dotsc, X_n]$为$\kk$的$n$元多项式环, 我们可以将$A$的元素看成从$\AF^n$到$k$的映射, 对于$f\in A$, $P\in \AF^n$, 记$f(P)=f(x_1, \dotsc, x_n)$. 于是我们可以讨论$f\in A$的\emph{零点}, 即为$\AF^n$的子集$Z(f)\coloneq \{P\in \AF^n\vert f(P)=0\}$. 更一般地, 我们还可以讨论$A$的子集$T$的\emph{零点集}$Z(T)\coloneq \{P\in \AF^n\vert f(P)=0, \forall f\in T\}$. 如果$\AF^n$的子集$Y$可以写成$Y=Z(T)$的形式, 则称$Y$为\emph{代数集}.

\begin{proposition}[{{\parencite[2, Proposition 1.1]{hartshorne_algebraic_1977}}}]\label{prop:affinezariskitopology}
  设$\Lambda$为任意指标集, $T_1, T_2\in A$, $\{T_\lambda\}_{\lambda\in\Lambda}\subseteq A$, 则
  \begin{enumerate}
    \item\label{enum:prop-affine-zariski-topology-1} $Z(A)=\varnothing, Z(0)=\AF^n$;
    \item\label{enum:prop-affine-zariski-topology-2} $Z(T_1\cap T_2)=Z(T_1)\cup Z(T_2)$;
    \item\label{enum:prop-affine-zariski-topology-3} $Z(\bigcup_{\lambda\in\Lambda} T_\lambda)=\bigcap_{\lambda\in\Lambda} Z(T_\lambda)$.
  \end{enumerate}
  于是$\AF^n$的代数集满足拓扑空间中闭集的公理, 如此可以在$\AF^n$定义以代数集全体为闭集全体的拓扑空间结构, 称为\emph{Zariski拓扑}.
\end{proposition}

\begin{proof}
  \ref{enum:prop-affine-zariski-topology-2} 设$Y_1=Z(T_1), Y_2 = Z(T_2)$, 记$T_1T_2=\{g_1g_2\vert g_1\in T_1, g_2\in T_2\}$, 则我们断言$Y_1\cup Y_2 = Z(T_1T_2)$. 先证$\subseteq$. 如果$P\in Y_1\cup Y_2$, 则存在$g_1\in T_1, g_2\in T_2$使得$g_1(P)=g_2(P)=0$, 于是$(g_1g_2)(P)=0$. 再证$\supseteq$. 因为$(g_1g_2)(P)=0$蕴含$g_1(P)=0$或者$g_2(P)=0$, 于是有$Z(T_1T_2)\subseteq Z(T_1)\cup Z(T_2)=Y_1\cup Y_2$.

  \ref{enum:prop-affine-zariski-topology-3} 根据定义即有$\bigcap Z(T_\lambda)=Z(\bigcup T_\lambda)$.
\end{proof}

回顾二维欧氏空间中二次曲线的分类, 多项式$XY\in \RR[X, Y]$虽然是二次的, 但是它所定义的解集可以写成两条直线的并, 被称为是退化的. 类似地, 我们想先对非退化的代数集展开研究, 这就引出了\ref{subsec:primespec}中定义过的拓扑空间中不可约的概念. $\AF^n$的不可约闭子集被称为\emph{仿射代数簇}, 仿射代数簇的开子集被称为\emph{拟仿射代数簇}.

\begin{remark}
  这里仿射代数簇与拟仿射代数簇的定义依赖于到$\AF^n$的嵌入.%, 等到()% TODO: where
  %定义了簇的同构之后我们才可以给出仿射簇与拟仿射簇的内蕴定义.
\end{remark}

为了后续讨论的方便, 我们需要陈述一个事实, 证明将放在下一部分\thref{prop:affinegaloisconnectionclosedradical}中.

\begin{proposition}\label{prop:affineprimeirreducible}
  如果$\ideal{p}$是$A$的素理想, 那么$Z(\ideal{p})$是$\AF^n$的不可约闭集. 特别地, $\AF^n$是仿射代数簇.
\end{proposition}

拓扑空间$X$的\emph{维数}$\dim (X)$被定义为满足存在不可约闭子集升链$Z_0\subsetneqq Z_1\subsetneqq \dotsb \subsetneqq Z_n$的$n$的上确界. 定义仿射代数簇与拟仿射代数簇的\emph{维数}为它们作为拓扑空间的维数.

\begin{example}
  考虑$\AF^1$的Zariski拓扑. 由于$\kk$为代数闭域, 每个正次数的多项式都可以分解成一次多项式的乘积, 因此$\AF^1$的代数集全体为$\AF^1$的有限集全体并上空集以及全集. 故$\AF^1$中有且仅有全集和单点集是不可约闭集, 从而$\AF^1$的维数为1.
\end{example}

\subsection{射影代数簇}

这一节我们将在射影空间上用类似于仿射空间的方法来定义射影空间上的Zariski拓扑, 从而引出射影代数簇等相关概念. 我们还将证明在同胚的意义下, $n$-射影空间可以被$n$-仿射空间开覆盖, 从而每个(拟)射影代数簇都可以被(拟)仿射代数簇开覆盖.

设$k$为固定的代数闭域, 定义$\kk$上的$n$-\emph{射影空间}$\PP^n$为商集$(\AF^{n+1}-\{(0, \dotsc, 0)\}){\divslash}{\sim}$, 其中两个点等价当且仅当它们在同一条过原点的直线上. 称一个$P\in\PP^n$为一个\emph{点}, 若$(n+1)$元组$(x_0, \dotsc, x_n)$在$P$对应的等价类中, 则称其为$P$的一组\emph{齐次坐标}.

设$S\coloneq \kk [X_0, \dotsc, X_n]$为$k$的$n$元多项式环, 与仿射情形不同, 我们一般不能把$S$的元素看成从$\PP^n$到$\kk$的映射, 为了定义零点集, 我们需要引入多项式的次数这一概念. 定义单项式的\emph{次数}为$X_0, \dotsc, X_n$的指数之和, 如果多项式$f\in \PP^n$可以写成有限个次数为$d$的单项式的和, 则称多项式$f$为\emph{次数为$d$的齐次多项式}, 称次数为一的齐次多项式为\emph{线性齐次多项式}, 称可以如此定义次数的多项式为\emph{齐次多项式}, 记齐次多项式全体为$\Shomo$. 次数为$d$的齐次多项式$f$满足性质$f(\lambda x_0, \dotsc, \lambda x_n)=\lambda^d f(x_0, \dotsc, x_n)$, 于是可以把$S$的元素看成从$\PP^n$到$\{0, 1\}$的映射, 其中$f(P)=0$当且仅当存在$P$的一组齐次坐标$(x_0, \dotsc, x_n)$使得$f(x_0, \dotsc, x_n)=0$. 从而我们可以讨论一个齐次多项式的\emph{零点}, 即$Z(f)=\{P\in \PP^n\vert f(P)=0\}$, 以及一个由齐次多项式构成的集合$T\subseteq\Shomo$的\emph{零点集}, 即$Z(T)=\{P\in\PP^n\vert f(P)=0, \forall f\in T\}$. 如果$\PP^n$的子集$Y$可以写成$Y=Z(T)$的形式, 则称$Y$为\emph{代数集}.

\begin{proposition}[{{\parencite[9, Proposition 2.1]{hartshorne_algebraic_1977}}}]
  设$\Lambda$为任意指标集, $T_1, T_2\in \Shomo$, $\{T_\lambda\}_{\lambda\in\Lambda}\subseteq \Shomo$, 则
  \begin{enumerate}
    \item $Z(\Shomo)=\varnothing, Z(0)=\AF^n$;
    \item $Z(T_1\cap T_2)=Z(T_1)\cup Z(T_2)$;
    \item $Z(\bigcup_{\lambda\in\Lambda} T_\lambda)=\bigcap_{\lambda\in\Lambda} Z(T_\lambda)$.
  \end{enumerate}
  于是$\PP^n$的代数集满足拓扑空间中闭集的公理, 如此可以在$\PP^n$定义以代数集全体为闭集全体的拓扑空间结构, 称为\emph{Zariski拓扑}.
\end{proposition}

\begin{proof}
  证明与\thref{prop:affinezariskitopology}的相当类似, 这里就不再重复了.
\end{proof}

利用\ref{sec:varietyaffine}关于拓扑空间的定义和性质, 称$\PP^n$的不可约闭子集为\emph{射影代数簇}, 射影代数簇的开子集被称为\emph{拟射影代数簇}.

与仿射情形相同, 为了后续讨论的方便, 我们需要陈述一个事实, 证明将放在下一部分中.

\begin{proposition}\label{prop:projectiveprimeirreducible}
  如果$\ideal{p}$是$S$的素理想, 那么$Z(\ideal{p})$是$\PP^n$的不可约闭集. 特别地, $\PP^n$是射影代数簇.
\end{proposition}

接下来我们将要构造$n$-射影空间中由$n$-仿射空间所构成的开覆盖, 这个构造与证明实射影空间是实微分流形中的构造\parencite[4--5, Definition 2.4]{flaherty_riemannian_1992}完全相同. 如果$f\in S$是一个齐次线性多项式, 则称$f$的零点集为一个\emph{超平面}. 特别地, 对于$j=0, \dotsc, n$, 记$H_j$为$X_j$的零点, 并记$U_j$为开集$\PP^n-H_j$. 定义映射$\varphi_j\colon U_j\to \AF^n$, 把有齐次坐标$(x_0, \dotsc, x_n)$的点$P\in U_j$映到有仿射坐标$(x_0{\divslash}x_j, \dotsc, x_{j-1}{\divslash}x_j, x_{j+1}{\divslash}x_j, \dotsc,  x_n{\divslash}x_j)$的点$Q\in \AF^n$. 因为$x_j\neq 0$, 并且$x_i{\divslash}x_j$的取值与齐次坐标的选取无关, 因此$\varphi_j$是良定的.

\begin{proposition}[{{\parencite[10, Proposition 2.2]{hartshorne_algebraic_1977}}}]\label{prop:projspaceopencoverhomeo}
  $\PP^n$可以被上述定义的开集族$U_j$开覆盖, 并且映射$\varphi_j$是$U_j$到$\AF^n$的同胚.
\end{proposition}

\begin{proof}
  不妨重新编号取$j=0$, 记$U\coloneq U_0$, $\varphi\coloneq \varphi_0$. $\varphi$显然是双射, 只需要证明闭集的像是闭集, 并且闭集的原像也是闭集.

  为了区别仿射空间与射影空间的多项式环, 我们在这里记$A\coloneq \kk [Y_1, \dotsc, Y_n]$. 记$S^h$为$S$的齐次多项式全体构成的集合, 我们将要定义从$S^h$到$A$的映射$\alpha$, 以及从$A$到$S^h$的映射$\beta$. 给定$f\in S^h$, 令$\alpha(f)=f(1, Y_1, \dotsc, Y_n)$; 另一方面, 给定次数为$d$的$g\in A$, 则$x_0^dg(X_1{\divslash}X_0, \dotsc, X_n{\divslash}X_0)$是一个$S$中次数为$d$的多项式, 它即是$\beta(g)$.

  现在令$Y\subseteq U$是$U$的闭子集. 记$\overline{Y}$为其在$\PP^n$中的闭包, 这是一个代数集, 因此存在子集$T\subseteq S^h$使得$\overline{Y}=Z(T)$. 令$T'=\alpha(T)$, 可以直接验证$\varphi(Y)=Z(T')$, 即闭集的像是闭集. 反过来的话, 令$W$为$\AF^n$的闭子集, 则存在$A$的子集$T''$使得$W=Z(T'')$, 很容易验证$\varphi^{-1}(W) = Z(\beta(T''))\cap U$, 即闭集的原像是闭集. 于是得证$\varphi$是一个同胚.
\end{proof}

\begin{proposition}[{{\parencite[11, Corollary 2.3]{hartshorne_algebraic_1977}}}]\label{prop:projvarietyopencoverhomeo}
  如果$Y$是一个(拟)射影代数簇, 则$Y$可以由开集族$Y\cap U_j$开覆盖, 并且$\varphi_j$诱导了从$Y\cap U_j$到(拟)仿射代数簇的同胚.
\end{proposition}

\begin{proof}
  只需要对$Y$是射影代数簇的情况, 证明$Y\cap U_j$是$U_j$中的不可约闭集. 注意到$Y\cap U_j$是不可约集$Y$的开子集, 故$Y\cap U_j$中的开集恰为$Y$中包含于$U_j$的开集\parencite[89, Lemma 16.2]{munkres_topology_2000}, 如果开集$V\subseteq Y\cap U_j$不稠密, 记$V$在$Y\cap U_j$中的闭包为$\overline{V}$, 那么$V$在$Y$中的闭包一定与$Y$的非空开集$Y\cap U_j-\overline{V}$不交\parencite[95, Theorem 17.5]{munkres_topology_2000}, 因此与$Y$是不可约集矛盾, 故$Y\cap U_j$不可约. 再由子拓扑的定义即可得到$Y\cap U_j$是$U_j$中的闭集.
\end{proof}

\begin{proposition}
  设$Y\subseteq \AF^n$为拟仿射代数簇, 则$Y$同胚于某个$\PP^n$中的拟射影代数簇.
\end{proposition}

\begin{proof}
  只需对$Y$是仿射代数簇的情况进行证明. 记$\beta$为\thref{prop:projspaceopencoverhomeo}的证明中定义的映射, $I(Y)\subseteq A$. 利用$U_0\cong\AF^n$将$Y$嵌入到$\PP^n$中, 则$Y$在$\PP^n$的闭包$\overline{Y}$恰为由$\beta (I(Y))$生成的齐次理想的零点集所定义的射影代数集, 而由$Y$是$U_0$的闭集可得$Y=U_0\cap \overline{Y}$, 即$Y$为$\overline{Y}$的不可约开子集, 因此$\overline{Y}$也是不可约的, 为射影代数簇, 故$Y$同胚于拟射影代数簇.
\end{proof}

\subsection{代数簇的态射}

之前我们定义了仿射代数簇和射影代数簇, 给出了其上的Zariski拓扑的结构. 事实上, 代数簇上面还有更丰富的结构, 我们将在这一节中给出.

一般来说, 如果给定了一个对象上的结构, 那么就会诱导出保持结构不变的映射, 在不同的语境下它们有不同的名字, 比如说映射, 线性映射, 环同态, 连续映射, 光滑映射等等. 在对这些具体情形的研究中, 我们发现很多时候这些保持结构不变的映射全体, 可以反过来决定一个对象的结构, 这就引出了范畴论的思想. 我们将分别给出仿射代数簇和射影代数簇上的函数结构, 从而定义代数簇之间的态射, 以此刻画代数簇的结构. 这之后我们将就可以定义簇的同构, 从而给仿射簇一个内蕴的定义.

先令$Y$为$\AF^n$中的拟仿射代数簇, 记$A=\kk [X_1, \dotsc, X_n]$, $A$的元素可以看成$Y$到$\AF^n$的函数. 考虑从$Y$到$\kk$的函数$f$, 如果$f$在局部可以写成两个多项式的商的形式, 即满足存在$\AF^n$的开集$P\in U\subseteq Y$与多项式$g, h\in A$, 使得$h$在$U$上没有零点, 并且在$U$上有等式$f=g{\divslash}h$成立, 则称$f$\emph{在点$P\in Y$正规}. 如果$f$在$Y$的每个点都正规, 则称$f$在$Y$上\emph{正规}.

\begin{proposition}[{{\parencite[15, Lemma 3.1]{hartshorne_algebraic_1977}}}]\label{prop:affineregularcontinuous}
  设$f\colon Y\to \kk$是从$\AF^n$的拟仿射代数簇到$\kk$的正规函数, 若给像集$\kk$赋予$\AF^1$的Zariski拓扑结构, 则$f$是连续函数.
\end{proposition}

\begin{proof}
  只需要证明闭集的原像是闭集. 因为$\AF^1$的非空真闭子集恰为有限子集, 因此只需要证明每个单点集$\{x\}$的原像$f^{-1}(x)=\{P\in Y\vert f(P)=x\}$是闭集即可. 利用\thref{prop:closedsetlocal}, 我们只需要证明在局部上是闭集就可以了. 考虑使得$f$能表示成两个多项式$g, h$的商的开集$U$, 则$f^{-1}(x)\cap U=\{P\in U\vert g(P)/h(P)=x\}$, 故$f^{-1}(x)\cap U=Z(g-xh)\cap U$为$U$中的闭集, 故得证.
\end{proof}

再令$Y$为$\PP^n$中的拟射影代数簇, 记$S=\kk [X_0, \dotsc, X_n]$, $S$中相同次数的齐次多项式的商可以看成$Y$到$\AF^n$的函数. 考虑从$Y$到$\kk$的函数$f$, 如果$f$在局部可以写成两个相同次数的齐次多项式的商的形式, 即满足存在$\PP^n$的开集$P\in U\subseteq Y$与次数相同的齐次多项式$g, h\in S$, 使得$h$在$U$上没有零点, 并且在$U$上有等式$f=g{\divslash}h$成立, 则称$f$\emph{在点$P\in Y$正规}. 如果$f$在$Y$的每个点都正规, 则称$f$在$Y$上\emph{正规}.

\begin{proposition}[{{\parencite[15, Remark 3.1.1]{hartshorne_algebraic_1977}}}]\label{prop:projregularcontinuous}
  设$f\colon Y\to \kk$是从$\PP^n$的拟射影代数簇到$\kk$的正规函数, 若给像集$\kk$赋予$\AF^1$的Zariski拓扑结构, 则$f$是连续函数.
\end{proposition}

\begin{proof}
  与\thref{prop:affineregularcontinuous}的证明雷同.
\end{proof}

将之前定义的仿射代数簇, 拟仿射代数簇, 射影代数簇, 拟射影代数簇统称为\emph{代数簇}, 我们已经定义了代数簇上的正规函数. 类似于单复变函数中非零全纯函数的零点是孤立的\parencite[127]{ahlfors_complex_1978}, 代数簇上的函数也有很强的刚性.

\begin{proposition}[{{\parencite[15, Remark 3.1.1]{hartshorne_algebraic_1977}}}]\label{prop:regularstiffness}
  设$f, g$为代数簇$X$上的正规函数, 如果在某个非空开集$U$上有$f=g$, 那么在整个$X$上都有$f=g$.
\end{proposition}

\begin{proof}
  只需注意$Z(f-g)\supseteq \overline{U}=X$即可.
\end{proof}

现在我们可以来定义代数簇之间的态射了. 设$\kk$为固定的代数闭域, $X, Y$为两个代数簇, $X$到$Y$的\emph{态射}即为连续映射$\varphi\colon X\to Y$, 并且满足对于每个开子集$V\subseteq Y$和每个正规函数$f\colon V\to \kk$都有复合函数$f\composite\varphi\colon\varphi^{-1}(V)\to \kk$是正规函数. 这定义了$k$上的代数簇所构成的范畴, 是一个具体范畴.

有了代数簇同构的定义, 我们可以重新考察之前定义过的代数簇之间的Zariski拓扑结构的同胚, 研究它们是不是代数簇结构的同构, 答案是肯定的.

\begin{proposition}[{{\parencite[18, Proposition 3.3]{hartshorne_algebraic_1977}}}]\label{prop:projectiveopencoverisomorphism}
  设$U_j\in \PP^n$为由$X_j\neq 0$定义的开集, 则\thref{prop:projspaceopencoverhomeo}中定义的同胚$\varphi_j\colon U_j\to \AF^n$是代数簇的同构. 更一般地, 如果$Y$是$\PP^n$中的(拟)射影代数簇, 则\thref{prop:projvarietyopencoverhomeo}中$\varphi_j$诱导的从$Y\cap U_j$到(拟)仿射代数簇的同胚也是代数簇的同构.
\end{proposition}

\begin{proof}
  对于第一个命题, 利用\thref{prop:projspaceopencoverhomeo}的证明中所定义的$\alpha$和$\beta$, 我们可以建立$U_j$和$\AF^n$由$\varphi_j$诱导的在每个局部的正规函数的对应, 因此它们作为代数簇是同构的. 第二个命题同理.
\end{proof}

下面一个命题阐述了任意代数簇到仿射代数簇的映射是态射当且仅当每个分量是正规函数, 我们将在\thref{thm:categoryisoaffinevariety}的构造中用到这一刻画.

\begin{proposition}[{{\parencite[20, Lemma 3.6]{hartshorne_algebraic_1977}}}]\label{prop:morphismtoaffinevariety}
  设$X$为任意代数簇, $Y\subseteq \AF^n$为仿射代数簇. 集合间的映射$\varphi\colon X\to Y$为态射当且仅当对于每个$Y$的坐标函数$X_j\subseteq \AF^n$都有$X_j\composite\varphi$是$X$上的正规函数.
\end{proposition}

\begin{proof}
  如果$\varphi$是态射, 那么根据定义$X_j\composite\varphi$一定是正规函数. 另一方面, 如果每个$X_j\composite\varphi$都是正规函数, 那么对于每个多项式$f\in A(Y)$都有$f\composite \varphi$是正规函数. 考察
  \begin{tikzcd}[cramped, sep=small]
    X \arrow[r, "\varphi"] & Y \arrow[r, "f"] & \kk
  \end{tikzcd}
  , 因为$Y$的闭集都可以写成若干(可能无限)个多项式零点集的交, 又正规函数是连续的并且$\{0\}$为$\kk\cong \AF^1$中的闭集, 因此$Y$的闭集在$\varphi$下的原像是闭集, 故$\varphi$为连续映射. 最后, 因为正规函数被定义为在局部是多项式的商, 所以对于每个$Y$的开子集上的正规函数$g$都有$g\composite\varphi$是两个正规函数的商, 还是正规函数, 于是$\varphi$是态射.
\end{proof}

在代数簇上我们可以定义一些函数环. 设$Y$是一个代数簇, 记$Y$的正规函数全体构成的环为$\OO (Y)$, 称为$Y$的\emph{整体函数环}.

如果$P$是$Y$的一点, 考虑由包含$P$的开集$U$与$U$上的正规函数$f\colon U\to \kk$的二元组$\lrbig<>{U, f}$全体构成的集合, 在上面定义等价关系$\sim$, 其中$\lrbig<>{U, f}$等价于$\lrbig<>{V, g}$当且仅当$f$和$g$在非空开集$U\cap V$的取值相同. 等价关系$\sim$下的商集被定义为$Y$在$P$点的\emph{局部环}, 其实是$P$附近的函数芽全体构成的环, 记作$\OO_{Y, P}$, 在不引起混淆的情况下也记为$\OO_P$. $\OO_P$是局部环, 它的唯一极大理想$\ideal{m}$为在$P$点取值为零的正规函数所在的函数芽, 剩余域$\OO_P{\divslash}\ideal{m}$同构于$\kk$.% TODO: direct limit

考虑由非空开集$U$与$U$上的正规函数$f\colon U\to \kk$的二元组$\lrbig<>{U, f}$全体构成的集合, 在上面定义等价关系$\sim$, 其中$\lrbig<>{U, f}$等价于$\lrbig<>{V, g}$当且仅当$f$和$g$在非空开集$U\cap V$的取值相同. 等价关系$\sim$下的商集被定义为$Y$的\emph{函数域}, 记作$K(Y)$, 称$K(Y)$的元素为\emph{有理函数}.因为每个非零正规函数$f\colon U\to \kk$都在某个非空开子集$U-U\cap Z(f)$上没有零点, 所以$K(Y)$是域.

这三个环之间其实有包含关系, 不难发现存在单同态$\OO (Y)\to \OO_P\to K(Y)$, 因此我们经常会把$\OO (Y)$和$\OO_P$看作$K(Y)$的子环. 这三个环在代数簇的同构下保持不变, 因此是同构不变量.

\subsection{有理映射}

\thref{prop:regularstiffness}我们介绍了正规函数的刚性, 事实上代数簇之间的态射也具有的刚性, 由此我们可以引出有理映射以及双有理同构的概念, 双有理同构下的不变量也是代数几何的一个非常基础的课题. 为了证明态射的唯一性, 我们需要一些准备.

设$r, s$为正整数, $N=(r+1)(s+1)-1$. 定义映射$\psi\colon \PP^r\times\PP^s\to\PP^N$, 将$((x_0, \dotsc, x_r), (y_0, \dotsc, y_s))$按照字典序映到$(\dotsc, x_jy_k, \dotsc)$, 则$\psi$为良定单射, 称为\emph{Segre嵌入}.

\begin{proposition}[{{\parencite[13, Exercise 2.13]{hartshorne_algebraic_1977}}}]\label{prop:segreembedding}
  上述定义的Segre嵌入的像是$\PP^N$中的射影代数簇.
\end{proposition}

\begin{proof}
  设$\PP^N$中的齐次坐标对应的未定元为$\{Z_{jk}\vert j=0, \dotsc, r, k=0, \dotsc, s\}$, 考虑环同态$\kk [\{Z_{jk}\}]\to \kk [X_0, \dotsc, X_r, Y_0, \dotsc, Y_s]$将$Z_{jk}$映到$X_jY_k$, 这是个到整环的满同态, 设其核为素理想$\ideal{p}$. 不难证明$\Im (\psi)=Z(\ideal{p})$, 于是根据\thref{prop:projectiveprimeirreducible}可得像集为不可约闭集.
\end{proof}

利用\thref{prop:projectiveprimeirreducible}, 我们可以给$\PP^r\times\PP^s$赋予射影代数簇的结构. 要注意的是, $\PP^r\times\PP^s$作为射影代数簇的拓扑与乘积拓扑不同. 现在想要来定义两个拟射影代数簇的乘积.

\begin{proposition}[{{\parencite[22, Exercise 3.16]{hartshorne_algebraic_1977}}}]\label{prop:projectiveproduct}
  设$X\subseteq\PP^r$与$Y\subseteq\PP^s$为两个拟射影代数簇, 考虑$X\times Y\subseteq \PP^r\times\PP^s$. 则有
  \begin{enumerate}
    \item $X\times Y$是拟射影代数簇;
    \item 如果$X, Y$都是射影代数簇, 那么$X\times Y$也是;
    \item\label{enum:prop-projective-product-3} $X\times Y$与典范投影映射是$X$和$Y$在代数簇范畴中的乘积.
  \end{enumerate}
\end{proposition}

现在我们可以来证明态射的唯一性了.

\begin{proposition}
  设$X, Y$为代数簇, $\varphi, \psi$为从$X$到$Y$的两个态射. 如果存在$\varphi$和$\psi$在某个非空开集上取值相同, 则$\varphi=\psi$.
\end{proposition}

\begin{proof}
  因为每个代数簇都同构于某个$\PP^n$中的拟射影代数簇, 再复合上到$\PP^n$的包含映射, 我们可以不妨假设$Y=\PP^n$. 考虑射影代数簇$\PP^n\times \PP^n$, 根据\thref{prop:projectiveproduct}\ref{enum:prop-projective-product-3}, 存在乘积态射$\varphi\times \psi$. 记$\Delta=\{(P, P)\vert P\in \PP^n\}$为$\PP^n\times \PP^n$的对角线子集, 由方程$\{X_jY_k-X_kY_j=0\vert i, j=0, \dotsc, n\}$定义, 因此是$\PP^n\times \PP^n$的闭子集. 根据假设以及乘积态射的定义, 有$\varphi\times \psi(U)\subseteq \Delta$, 又根据\thref{prop:continuousimageclosure}, 得$\varphi\times \psi(X)\subseteq\Delta$, 于是根据乘积态射的定义可知$\varphi=\psi$.
\end{proof}

现在来定义代数簇之间的有理映射. 设$X, Y$为代数簇, 考虑由非空开集$U\subseteq X$和态射$\varphi_U\colon U\to Y$组成的二元组$\lrbig<>{U, \varphi_U}$全体构成的集合, 在上面定义等价关系$\sim$, 其中$\lrbig<>{U, \varphi_U}$等价于$\lrbig<>{U, \varphi_U}$当且仅当$\varphi_U, \varphi_V$在非空开集$U\cap V$的取值相同. 于是一个从$X$到$Y$的\emph{有理映射}$\varphi$被定义为等价关系$\sim$下的一个等价类. 如果有理映射$\varphi$满足对于每一对$\lrbig<>{U, \varphi_U}$都有$\varphi_U$的像都在$Y$中稠密, 则称$\varphi$为\emph{控制有理映射}. $\varphi$为控制有理映射等价于对于某一对$\lrbig<>{U, \varphi_U}$成立$\varphi_U$的像都在$Y$中稠密\parencite{andrew_how_2016}. 代数簇关于有理映射也构成一个范畴, 称这个范畴中的同构为\emph{双有理同构}.
