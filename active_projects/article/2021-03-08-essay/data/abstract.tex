% !TeX root = ../main.tex

% 中英文摘要和关键字

\begin{abstract}{代数几何, 交换代数, 准素分解, 维数理论}
  代数几何历史悠久而又充满活力, 与其它数学分支有着深刻的联系, 是一门重要的学科. 代数几何的理论根基在于代数. 本文从基本定义开始建立了以环论为主模论为辅的交换代数理论, 研究了商环与分式环的基本性质, 证明了Noether环上准素分解的存在性及其满足的唯一性, 以域论为基础利用Noether正规化引理证明了域的有限生成整环上的维数定理和Hilbert零点定理.

  代数几何的研究对象在于几何. 本文研究了固定代数闭域上的仿射与射影空间中的代数集, 建立了根式理想与代数集之间的对应, 从而将准素分解理论与几何相联系. 本文还讨论了代数簇上的函数结构, 以此将维数理论应用到几何中, 并证明了两组代数范畴与几何范畴的等价. 最后本文简要介绍了概形的概念, 其相比于代数簇能更完整地体现代数所提供的信息.
\end{abstract}

\begin{abstract*}{algebraic geometry, commutative algebra, primary decomposition, dimension theory}
  This dissertation provides a general introduction to algebraic geometry, which is one of the oldest and the most active subjects in mathematics with deep connections to almost all other branches.

  Algebraic geometry lays its foundation on algebra. The author establishes commutative algebra mainly through commutative ring theory, with a little help from module theory, where quotient rings and fraction rings have been studied, existence and uniqueness of primary decomposition in Noetherian rings have been proved, and dimension theory of finitely generated domains, followed by Hilbert's Nullstellensatz, has been carried out by means of Noetherian normalization lemma.

  Algebraic geometry marks geometry as its main subject. The author researches on algebraic sets in affine and projective spaces over a fixed algebraically closed field, and establishes the Galois connection between algebraic sets and radical ideals, showing relationship between primary decomposition and irrudicible decomposition in geometry. Then the author calculates the function structures on algebraic varieties, in either affine space or projective space, where dimension theory applies, which leads to two different category equivalences between algebraic object and geometrical ones. Also, the definition and first example of schemes have been included, which can reflect, comparing to varieties, more information from algebra.
\end{abstract*}
