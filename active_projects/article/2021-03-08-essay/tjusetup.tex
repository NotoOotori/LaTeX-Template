% !TeX root = ./main.tex

% 论文基本信息配置

\tjusetup{
  %******************************
  % 注意:
  %   1. 配置里面不要出现空行
  %   2. 不需要的配置信息可以删除
  %   3. 建议先阅读文档中所有关于选项的说明
  %******************************
  %
  % 输出格式
  %   选择打印版(print)或用于提交的电子版(electronic),前者会插入空白页以便直接双面打印
  %
  % output = print,
  %
  % 标题
  %   不可使用“\\”命令手动控制换行
  %   副标题为可选项
  %
  title  = {代数理论的几何应用},
  % subtitle = {使用示例文档 v\version},
  title* = {Geometric Application of Algebraic Theory},
  %
  % 学位
  %   1. 学术型
  %      - 中文
  %        需注明所属的学科门类,例如:
  %        哲学、经济学、法学、教育学、文学、历史学、理学、工学、农学、医学、
  %        军事学、管理学、艺术学
  %      - 英文
  %        博士:Doctor of Philosophy
  %        硕士:
  %          哲学、文学、历史学、法学、教育学、艺术学门类,公共管理学科
  %          填写“Master of Arts“,其它填写“Master of Science”
  %   2. 专业型
  %      直接填写专业学位的名称,例如:
  %      教育博士、工程硕士等
  %      Doctor of Education, Master of Engineering
  %   3. 本科生不需要填写
  %
  % degree-name  = {工学硕士},
  % degree-name* = {Master of Science},
  %
  % 培养单位
  %   填写所属院系的全名
  %
  department = {数学科学学院},
  %
  % 学科
  %   1. 学术型学位
  %      获得一级学科授权的学科填写一级学科名称,其他填写二级学科名称
  %   2. 工程硕士
  %      工程领域名称
  %   3. 其他专业型学位
  %      不填写此项
  %   4. 本科生填写专业名称,第二学位论文需标注“(第二学位)”
  %
  discipline  = {数学与应用数学},
  % discipline* = {Computer Science and Technology},
  %
  % 姓名
  %
  author  = {陈旭阳},
  % author* = {Xue Ruini},
  %
  % 学号
  %
  id = {1753763},
  %
  % 指导教师
  %   中文姓名和职称之间以英文逗号“,”分开,下同
  %
  supervisor  = {李灵光, 副教授},
  % supervisor* = {Professor Zheng Weimin},
  %
  % 副指导教师
  %
  % associate-supervisor  = {陈文光, 教授},
  % associate-supervisor* = {Professor Chen Wenguang},
  %
  % 联合指导教师
  %
  % joint-supervisor  = {某某某, 教授},
  % joint-supervisor* = {Professor Mou Moumou},
  %
  % 日期
  %   使用 ISO 格式;默认为当前时间
  %
  % date = {2019-07-07},
  %
  % 是否在中文封面后的空白页生成书脊(默认 false)
  %
  % include-spine = false,
  %
  % 生成的声明页是否要插入页眉和页脚(默认 empty)
  % 仅在需要进行电子签名时,才需要打开这一选项
  % 插入的扫描声明页总是会生成页眉(研究生)和页脚,不受这一选项影响
  %
  % statement-page-style = plain,
  %
  % 密级和年限
  %   秘密, 机密, 绝密
  %
  % secret-level = {秘密},
  % secret-year  = {10},
  %
  % 博士后专有部分
  %
  % clc                = {分类号},
  % udc                = {UDC},
  % id                 = {编号},
  % discipline-level-1 = {计算机科学与技术},  % 流动站(一级学科)名称
  % discipline-level-2 = {系统结构},          % 专业(二级学科)名称
  % start-date         = {2011-07-01},        % 研究工作起始时间
}

%% Put any packages you would like to use here

% 表格中支持跨行
% \usepackage{multirow}

% 跨页表格
% \usepackage{longtable}

% 固定宽度的表格。放在 hyperref 之前的话,tabularx 里的 footnote 显示不出来。
% \usepackage{tabularx}

% 表格加脚注
% \usepackage{threeparttable}
% \pretocmd{\TPTnoteSettings}{\footnotesize}{}{}

% 确定浮动对象的位置,可以使用 H,强制将浮动对象放到这里(可能效果很差)
% \usepackage{float}

% 浮动图形控制宏包。
% 允许上一个 section 的浮动图形出现在下一个 section 的开始部分
% 该宏包提供处理浮动对象的 \FloatBarrier 命令,使所有未处
% 理的浮动图形立即被处理。这三个宏包仅供参考,未必使用:
% \usepackage[below]{placeins}
% \usepackage{floatflt} % 图文混排用宏包
% \usepackage{rotating} % 图形和表格的控制旋转

% 定理类环境宏包
% \usepackage{amsthm}
% 也可以使用 ntheorem
\usepackage[amsmath,thmmarks,hyperref,thref]{ntheorem}


% 给自定义的宏后面自动加空白
% \usepackage{xspace}

% 定义所有的图片文件在 figures 子目录下
% \graphicspath{{figures/}}

% 定义自己常用的东西
% \def\myname{薛瑞尼}

% 数学命令
\DeclareMathOperator{\Aut}{Aut}
\DeclareMathOperator{\Ann}{Ann}
\let\dim\relax
\DeclareMathOperator{\dim}{dim}
\DeclareMathOperator{\codim}{codim}
%https://tex.stackexchange.com/questions/175251/how-to-redefine-a-command-using-declaremathoperator
\newcommand{\lr}[3]{\left#1#3\right#2}
\newcommand{\ideal}[1]{\symfrak{#1}}
% \newcommand{\rad}[1]{\sqrt{\ideal{#1}}}
\newcommand{\axiom}[1]{\ensuremath{\symsfup{#1}}}
\newcommand{\field}[1]{#1}
\newcommand{\powerset}[1]{2^{#1}}

\newcommand{\CC}{\symbb{C}}
\newcommand{\ZZ}{\symbb{Z}}
\newcommand{\NN}{\symbb{N}}
\newcommand{\RR}{\symbb{R}}
\newcommand{\AF}{\symbb{A}}
\newcommand{\PP}{\symbb{P}}
\newcommand{\kk}{\field{k}}
\newcommand{\mapsfrom}{\mathrel{\reflectbox{\ensuremath{\mapsto}}}}
% \newcommand{\niton}{\mathrel{\reflectbox{\ensuremath{\notin}}}}
\newcommand{\depend}{\prec}
\newcommand{\ndepend}{\nprec}
\newcommand{\SAa}{\overline{S}^{-1}(A{\divslash}\ideal{a})}
\newcommand{\SASa}{S^{-1}A{\divslash}S^{-1}\ideal{a}}

% \newcommand\dif{\mathop{}\!\mathrm{d}}  % 微分符号
% \newcommand\real{{\mathbf{R}}}  % 实数集
% \newcommand\abs[1]{\lvert#1\rvert}
% \newcommand\VECTOR{\symbf}  % 向量
% \newcommand\MATRIX{\symbf}  % 矩阵
% \newcommand\vn{{\VECTOR{n}}}
% \newcommand\vx{{\VECTOR{x}}}
% \newcommand\mA{{\MATRIX{A}}}
% \newcommand\mK{{\MATRIX{K}}}

% 借用 ltxdoc 里面的几个命令方便写文档。
% \DeclareRobustCommand\cs[1]{\texttt{\char`\\#1}}
% \providecommand\pkg[1]{{\sffamily#1}}

% 借用 lipsum与zhlipsum 来填充内容
\usepackage{lipsum}
\usepackage{zhlipsum}

% 交叉引用
\usepackage{varioref}
\labelformat{section}{\S#1}
\labelformat{subsection}{\S#1}

% hyperref 宏包在最后调用
\usepackage{hyperref}

\hypersetup{linktoc=all}% TOC中章节名称和页码都是超链接
% \hypersetup{% 超链接样式
%   colorlinks  =   true,
%   linkcolor   =   cyan,
%   anchorcolor =   black,
%   citecolor   =   green,
%   filecolor   =   cyan,
%   menucolor   =   red,
%   runcolor    =   filecolor,
%   urlcolor    =   magenta,
% }
\hypersetup{% 书签
  bookmarksopen = true,
  bookmarksopenlevel = 2,
  bookmarksnumbered = true,% 将章节编号放入书签中
}
\hypersetup{% PDF默认打开视图
  pdfstartview        =   FitH,
  pdfpagelayout       =   OneColumn,
}
\hypersetup{% PDF信息
  pdftitle={\tjutitle},
  pdfauthor={\tjuauthor},
}

% One should always load \usepackage{hyperref} before the first use of \newtheorem to obtain correct handling and referencing of counters.
\theoremstyle{plain}
\theoremheaderfont{\bfseries}\theorembodyfont{\upshape}
\theoremindent0pt
% \theoremprework{}
\newtheorem{theorem}{定理}[subsection]
\newtheorem{corollary}[theorem]{推论}
\newtheorem{lemma}[theorem]{引理}
\newtheorem{proposition}[theorem]{命题}
% \newtheorem{axiom}[theorem]{公理}
% \newtheorem{conjection}[theorem]{猜想}
\theoremstyle{nonumberplain}
\theoremheaderfont{\bfseries}\theorembodyfont{\upshape}
\theoremindent0pt
\newtheorem{definition}{定义}
\newtheorem{remark}{注}
\newtheorem{example}{例}
% \theoremsymbol{\ensuremath{\enclosesquare}}
\theoremsymbol{\ensuremath{\mdlgwhtsquare}}
\newtheorem{proof}{证明}
\theoremsymbol{}
% \newtheorem{proofsketch}{证明梗概}
% \newenvironment{breakexample}{\begin{example}\leavevmode\vspace{-\baselineskip}}{\end{example}}

% \makeatletter
% \newcommand{\ntheoremheader}[1]{\expandafter\csname mkheader@#1\endcsname}
% \makeatother

\appto\appendix{%
  \theoremstyle{plain}%
  \theoremheaderfont{\bfseries}%
  \theorembodyfont{\upshape}%
  \theoremindent0pt%
  \renewtheorem{theorem}{定理}[section]%
}% etoolbox
\makeatletter
\newif\if@appendix
\appto\appendix{\@appendixtrue}
\appto\backmatter{\@appendixfalse}
\preto\section{\if@appendix\else\clearpage\fi}
\makeatother
