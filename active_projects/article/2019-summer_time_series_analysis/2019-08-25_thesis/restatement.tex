某商业银行的ATM应用系统包括前端和后端两个部分. 前端是部署在银行营业部和各自助服务点的ATM机(系统), 后端是总行数据中心的处理系统. 前端的主要功能是和客户直接交互, 采集客户请求信息, 然后通过网络传输到后端, 再进行数据和账务处理. 持卡人从前端设备提交查询或转账或取现等业务请求, 到后台处理完毕, 并将处理结果返回到前端, 通知持卡人业务处理最终状态, 本文称这样完整的一个流程为一笔交易.

商业银行总行数据中心监控系统为了实时掌握全行的业务状态, 每分钟对各分行的交易信息进行汇总统计. 汇总信息包括业务量, 交易成功率, 交易响应时间三个指标, 各指标解释如下:

\begin{enumerate}
    \item 业务量: 每分钟总共发生的交易总笔数.
    \item 交易成功率: 每分钟交易成功笔数和业务量的比率.
    \item 交易响应时间: 一分钟内每笔交易在后端处理的平均耗时(单位: 毫秒).
\end{enumerate}

交易数据分布存在以下特征: 工作日和非工作日的交易量存在差别; 一天内, 交易量也存在业务低谷时间段和正常业务时间段. 当无交易发生时, 交易成功率和交易响应时间指标为空.

商业银行总行数据中心监控系统通过对每家分行的汇总统计信息做数据分析, 来捕捉整个前端和后端整体应用系统运行情况以及时发现异常或故障. 常见的故障场景包括但不限于如下情形:

\begin{enumerate}
    \item 分行侧网络传输节点故障, 前端交易无法上送请求, 导致业务量陡降.
    \item 分行侧参数数据变更或者配置错误, 数据中心后端处理失败率增加, 影响交易成功率指标.
    \item 数据中心后端处理系统异常(如操作系统CPU负荷过大)引起交易处理缓慢, 影响交易响应时间指标.
    \item 数据中心后端处理系统应用进程异常, 导致交易失败或响应缓慢.
\end{enumerate}

根据附件中某商业银行ATM应用系统某分行的交易统计数据, 本文需要
\begin{itemize}
    \item 选择, 提取和分析ATM交易状态的特征参数.
    \item 设计一套交易状态异常检测方案, 在对该交易系统的应用可用性异常情况下能做到及时报警, 同时尽量减少虚警误报.
    \item 设想可增加采集的数据. 基于扩展数据, 提升前两个任务能达到的目标.
\end{itemize}
