% -------------------- Packages --------------------

\documentclass{assignment}[2019/10/15]
\usepackage[lineno]{packages}[2019/11/14]

% -------------------- Settings --------------------

% Title

\title{Homework of Chapter 2}
\author{Chen Xuyang}
\date{\today}
\institute{School of Mathematical Science}
\professor{Shan Haiying}
\course{Combinatorics}
\subject{Combinatorics}
\keywords{}

% -------------------- New commands --------------------

\newcommand{\BR}{\symbb{R}}
\newcommand{\BZ}{\symbb{Z}}
\newcommand{\diag}{\mathop{}\!\symup{diag}}
\newcommand{\pr}{\mathop{}\!\symup{Pr}}
\newcommand{\expect}{\mathop{}\!\symup{E}}
\newcommand{\cov}{\mathop{}\!\symup{Cov}}
\newcommand{\var}{\mathop{}\!\symup{Var}}

\def\multiset#1#2{\ensuremath{\left(\kern-.3em\left(\genfrac{}{}{0pt}{}{#1}{#2}\right)\kern-.3em\right)}}

\newcommand{\lr}[3]{\left#1#3\right#2}
\newcommand{\lmr}[5]{\left#1#4\middle#2#5\right#3}

% -------------------- Document --------------------

\begin{document}
    \maketitle
    \begin{problem}\label{pr:1}
       Calculate the number of solutions of the linear Diophantine equation
        \begin{equation}
            x_1+x_2+\dotsb +x_8=40,\quad x_i\geq i\ (i=1, 2, \dotsc, 8).
        \end{equation}
    \end{problem}
    \begin{solution}
        It suffices to calculate the number of nonnegative solutions of the linear Diophantine equation
        \begin{equation}
            y_1 + y_2 + \dotsb + y_8 = 4,
        \end{equation}
        by letting $y_i = x_i - i$, where $i=1, 2, \dotsc, 8$. Thus the total number of solutions is
        \begin{equation}
            \binom{8+4-1}{4} = \binom{11}{4} = 330
        \end{equation}
        by the combinatorics of multisets.
    \end{solution}

    \begin{problem}\label{pr:2}
        Let $A$ denote a set of $n$ distinct integers and let $P = (P_1, P_2)$ be a nontrivial ordered partition of $A$, such that the minimum number of $P_1$ is greater than the maxium number of $P_2$, that is to say
        \begin{equation}
            \min P_1 > \max P_2.
        \end{equation}
        Calculate the number of all possible cases of such partition $P$.
    \end{problem}
    \begin{note}
    {\bf
        The problem is stated wrong here and the answer of the original problem in the textbook should be
        \begin{equation}
            \sum_{j=1}^{n-1}2^{j-1}(2^{n-j}-1) = (n-2)2^{n-1}+1,
        \end{equation}
        by first fixing the smallest element of the larger subset and then sum through the cases of all possible smallest element of the larger subset. Need to notice that both subsets should be nonempty.
    }
    \end{note}
    \begin{solution}
        {\bf This is the solution of the misstated \ref{pr:2}.}
        Suppose $m = \max P_2$. Then for each $a\in A$ such that $a > m$, since $m = \max P_2$, there is $a \notin P_2$, which implies that $a \in P_1$. Since $\min P_1 > \max P_2 = m$, there is $a\in P_2$ for each $a\in A$ such that $a < m$. Therefore $m$ can induce a bijection
        \begin{equation}
            \phi_m : a\in A-\max A \to \lr(){\{x\in A|\ x > a\}, \{x\in A|\ x\leq a\}}\in P,
        \end{equation}
        where it should be noticed that $P_1, P_2$ need to be nonempty sets. It follows that the all possible cases of such a partition $P$ equals to the order of $A$ minus 1, which is $n-1$.
    \end{solution}

    \begin{problem}
        Suppose $A$ is a convex 10-gen such that every three diagnoals do not intersect. Calculate the total number of intersection points of all diagonals, and the total number of open line segment separated by those diagonals.
    \end{problem}
    \begin{solution}
        To count the number of intersection points of diagnoals, it suffices to count the number of convex subpolygens of $A$, that is
        \begin{equation}
            \binom{10}{4} = 210.
        \end{equation}
        These points separate diagonals into
        \begin{equation}
            210\times 2 + 35 = 455
        \end{equation}
        open line segments, where 35 denotes the number of diagnoals.
    \end{solution}

    \begin{problem}
        Let $I_{1000}$ denote the set of positive integers not greater than 1000, and let $A$ denote a subset of 3 elements of $I_{1000}$ such that the sum of the elements of $A$ is a multiplier of 4.
        Calculate the number of all possible cases of such a subset $A$.
    \end{problem}
    \begin{solution}
        Since we only take into account the reminder of 4, it suffices to consider the quotient multiset
        \begin{equation}
            E = \lmr./.{I_{1000}}\sim = \lmr\{\vert\}{k_a\cdot a}{\ k_a = \lr||{[a]_\sim}=250, a\in I_{4}}
        \end{equation}
        of $I$, where $a\sim b$ implies that $a$ is congruence to $b$ mod 4.
        Since $0<I + I + I + I\leq 12$, all possible cases are listed below
        \begin{equation}
            \begin{aligned}
                &1+1+2=4, &&1+3+4=8, &&2+2+4=8,\\
                &2+3+3=8, &&4+4+4=12. &&\\
            \end{aligned}
        \end{equation}
        Hence the answer should be
        \begin{equation}
            \binom{250}{2}\times 250
            + 250^3
            + \binom{250}{2}\times 250
            + \binom{250}{2}\times 250
            + \binom{250}{3}
            = 41541750
        \end{equation}
        by elementry combinatorics.
    \end{solution}

    \begin{problem}
        Zheng Bowen wants to send 5 distinct letters through communication channel, where there should be at least 3 spaces between every two letters. Suppose there are 15 spaces in total to be placed between the letters. Please calculate the number of all possible cases of placing the spaces.
    \end{problem}
    \begin{solution}
        Similar to \ref{pr:1}, the answer is
        \begin{equation}
            \multiset{4}{3} = 20,
        \end{equation}
        by transforming the problem to the calculationi of all solutions of a nonnegative linear Diophantine equation.
    \end{solution}

    \begin{problem}
        A naughty boy named Zheng Bowen wants to place the letters `a', `b', `c', `d', `e', `f', `g' and `h' in a row, such that `a' is on the left hand side of `b' and `b' is on the left hand side of `c'. Calculate the number all possible permutations satisfies these conditions.
    \end{problem}
    \begin{solution}\hspace{\fill}
        \begin{itemize}
            \item \textbf{Method 1} First he places the first three letters `a', `b' and `c' in order, like ` a b c ', with some spaces surrounding each letter. Then he inserts the other 5 letters into these spaces, where it is allowed to leave some spaces empty. Hence the answer is
            \begin{equation}
                4\times 5\times 6\times 7\times 8=6720.
            \end{equation}
            \item \textbf{Method 2} Now Zheng Bowen has become more clever than before. He first places all the letters no matter the rules. Then he changes the position of the three letters `a', `b' and `c' to meet the rules. He finds out the needed number is the order of a quotient group of the symmetric group $S_8$. Hence by the counting theorem, he gets the number
            \begin{equation}
                \frac{8!}{3!}=6720,
            \end{equation}
            which is also true.
        \end{itemize}
    \end{solution}
\end{document}
