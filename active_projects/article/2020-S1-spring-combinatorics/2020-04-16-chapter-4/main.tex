% -------------------- Packages --------------------

\documentclass[chinese]{assignment}[2019/10/15]
\usepackage[lineno]{packages}[2019/11/14]

% -------------------- Settings --------------------

% Title

\title{Homework of Chapter 4}
\author{Chen Xuyang}
\date{\today}
\institute{School of Mathematical Science}
\professor{Shan Haiying}
\course{Combinatorics}
\subject{Combinatorics}
\keywords{}

% -------------------- New commands --------------------

\newcommand{\BR}{\symbb{R}}
\newcommand{\BZ}{\symbb{Z}}
\newcommand{\diag}{\mathop{}\!\symup{diag}}
\newcommand{\pr}{\mathop{}\!\symup{Pr}}
\newcommand{\expect}{\mathop{}\!\symup{E}}
\newcommand{\cov}{\mathop{}\!\symup{Cov}}
\newcommand{\var}{\mathop{}\!\symup{Var}}

\def\multiset#1#2{\ensuremath{\left(\kern-.3em\left(\genfrac{}{}{0pt}{}{#1}{#2}\right)\kern-.3em\right)}}

\newcommand{\lr}[3]{\left#1#3\right#2}
\newcommand{\lmr}[5]{\left#1#4\middle#2#5\right#3}

% -------------------- Document --------------------

\begin{document}
    \maketitle
    \begin{problem}
        求多重集合$S=\{\infty\cdot a, 3\cdot b, 5\cdot c, 7\cdot d\}$的10组合数.
    \end{problem}
    \begin{solution}
        容斥原理, 设$X= \{\infty\cdot a, \infty\cdot b, \infty\cdot c, \infty\cdot d\}$. 令$A_2$表示$X$中满足$b$的个数多于3的10组合全体, 令$A_3$表示$X$中满足$c$的个数多于5的10组合全体, 令$A_4$表示$X$中满足$d$的个数多于7的10组合全体. 有
        \begin{equation}
            \begin{aligned}
                |X| &= \binom {13}3 = 286,\\
                |A_2| &= \binom 93 = 84,\\
                |A_3| &= \binom 73 = 35,\\
                |A_4| &= \binom 53 = 10,\\
                |A_2 \cap A_3| &= \binom 33 = 1,\\
                |A_2 \cap A_4| &= 0,\\
                |A_3 \cap A_4| &= 0,\\
                |A_2 \cap A_3 \cap A_4| &= 0.
            \end{aligned}
        \end{equation}
        因此$S$的10组合数为$286-84-35-10+1=158$.
    \end{solution}
    \begin{problem}
        求不定方程$x_1+x_2+x_3=14$的数值不超过8的正整数解的个数.
    \end{problem}
    \begin{solution}
        考虑多重集合$S = \{8\cdot x_1, 8\cdot x_2, 8\cdot x_3\}$的11组合数. 设$A_i$表示$x_i$不少于8个, 其余元素无限制的11组合全体. 类似于上题, 有答案为$\binom {13}2 - 3\binom 52 = 48$.
    \end{solution}
    \begin{problem}
        求多重集合$S=\{3\cdot a, 4\cdot b, 2\cdot c\}$的全排列数, 使得在这些排列中同一个字母的全体不能相邻(例如, 不允许$abbbbcaac$, 但允许$baabbacbc$).
    \end{problem}
    \begin{solution}
        记$X$为$S$的排列全体, 记$X_a$表示$X$中所有的$a$都相邻的排列全体, 记$X_b$表示$X$中所有的$b$都相邻的排列全体, 记$X_c$表示$X$中所有的$c$都相邻的排列全体. 由容斥原理, 得所求全排列数为$A_0-A_1+A_2-A_3=871$, 其中
        \begin{equation}
            \begin{aligned}
                A_0 &= \frac{(3+4+2)!}{3!4!2!}=1260,\\
                A_1 &= \frac{(1+4+2)!}{1!4!2!}+\frac{(3+1+2)!}{3!1!2!}+\frac{(3+4+1)!}{3!4!1!}=445,\\
                A_2 &= \frac{(1+1+2)!}{1!1!2!}+\frac{(3+1+1)!}{3!1!1!}+\frac{(1+4+1)!}{1!4!1!}=62,\\
                A_3 &= \frac{(1+1+1)!}{1!1!1!}=6
            \end{aligned}
        \end{equation}
        分别表示至少有$i$种字母都相邻时的情况数.
    \end{solution}
    \begin{problem}
        计算机系有三个运动队, 每人一套运动服. 现计算机系有足球服38套, 篮球服15套, 排球服20套. 三个运动队共有58人, 其中只有3人同时是三个队的队员, 问恰好参加两个队的人数以及至少参加两个队的人数.
    \end{problem}
    \begin{solution}
        设$X$表示计算机系中三个运动队里的所有的人构成的集合, $A_1$表示$X$中参加足球队的人构成的集合, $A_2$表示$X$中参加篮球队的人构成的集合, $A_3$表示$X$中参加排球队的人构成的集合. $Y_1$表示恰好参加两个队的人数, $Y_2$表示至少参加两个队的人数, 有公式
        \begin{equation}
            \begin{aligned}
                0 &= |A_1\cup A_2 \cup A_3| - \left(|A_1| + |A_2| + |A_3|\right)\\
                &\phantom{{}={}} + \left(|A_1\cap A_2| + |A_1\cap A_3| + |A_2\cap A_3|\right) - |A_1\cap A_2\cap A_3|,\\
                Y_1 &= |A_1\cap A_2| + |A_1 \cap A_3| + |A_2 \cap A_3| - 3|A_1\cap A_2\cap A_3|,\\
                Y_2 &= Y_1 + |A_1\cap A_2\cap A_3|.
            \end{aligned}
        \end{equation}
        有条件
        \begin{equation}
            \begin{aligned}
                |A_1| &= 38,\\
                |A_2| &= 15,\\
                |A_3| &= 20,\\
                |A_1\cap A_2\cap A_3| &= 3.\\
            \end{aligned}
        \end{equation}
        计算得$Y_1=9, Y_2 = 12$.
    \end{solution}
    \begin{problem}
        8个小孩围坐在木马上, 问有多少种变换座位的方法, 使得每个小孩前面坐的都不是原来的小孩?
    \end{problem}
    \begin{solution}
        记置换群$S_8$为全集, $\tau = (12345678)\in S_8$为cycle, $j=\{\sigma \in S_8\vert\ \tau\sigma(j)=\sigma\tau(j)\}$表示使得"第$j$个小孩前面坐的是原来的小孩"的变换座位的方法所构成的集合($j\in I_8$), 则所求方法数即为
        \begin{equation}
            \lr\vert\vert{\bigcap_{j=1}^8 {A_j}^c}.
        \end{equation}
        设$X\subseteq I_8$, $|X|=k$为$k$元指标集($0\leq k\leq 8$), 考虑$\lr\vert\vert{\bigcap_{x\in X}A_x}$. 注意到$X$导出了$I_8$上的等价关系$R_X$, 其中$xR_Xy$当且仅当
        \begin{equation}
            (x\in X \wedge y = \tau(x)) \vee (y\in X \wedge x = \tau(y)).
        \end{equation}
        称$I_8$在$R_X$关系中的一个等价类为一个连续团, 则$X$将$I_8$分为了$\max\{8-k, 1\}$个连续团. 我们先任意固定一个连续团的位置, 共有8种可能, 确定剩余连续团的位置就转化为了线排列问题, 共有$(8-k-1)!$种可能(记$(-1)!=1$), 因此
        \begin{equation}
            \lr\vert\vert{\bigcap_{x\in X}A_x} = 8(8-k-1)!,\quad 1\leq k\leq 8.
        \end{equation}
        因此
        \begin{equation}
            \lr\vert\vert{\bigcap_{j=1}^8 {A_j}^c} = \sum_{X\subseteq I_8}\lr(){(-1)^{|X|}\lr\vert\vert{\bigcap_{x\in X}A_x}} = 13000,
        \end{equation}
        其中指标集为空集的集合族之交记为全集.
    \end{solution}
    \begin{problem}
        利用容斥原理证明
        \begin{equation}
            \sum_{k=0}^m(-1)^k\binom mk \binom {n-k}r = \binom {n-m}{r-m}\quad(n\geq r\geq m\geq 0).
        \end{equation}
    \end{problem}
    \begin{proof}
        考虑这样一个组合模型, 我们把$n+1$个相同的小球和放入$r+1$个不同的盒子中, 每个盒子中都至少有一个小球; 盒子按顺序编号, 其中前$m$个盒子中放入且仅放入一个小球. 这样的放法共有$\binom {n-m}{r-m}$种. 再利用容斥原理, 考虑$A_i$为第$i$个盒子中放入至少两个小球的放法全体, 其中$i\in I_m$, 则若固定$X\subseteq I_m$, 有
        \begin{equation}
            \lr\vert\vert{\bigcap_{i\in X}A_i} = \binom {n-|X|}r
        \end{equation}
        只与$X$的基数有关, 因此由容斥原理, 有
                \begin{equation}
            \sum_{k=0}^m(-1)^k\binom mk \binom {n-k}r = \binom {n-m}{r-m}\quad(n\geq r\geq m\geq 0).
        \end{equation}
        得证.
    \end{proof}
\end{document}
