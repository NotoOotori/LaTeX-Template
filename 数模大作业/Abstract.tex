% !Mode::"TeX:UTF-8"
\begin{abstract}
    本文主要建立了Kakuro求解算法模型,
    Kakuro难度评估模型, 蜂巢数独求解模型,
    并且还给出了生成不同难度和具有唯一解的Kakuro的方法.
    
    在求解Kakuro数独和求解蜂巢数独的模型中,
    本文使用深度优先搜索的算法, 
    采用了回溯的思想,
    并采用人们所常用的几个数独技巧作为剪枝方法, 其中包括了可行性剪枝, 最优化剪枝和归谬法剪枝,
    实现了两种不同数独的快速求解.
    在蜂巢数独求解模型中, 本文将一整行当作一个状态, 从而只用较少的递归次数就能实现数独的求解.
    在建立难度评估系统模型中, 本文基于深度优先搜索树的性质,
    搜索树上根节点到目标节点的唯一路径的分支总数来刻画数独的难度, 得出了难度评估公式,
    并给出了分级标准, 实现了对Kakuro数独不同难度的划分.
    对于Kakuro的生成的方法, 本文使用演绎法, 先确定数独中黑格子的位置,
    再一步步地添加约束条件, 在保证数独的唯一解的原则下, 由局部到整体地生成数独.
    
    本文的模型能够在极短的时间内对两种数独进行求解, 在所有搜集到的两种数独之中,
    Kakuro都能在1s以内计算出结果,
    蜂巢数独能在0.1s以内完成求解.
    本文建立的难度评估系统, 在实例分析中, 能够对不同难度的Kakuro数独实现较为准确的分级,
    且具有稳定性, 可以不受搜索的随机性的影响.
    本文建立的数独生成模型能够生成不同难度具有唯一解的数独, 并且难度可以在生成之前指定.
\end{abstract}
