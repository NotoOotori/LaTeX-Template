% !Mode::"TeX:UTF-8"

% -------------------- Information --------------------

\newcommand{\TITLE}{小组作业4}
\newcommand{\AUTHOR}{Team 29}
\newcommand{\SUBJECT}{经典数学专题选讲}

% -------------------- Packages --------------------

\documentclass[a4paper,12pt]{ctexart}
\usepackage{amsmath}
%\usepackage{amsthm} %定理格式 由ntheorem代替
\usepackage{amssymb}
\usepackage[thmmarks, amsmath, thref]{ntheorem} % 定理格式
\usepackage{DefaultTheoremStyle}
\usepackage{lastpage}
\usepackage{makecell} % 表格线加粗 \Xhline{1.2pt}
\usepackage{boldline} % 长表格表格线加粗
\usepackage{multirow} % 合并单元格
\usepackage{array}
\usepackage{longtable} %长表格
\usepackage[dvipsnames]{xcolor} % 颜色声明
\usepackage{varioref} % For Cross References
\usepackage{caption} %题注
\usepackage{fancyhdr} %脚注
\usepackage[square, numbers, sort&compress]{natbib} %引用
\usepackage{hyperref}
\usepackage{graphicx}
\usepackage[section]{placeins} % 使图像不会显示在别的部分 若过于严格则换成[below]
%\usepackage{fontspec} % 字体
\usepackage{titlesec} %Section标题格式
\usepackage{SUBSubsubsection}
\usepackage{authblk} %作者
\usepackage{stackrel} %上下写
\usepackage{mathtools} %\xleftrightarrow
%\usepackage{enumitem} 用enumerate包代替
\usepackage{listings} %排版程序语言
\usepackage{enumerate}
\usepackage{flafter} %不让float出现在定义之前的地方
\usepackage{float} %你们这帮float给我乖乖听话 HHHHHHHHHHH
\usepackage{pgfplots}
\usepackage{bigfoot} % to allow verbatim in footnote
\usepackage[numbered, framed]{matlab-prettifier}
\usepackage{filecontents}
\usepackage[all,cmtip]{xy} % Commutive diagram.
\usepackage{lineno} % Line numbers.

% -------------------- Settings --------------------

% Title
    \title{\TITLE}
    \author{\AUTHOR}
    \date{\today}

% Package: caption
    \captionsetup{
        margin    =   6pt,
        font      =   small,
        labelfont =   bf
    }

% Package: fancyhdr
    \setlength{\headheight}{15pt}
    \lhead{Copyright \copyright\ \AUTHOR}
    \rhead{Page \thepage\ of \pageref{LastPage}}

% Package: graphicx
    \graphicspath{{figures/}} %图像文件目录
    
% Package: hyperref
    \hypersetup{
        linktoc             =   all,
        colorlinks          =   true,
        linkcolor           =   TealBlue,
       %anchorcolor         =   Black,
        citecolor           =   Blue,
       %filecolor           =   Cyan,
       %menucolor           =   Red,
       %runcolor            =   filecolor,
        urlcolor            =   magenta,
        pdfinfo             =   {
            Title           =   {\TITLE},
            Author          =   {\AUTHOR},
            Subject         =   {\SUBJECT}},
        bookmarksnumbered   =   true,
        pdfstartview        =   FitH,
        pdfpagelayout       =   OneColumn
    }
            
% Package: pgfplot
    \pgfplotsset{width=7cm}

% Package: varioref
    \renewcommand{\reftextbefore}
        {on the \reftextvario{preceding page}{page before}}
    \renewcommand{\reftextafter}
        {on the \reftextvario{following}{next} page}
    \renewcommand{\reftextfacebefore}
        {on the \reftextvario{facing}{preceding} page}
    \renewcommand{\reftextfaceafter}
        {on the \reftextvario{facing}{next}{page}}
    \renewcommand{\reftextfaraway}[1]
        {on page \pageref{#1}}

% -------------------- General new commands --------------------

\newcommand{\diff}{\mathop{}\!\mathrm{d}}
\newcommand{\e}{\mathrm{e}}

% -------------------- Specific new commands --------------------

\newcommand{\rank}{\mathrm{rank}}

% -------------------- Document --------------------

\begin{document}

    % -------------------- Title Page --------------------

    \maketitle
    %\thispagestyle{empty}
    %\pagenumbering{roman}
    %\newpage

    % -------------------- Abstract Page --------------------

    % -------------------- Contents --------------------

    % -------------------- Body --------------------

    \pagestyle{fancy}
    \pagenumbering{arabic}
    \linenumbers
    
    \begin{problem}
        设$A$是$n$阶方阵, 证明: $\rank(A^2)-\rank(A^3)\leq\rank(A)-\rank(A^2)$.
    \end{problem}

    \begin{lemma}
        设$A,B,C$为$n$阶方阵, $O$为$n$阶零矩阵, 则
        \begin{equation}
            \label{lemma_rank}
            \rank
            \begin{pmatrix}
                A & B\\
                O & C
            \end{pmatrix}
            \geq \rank(A)+\rank(C),
        \end{equation}
        并且当$B=O$时等号成立.
    \end{lemma}

    \begin{theorem}
        设$A$是$n$阶方阵, 则$\rank(A^2)-\rank(A^3)\leq\rank(A)-\rank(A^2)$.
    \end{theorem}

    \begin{proof}
        根据分块矩阵的初等变换, 有
        \begin{equation}
            \begin{pmatrix}
                A^2 & A\\
                O & A^2
            \end{pmatrix}
            \leftrightarrow
            \begin{pmatrix}
                A^2 & A\\
                -A^3 & O
            \end{pmatrix}
            \leftrightarrow
            \begin{pmatrix}
                O & A\\
                -A^3 & O
            \end{pmatrix}.
        \end{equation}
        因此
        \begin{equation}
            \begin{aligned}
                \rank(A^2) + \rank(A^2)
                &\leq\rank
                \begin{pmatrix}
                    A^2 & A\\
                    O & A^2
                \end{pmatrix}\\
                &=\rank
                \begin{pmatrix}
                    O & A\\
                    -A^3 & O
                \end{pmatrix}\\
                &=\rank(A) + \rank(-A^3).
            \end{aligned}
        \end{equation}
        即
        \begin{equation}
            \rank(A^2)-\rank(A^3)\leq\rank(A)-\rank(A^2).
        \end{equation}
    \end{proof}


    \begin{problem}
        设$\displaystyle x_n=\sum_{j=0}^{\infty}{\frac{j^n}{j!}}
        \,(n\geq 0),$
        其中$0^0, 0!=1$. 证明$x_n$是$\e$的正整数倍.
    \end{problem}

    \begin{proof}
        首先, 发现对于任意的$n\geq 0$, 级数$x_n$均收敛.

        我们用数学归纳法来证明.

        当$n=0$时, 
        \begin{equation}
            x_0=\sum_{j=0}^{\infty}{\frac{1}{j!}}=e.
        \end{equation}

        设$n\leq k$时$x_n$都是$\e$的正整数倍, 并记倍数为$e_n$, 当$n=k+1$时, 有
        \begin{equation}
            \begin{aligned}
                x_{k+1}
                &= \sum_{j=0}^{\infty}{\frac{j^{k+1}}{j!}}\\
                &= \sum_{j=1}^{\infty}{\frac{j^{k+1}}{j!}}\\
                &= \sum_{j=1}^{\infty}{\frac{j^{k}}{(j-1)!}}\\
                &= \sum_{j=0}^{\infty}{\frac{(j+1)^{k}}{j!}}\\
                &= \sum_{j=0}^{\infty}
                    {\frac{\sum_{\ell=0}^{k}{a_{\ell}j^{\ell}}}{j!}},
            \end{aligned}
        \end{equation}
        其中$a^{\ell}$为正整数. 因为$x_{n}$收敛, 故两个求和号可交换, 即
        \begin{equation}
            x_{k+1}=\e\sum_{\ell=0}^{k}{e_{\ell}a_{\ell}},
        \end{equation}
        是$\e$的正整数倍.
    \end{proof}

    % -------------------- Bibliography --------------------

\end{document}
