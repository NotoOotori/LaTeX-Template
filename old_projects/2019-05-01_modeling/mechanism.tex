\subsection{模型假设}

与前题相同, 假设广告视频长度固定, 并且额外假设当频道和时段固定时,
广告播放对买方带来的收益分布与广告与频道的动态匹配度呈正相关,
而与被广告物本身的价值无关.

\subsection{模型建立}

在本题中需要建立竞价交易模型,
极大化卖方收益的同时提升收视率和买方产品销售量.

本文按如下方式建立经济模型: 固定频道特征和广告播放时段,
设共有$n$个买方为潜在的参与竞价者,
他们可以有放弃竞价的权利.
买方$i$的经济特征$e_{i}$即为买方广告的动态匹配度$a_{i}$和买方的报价$b_{i}$.
记社会选择函数$f$为从经济环境$E$到可行配置结果空间$Y$的映射,
其中可行配置结果空间$Y$包含赢得竞价的买家和其应付的价格$p_{i}$.
应满足动态匹配度$a_{i}$和报价$v_{i}$都尽可能高.
本文寻找的经济机制需要使得对所有的经济环境$e\in E$都有其配置结果符合社会选择.

经济机制在竞价之前公开给所有买方, 它由信息空间$M$和配置规则函数$h$所组成,
其中信息空间内的元素$m$包含买方的报价以及买方广告的动态匹配度,
而配置规则函数$h$根据信息$m$决定赢得竞价的买方.

在本题中, 对于配置结果$y\in Y$, 如果买方$i$赢得竞价, 即$i\in y$,
那么买方所获得的效益为他的真实估价$v_{i}$减去付出的价格$p_{i}(y)$,
其中真实估价$v_{i}$与配置结果$y$无关;
如果买方$i$未赢得竞价, 则效益为0, 即买方$i$的价值函数可写为
\begin{equation}
    u_{i}(y) = \chi_{y}(i)(v_{i}-p_{i}(y)).
\end{equation}
设卖方对于把广告播放时段卖给动态匹配度为$a_{i}$的买家$i$的估价为$F_{i}=F(a_{i})$,
记$\Delta F_{i}=F_{i} - F$, 其中$F$为平均估价,
那么卖方的价值函数可写为
\begin{equation}
    u(y) = \sum_{i=1}^{n}{\chi_{y}(i)(p_{i}(y)-F_{i})}.
\end{equation}

直观上本题中设立的经济机制需要能够诱导买方报出他们的真实估价$v_{i}$,
如此一来根据假设, 动态匹配度高的买方愿意出较高的价格赢得竞价.
由于买方的报价接近于他们对广告播放时段的估价, 并不会恶意拉低报价,
所以卖方的收益得以保证; 与此同时由于动态匹配度高的广告得到播放,
那么电视观众较愿意观看广告也较愿意购买产品, 于是收视率不会被拉低,
买方产品销售量也能提升.

于是文献\citep{Tian}中提到的格罗夫斯-克拉克-威克瑞(Groves-Clark-Vickrey)
需求显示机制能很好地解决本题中涉及到的问题.
本文推广了原始的格罗夫斯机制, 引入了动态匹配度对于卖方估价的影响,
从而引入了动态匹配度对买方报价的修正.
以下我们先假定有且仅有一个买家能赢得广告播放时段.

首先格罗夫斯机制要求每个买方$i$报出他的报价$b_{i}$. 每个人的报价有真有假,
因此$b_{i}$不一定等于$v_{i}$. 然后根据报价与卖方估价的差$b_{i}-F_{i}$是否为最大值
决定买方是否赢得竞价, 即格罗夫斯机制规定赢得竞价的买方由下式决定
\begin{equation}
    y(b) = i, \quad \text{if}\ b_{i}-F_{i} = \max_{j}{(b_{j}-F_{j})}.
\end{equation}

每个买方$i$的转移支付(transfer payment)记为$t_{i}$,
若$t_{i}<0$则被解释为附加税, 若$t_{i}>0$则被解释为补偿.
$t_{i}$体现了买方$i$的报价对于其余买方收益的影响, 由下式决定
\begin{equation}
    t_{i}(b) =
    \begin{cases}
        \displaystyle\sum_{j\neq i}b_{j} + d_{i}(b_{-i}),
        &\text{if}\ b_{i}-F_{i} = \displaystyle\max_{j}{(b_{j}-F_{j})},\\
        d_{i}(b_{-i}), &\text{else}.
    \end{cases}
\end{equation}
这里$d_{i}$是可任意给定的函数. 于是买家$i$的支付函数为
\begin{equation}
    p_{i}(y(b))=
    \begin{cases}
        v_{i}+\displaystyle\sum_{j\neq i}b_{j} + d_{i}(b_{-i}),
        &\text{if}\ b_{i}-F_{i} = \displaystyle\max_{j}{(b_{j}-F_{j})},\\
        d_{i}(b_{-i}), &\text{else}.
    \end{cases}
\end{equation}

可以证明每个人真实显示他的估价$v_{i}$是优势均衡策略, 即每个人都有激励说真话.
由于$d_{i}$为任意给定函数, 所以我们不妨令
\begin{equation}
    d_{i}(b_{-i}) = -\max_{y}{\sum_{j\neq i}{b_{i}(y)-\Delta F_{i}(y)}},
\end{equation}
则转移支付函数变为
\begin{equation}
    t_{i}(b)=\sum_{j\neq i}{b_{j}} - \max_{y}{\sum_{j\neq i}{b_{i}(y)
    -\Delta F_{i}(y)}}.
\end{equation}
在本竞价机制中, 如果$i$赢得竞价, 则$\sum_{j\neq i}{b_{j}}=0$, 于是
$t_{i}(v)=-\max_{j\neq i}{b_{j}}$. 如果$i$未赢得竞价,
则$t_{i}(v)=0$. 所以$b_{i}-\Delta F_{i}$最高的买方赢得竞价,
而最终出价为第二高未修正报价.

上文的讨论是基于只有一个买家能赢得广告播放时段进行的,
事实上可以直接推广到有限个买家能赢得广告播放时段的情形.
设共有$K$个赢家, 先将多赢家竞价问题分解为$K$个单赢家竞价问题:
每个买方在每次竞价中的出价独立, 每次竞价取修正出价最高的买方赢得广告播放时段,
实际出价为第二高价, 赢得广告播放时段的买家不参与下一次竞价.
而事实上之前已经证明了在单赢家竞价模型中买方报出真实估价为优势均衡策略,
因此可以假设每个买方在各次竞价中出价相同, 所以多赢家竞价机制即为:
修正报价前$K$高的买方赢得广告播放时段, 实际出价为排名比他低一位的未修正报价.

至此本文得到了基于动态匹配度的推广格罗夫斯竞价交易机制,
它能较好地解决电视台广告竞价问题.
