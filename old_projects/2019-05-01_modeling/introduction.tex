\subsection{问题重述}
电视广告是地方电视台商业营运的主要业务之一,
其中一个重要概念是广告播放时段.
广告播放时段表示某个广告在该时段中在某一个频道得到播放的权利,
该权利由电视台(下称卖方)向广告视频制作者(下称买方)分周期竞卖,
并且当前周期组织完成下一个周期的竞价交易.
通过竞卖, 卖方能得到卖出广告播放时段的直接收益,
但是可能因为播放广告影响收视率, 而造成潜在的损失.
买方通过在适当的频道播放广告, 提升销售量的期望, 获得潜在收益.
所以我们需要
\begin{enumerate}
    \item 通过选择提取视频广告和电视频道用户的分类特征, 建立衡量二者
        匹配度的模型, 从而完成静态推送.
    \item 对于该分时段竞卖, 通过估计广告播放时段的价值和买方愿意付的价格
        设计卖方合理底价(reserve price)的估算模型.
    \item 基于电视频道用户的收视历史情况和在播视频广告的产品销售情况,
        建立带有更新函数的视频广告和电视频道用户的匹配度模型.
    \item 基于上述匹配度模型, 平衡买家的报价和对收视率的影响,
        以极大化卖方收益为目的建立竞价交易模型, 同时兼顾提升收视率和卖方产品销售量.
    \item 最后设计并建立模型的求解算法, 通过实际或编撰的数据给出算例.
\end{enumerate}

\subsection{问题分析}

该问题主要分为两部分, 第一部分是广告与频道的匹配度模型,
第二部分是基于匹配度模型的竞价模型.

关于分类匹配推送静态模型, 本文将按以下步骤入手:
\begin{enumerate}
    \item 了解电视等媒体向目标观众群体推送广告的机制,
        搜集视频广告和电视频道用户的分类特征相关的信息, 由此初步确定广告的特征参数以及
        电视频道用户的画像.
    \item 通过主成分分析实现广告分类特征的去噪和降维,
        并使用快速聚类算法($K$-means聚类分析)实现广告分类.
    \item 计算步骤二分类后各广告聚类质心的特征向量与电视频道用户画像均值
        的特征向量之间的欧氏距离, 确定其关联度, 依据关联度向目标用户群匹配推送
        相应的广告簇.
\end{enumerate}

关于竞价模型, 谭国富从对拍卖机制, 买方卖方的收益, 影响拍卖结果的因素等方面
对拍卖理论作了综述\citep{Tan},
其中拍卖指的是卖方给出拍卖物并事先制定好规则, 之后由多位买方自由竞价,
根据规则得出最后赢得物品的买方和应付的价格, 通常情况下为出价最高的买方赢得物品.
在本文讨论的问题中, 卖方组织的竞价实际上就是拍卖行为, 但是与传统的拍卖不同的是
买方不仅提供报价, 还提供广告视频给电视台播放.
同时由于卖方播放广告会对其期望收益造成影响,
所以卖方不只关心买方给出价格, 还会考虑买方广告与频道的匹配度.
因此本文将会给买方制定基于匹配模型的广告价值修正模型,
给实际报价一个修正, 以修正后的报价从高到低决定赢得广告播放时段的买家.
