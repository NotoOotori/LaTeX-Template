% -------------------- Packages --------------------

\documentclass{assignment}[2019/10/15]
\usepackage[lineno]{packages}[2019/11/14]

% -------------------- Settings --------------------

% Title

\title{Homework of Chapter 3}
\author{Chen Xuyang}
\date{\today}
\institute{School of Mathematical Science}
\professor{Shan Haiying}
\course{Combinatorics}
\subject{Combinatorics}
\keywords{}

% -------------------- New commands --------------------

\newcommand{\BR}{\symbb{R}}
\newcommand{\BZ}{\symbb{Z}}
\newcommand{\diag}{\mathop{}\!\symup{diag}}
\newcommand{\pr}{\mathop{}\!\symup{Pr}}
\newcommand{\expect}{\mathop{}\!\symup{E}}
\newcommand{\cov}{\mathop{}\!\symup{Cov}}
\newcommand{\var}{\mathop{}\!\symup{Var}}

\def\multiset#1#2{\ensuremath{\left(\kern-.3em\left(\genfrac{}{}{0pt}{}{#1}{#2}\right)\kern-.3em\right)}}

\newcommand{\lr}[3]{\left#1#3\right#2}
\newcommand{\lmr}[5]{\left#1#4\middle#2#5\right#3}

% -------------------- Document --------------------

\begin{document}
    \maketitle
    \begin{problem}
        Prove the identities
        \begin{equation}
            \sum_{k=0}^n(k+1)^2\binom{n}{k}=2^{n-2}\cdot(n^2+5n+4)
        \end{equation}
        and
        \begin{equation}
            \sum_{k=0}^n\frac{1}{k+2}\binom{n}{k}=\frac{n\cdot 2^{n+1}+1}{(n+1)(n+2)}
        \end{equation}
    \end{problem}
    \begin{proof}
        By the principle of counting twice, we can easily get
        \begin{equation}
            \begin{aligned}
                \sum_{k=0}^n(k+1)^2\binom{n}{k}
                &=\sum_{k=0}^n k^2\binom{n}{k}
                +2\sum_{k=0}^n k\binom{n}{k}
                +\sum_{k=0}^n\binom{n}{k}\\
                &=n(n+1)2^{n-2}+2n\cdot 2^{n-1}+2^n\\
                &=(n^2+5n+4)2^{n-2},
            \end{aligned}
        \end{equation}
        while
        \begin{equation}
            \begin{aligned}
                \sum_{k=0}^n\frac{1}{k+2}\binom{n}{k}
                &= \sum_{k=0}^n\frac{k+1}{(n+1)(n+2)}\binom{n+2}{k+2}\\
                &= \frac{1}{(n+1)(n+2)}\sum_{k=0}^n(k+1)\binom{n+2}{k+2}\\
                &= \frac{1}{(n+1)(n+2)}\sum_{k=2}^{n+2}(k-1)\binom{n+2}{k}\\
                &= \frac{1}{(n+1)(n+2)}\lr(){(n+2)2^{n+1}-(n+2)-2^{n+2}+(n+2)+1}\\
                &= \frac{n\cdot 2^{n+1}+1}{(n+1)(n+2)}
            \end{aligned}
        \end{equation}
        by the relationship between combinatorial numbers. It ends the proof.
    \end{proof}
    \clearpage
    \begin{problem}
        Prove that
        \begin{itemize}
            \item there is
                \begin{equation}
                    \binom n1-2\binom n2 + 3\binom n3+\dotsb+(-1)^{n-1}n\binom nn=0
                \end{equation}
                where $n$ denotes an integer not less than 2, and that
            \item there is
                \begin{equation}
                    1+\frac{1}{2}\binom n1 + \frac{1}{3}\binom n2 + \frac{1}{4}\binom n3 + \dotsb + \frac{1}{n+1}\binom nn=\frac{1}{n+1}(2^{n+1}-1)
                \end{equation}
                where $n$ denotes an positive integer.
        \end{itemize}
    \end{problem}
    \begin{proof}
        Let $x=-1$ in the identity
        \begin{equation}
            \sum_{k=0}^n k\binom nk x^{k-1}=n(1+x)^{n-1}.
        \end{equation}
        Then we have
        \begin{equation}
            \sum_{k=0}^n (-1)^{k-1}k\binom nk =0,
        \end{equation}
        as desired. And since
        \begin{equation}
            \frac{1}{k+1}\binom nk = \frac{1}{n+1}\binom{n+1}{k+1},
        \end{equation}
        we have
        \begin{equation}
            \sum_{k=0}^n \frac{1}{k+1}\binom{n}{k}=\frac{1}{n+1}\binom{n+1}{k+1}=\frac{2^{n+1}-1}{n+1}.
        \end{equation}
        It ends the proof.
    \end{proof}
    \begin{problem}
        Find integers $a$, $b$ and $c$ such that
        \begin{equation}
            m^3 = a\binom{m}{3} + b\binom m2 + c\binom m1.
        \end{equation}
        Then calculate the value of
        \begin{equation}
            \sum_{m=1}^{n}m^3
        \end{equation}
        for arbitrary positive integer $n$.
    \end{problem}
    \begin{solution}
        It is easy to conclude that $a=6$, $b=6$ and $c=1$. Thus we have
        \begin{equation}
            \begin{aligned}
                \sum_{m=1}^{n}m^3
                &= 6\sum_{m=1}^n \binom{m}{3} + 6\sum_{m=1}^n \binom{m}{2} + \sum_{m=1}^n \binom{m}{1}\\
                &= 6\binom{n+1}{4} + 6\binom{n+1}{3} + \binom{n+1}{2}\\
                &= \frac{1}{4}n^2(n+1)^2,
            \end{aligned}
        \end{equation}
        as desired.
    \end{solution}
    \begin{problem}
        Prove that
        \begin{itemize}
            \item there are $(3n+1)/2$ strings generated by the character set $\{0, 1, 2\}$ with length $n$ where each string contains an even number of character 0, and that
            \item there is
                \begin{equation}
                    \binom n0 2^n + \binom n2 2^{n-2}+\dotsb + \binom nq 2^{n-q}=\frac{3^n+1}{2},
                \end{equation}
                where $q=2\lfloor n/2\rfloor$.
        \end{itemize}
    \end{problem}
    \begin{proof}
        These two propositions are equivalent since we can determine the string by first fixing the 0's and then placing the other characters. Hence it suffices to prove the identity, which is easy to show by taking summation at both sides of the following identities
        \begin{equation}
            \sum_{k=0}^n \binom nk 2^{n-k} = 3^n,
        \end{equation}
        and
        \begin{equation}
            \sum_{k=0}^n \binom nk (-2)^{-k} = (-1)^n\sum_{k=0}^n \binom nk (-2)^{n-k} = (-1)^{2n}=1,
        \end{equation}
        and then dividing both side of the answer by 2.
    \end{proof}
    \begin{problem}
        Calculate the value of
        \begin{equation}
            \sum_{\substack{r,s,t\geq 0\\ r+s+t=n}}\binom{m_1}{r} \binom {m_2}s \binom {m_3}t.
        \end{equation}
    \end{problem}
    \begin{solution}
        By reordering the summation, we have
        \begin{equation}
            \begin{aligned}
                \sum_{n=0}^{m_1+m_2+m_3}\binom{m_1+m_2+m_3}{n}x^n
                &= (1+x)^{m_1+m_2+m_3}\\
                &= \lr(){\sum_{r=0}^{m_1}\binom{m_1}{r}x^r}\lr(){\sum_{s=0}^{m_2}\binom{m_2}{s}x^s}\lr(){\sum_{t=0}^{m_3}\binom{m_3}{t}x^t}\\
                &=\sum_{n=0}^{m_1+m_2+m_3}\sum_{\substack{r,s,t\geq 0\\ r+s+t=n}}\binom{m_1}{r} \binom {m_2}s \binom {m_3}t x^n.
            \end{aligned}
        \end{equation}
        By checking the coefficient of each $x^n$, we have
        \begin{equation}
            \sum_{\substack{r,s,t\geq 0\\ r+s+t=n}}\binom{m_1}{r} \binom {m_2}s \binom {m_3}t  = \binom{m_1+m_2+m_3}{n},
        \end{equation}
        as desired.
    \end{solution}
    \begin{problem}
        Prove the following identity by means of Combinatorics
        \begin{equation}
            n(n+1)2^{n-2}=\sum_{k=1}^nk^2\binom nk,
        \end{equation}
        where $n$ and $k$ are positive integers.
    \end{problem}
    \begin{solution}
        Let $S$ denote the set of 2-tuples
        \begin{equation}
            \lmr\{|\}{(A, P)}{\ A\subseteq I_n \wedge \lr(){\text{$P$ is 2-permutation of the multiset $\{\infty\cdot a|\ a\in A\}$}}}.
        \end{equation}
        Then we assert that
        \begin{equation}
            |S| = n(n+1)2^{n-2} \quad\text{and}\quad |S| = \sum_{k=1}^nk^2\binom nk.
        \end{equation}
        If $A$ is decided first with $k$ elements, then there are $\binom nk$ distinct $A$s and there are $k^2$ distinct $P$s. Since $k$ is arbitrary,  by the rule of addition and the rule of product we have
        \begin{equation}
            |S| = \sum_{k=1}^nk^2\binom nk.
        \end{equation}
        On the other hand, consider the cases that $P$ is determined first. If the two elements in $P$ coincide, then $A$ is free to choose the other $n-1$ elements. Thus there are $n\cdot 2^{n-1}$ different tuples in this case. If the two elements in $P$ are distinct, then $A$ is free to choose the other $n-2$ elements. Thus there are $n(n-1) 2^{n-2}$ different tuples in this case. And all these tuples are distinct. Thus by the rule of addition,
        \begin{equation}
            |S| = n\cdot 2^{n-1} + n(n-1) 2^{n-2} = n(n+1)2^{n-2}.
        \end{equation}
        It ends the proof.
    \end{solution}
\end{document}
