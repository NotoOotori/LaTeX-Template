\begin{abstract}
    本文通过某商业银行总行数据中心监控系统提供的交易信息数据的特征提取与分析, 建立了处理交易过程中故障的鉴别模型, 以此提高ATM系统运行的效率.

    第一部分本文分析了交易量数据, 先着重关注单日交易总量这一交易参数, 发现单日交易数在1月27号前后明显异常, 在此之前交易量陡增, 在这天之后有陡降. 通过分析查日历得知这一天是2017农历除夕, 所以重点节假日对ATM的交易量有显著影响, 可以帮助商业银行在重点节假日工作安排提供参考. 另外, 通过一日内的交易量的实时分析, 本文发现每日上下午各有一次交易高峰, 本文利用EM算法模拟双峰高斯函数, 较为精确的得到各具体参数值, 通过图像与语言文字的宏观分析, 并结合参数的定量细致研究, 从而建立了良好的交易状态特征参数体系, 并且为后续的异常方案检测提供了好的方向.

    第二部分本文分析了成功率数据. 通过对成功率随交易量变化的散点图, 找到了二者之间一定的相关关系, 但是并不足以用来衡量是否出现异常与故障. 由独立同分布的假设, 本文采用了概率论与统计学的方法, 对数据进行标准化后, 根据随机变量观测值的数字特征很容易地找到了过度偏移的异常点. 通过对相关的交易成功率, 交易响应时间的考察, 发现这些交易点确实为系统可能出现了故障的点, 从而证明了本文采用的利用数据数字特征进行检测的方法的合理性.

    第三部分本文分析了响应时间. 通过对响应时间随交易量变化的散点图, 发现在交易量较大时绝大部分的交易响应时间都在600毫秒以内, 因此以600毫秒为阈值进行讨论. 同时在交易量较小时响应时间波动较大, 并且通过观察估计交易响应时间与交易量近似符合反比例函数关系. 于是通过设定阈值对异常点与非异常点进行判断, 结合交易成功率, 交易响应时间数据发现其中部分异常值为网络波动引起, 而另一部分是确实由系统故障引起, 从而证明了本文采用的阈值判别方法的合理性.

    \textbf{关键词}: ATM, 特征提取, EM算法, 统计检验
\end{abstract}
