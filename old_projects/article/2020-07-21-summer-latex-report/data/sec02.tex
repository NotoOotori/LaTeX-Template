\section{数论选讲}

本部分旨在用Rivoal方法证明$\zeta(3)$是无理数. 主要参考资料为\cite{zoudiji2019zeta1}.

\begin{definition}[Riemann zeta函数]
    Riemann zeta函数$\zeta(s)$当$s$的实部大于1时, 由如下式子定义:
    \begin{equation}
        \zeta(s) = \sum_{n=1}^\infty \frac{1}{n^s}.
    \end{equation}
\end{definition}

\begin{definition}[Legendre多项式]
    多项式
    \begin{equation}
        P_n(x)=\frac{1}{n!}\frac{\diff^n}{\diff x^n}\lr(){x^n(1-x)^n} = \sum_{j=0}^n p_{nj}x^j
    \end{equation}
    为整系数多项式, 称为Legendre多项式.
\end{definition}

\begin{theorem}
    $\zeta(3)$是无理数.
\end{theorem}

\begin{proof}
    取辅助函数
    \begin{equation}
        f_n(x) = \int_0^1 \frac{P_n(y)}{1-xy}\log(xy)\diff y.
    \end{equation}
    则积分$\int_0^1 P_n(x)f_n(x)\diff x$是
    \begin{equation}
        \iint_{[0, 1]^2}\frac{x^iy^j}{1-xy}\log(xy)\diff x\diff y,\quad 0\leq i, j\leq n
    \end{equation}
    的整系数线性组合. 由于
    \begin{equation}
        \frac{1}{1-xy}=\sum_{k=0}^{\infty}x^ky^k,
    \end{equation}
    我们有
    \begin{align}
        &{\phantom{{}={}}}\iint_{[0, 1]^2}\frac{x^iy^j}{1-xy}\log(xy)\diff x\diff y\label{eqn: Fseries}\\
        &= \sum_{k=0}^\infty \iint_{[0, 1]^2} x^{i+k}y^{j+k}\log(xy)\diff x\diff y\\
        &= \sum_{k=0}^\infty\lr(){\int_0^1 x^{i+k}\log(x)\diff x\int_0^1 y^{j+k}\diff y + \int_0^1 x^{i+k}\diff x\int_0^1 y^{j+k}\log(y)\diff y}\\
        &= -\sum_{k=0}^\infty\lr(){\frac{1}{(i+k+1)^2(j+k+1)}+\frac{1}{(i+k+1)(j+k+1)^2}}.
    \end{align}
    注意到当$i\neq j$时有
    \begin{equation}
        \frac{1}{(i+k+1)^2(j+k+1)}+\frac{1}{(i+k+1)(j+k+1)^2} = \frac{1}{j-i}\lr(){\frac{1}{(i+k+1)^2}-\frac{1}{(j+k+1)^2}},
    \end{equation}
    而当$i=j$时有\ref{eqn: Fseries}等于
    \begin{equation}
        -2\lr(){\zeta(3)-\lr(){1+\frac{1}{2^3}+\dotsb + \frac{1}{i^3}}}.
    \end{equation}
    因此, 如果$\zeta(3)=a\divslash b$是有理数, 则有
    \begin{equation}
        \int_0^1P_n(x)f_n(x)\diff x = \frac{A_n}{bd_n^3},
    \end{equation}
    其中$d_n$为前$n$个正整数的最小公倍数.

    另一方面, 利用分部积分, 我们可以得到
    \begin{align}
        \int_0^1 P_n(x)f_n(x)\diff x
        &= \iint_{[0, 1]^2}\frac{P_n(x)P_n(y)}{1-xy}\log(xy)\diff x\diff y\label{eqn: Fint}\\
        &= \iiint_{[0, 1]^3}\frac{P_n(x)P_n(y)}{1-(1-xy)z}\diff x\diff y\diff z\\
        &= (-1)^n\iiint_{[0, 1]^3}\frac{x^n(1-x)^nP_n(y)y^nz^n}{(1-(1-xy)z)^{n+1}}\diff x\diff y\diff z.
    \end{align}
    做变量代换
    \begin{equation}
        w = \frac{1-z}{1-(1-xy)z},
    \end{equation}
    得\ref{eqn: Fint}等于
    \begin{equation}
        (-1)^{n+1}\iiint_{[0, 1]^3}\frac{(1-x)^nP_n(y)(1-w)^n}{1-(1-xy)w}\diff x\diff y\diff w.
    \end{equation}
    再对$y$进行$n$次分部积分, 可得\ref{eqn: Fint}等于
    \begin{equation}
        -\iiint_{[0, 1]^3}\lr(){\frac{x(1-x)y(1-y)w(1-w)}{1-(1-xy)w}}^n\frac{\diff x\diff y\diff w}{1-(1-xy)w}\neq 0.
    \end{equation}
    因为$d_n <3^n$, 且
    \begin{equation}
        \sup_{(x, y, w)\in [0, 1]^3}\frac{x(1-x)y(1-y)w(1-w)}{1-(1-xy)w} = (\sqrt{2}-1)^4,
    \end{equation}
    于是有
    \begin{equation}
        1\leq |A_n|=\lr\vert\vert{bd_n\int_0^1 P_n(x)f_n(x)\diff x}\leq\lr\vert\vert{b\lr(){27(\sqrt{2}-1)^4}^n\iiint_{[0, 1]^3}\frac{\diff x\diff y\diff w}{1-(1-xy)w}}\to 0,
    \end{equation}
    矛盾! 因此$\zeta(3)$是无理数.
\end{proof}

% \begin{theorem}
%     设$\xi$是实数. 如果存在实函数$f$, 使得对于$j\in \BN$, 存在有理系数多项式$F_i\in \BQ[x]$, 使得
%     \begin{equation}
%         \int_0^1 x^jf(x)\diff x=F_j(\xi)
%     \end{equation}
%     成立, 并且对于无穷多个$n$有
%     \begin{equation}
%         \int_0^1 P_n(x)f(x)\diff x\neq 0,
%     \end{equation}
%     其中$P_n$为Legendre多项式
%     \begin{equation}
%         P_n(x)=\frac{1}{n!}\frac{\diff^n}{\diff x^n}\lr(){x^n(1-x)^n},
%     \end{equation}
%     则$\xi$是无理数.
% \end{theorem}

% \begin{proof}
%     考虑Legendre多项式
%     \begin{equation}
%         P_n(x)=\frac{1}{n!}\frac{\diff^n}{\diff x^n}\lr(){x^n(1-x)^n} = \sum_{j=0}^n p_{nj}x^j,
%     \end{equation}
%     有每个$p_{nj}\in\BZ$, 并且积分
%     \begin{equation}
%         \int_0^1 P_n(x)f(x)\diff x=\sum_{j=0}^np_{nj}F_j(\xi).
%     \end{equation}
%     因此若$\xi$是有理数, 则对每个$n$上述积分都是有理数. 设
%     \begin{equation}
%         \int_0^1 P_n(x)f(x)\diff x = \frac{A_n}{B_n},
%     \end{equation}
%     其中$A_n, B_n$为互素整数.
% \end{proof}
