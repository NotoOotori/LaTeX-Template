% !Mode::"TeX:UTF-8"
\documentclass[a4paper,12pt]{ctexart}
\usepackage{amsmath}
%\usepackage{amsthm} %定理格式 由ntheorem代替
\usepackage{amssymb}
\usepackage[thmmarks, amsmath, thref]{ntheorem}
\usepackage{DefaultTheormStyle}
\usepackage{lastpage}
\usepackage{makecell} %表格线加粗 \Xhline{1.2pt}
\usepackage{boldline} %长表格表格线加粗
\usepackage{multirow} %合并单元格
\usepackage{array}
\usepackage{longtable} %长表格
\usepackage[dvipsnames]{xcolor} %颜色声明
\usepackage{varioref} %For Cross References
\renewcommand{\reftextbefore}
    {on the \reftextvario{preceding page}{page before}}
\renewcommand{\reftextafter}
    {on the \reftextvario{following}{next} page}
\renewcommand{\reftextfacebefore}
    {on the \reftextvario{facing}{preceding} page}
\renewcommand{\reftextfaceafter}
    {on the \reftextvario{facing}{next}{page}}
\renewcommand{\reftextfaraway}[1]
    {在第\pageref{#1}页}
\usepackage{caption} %题注
\captionsetup{margin    =   6pt,
              font      =   small,
              labelfont =   bf}
\usepackage{fancyhdr} %脚注
\setlength{\headheight}{15pt}
\usepackage[square, numbers, sort&compress]{natbib} %引用
\renewcommand{\bibsection}{} %不显示"Reference"
\usepackage{hyperref}
\hypersetup{linktoc             =   all,
            colorlinks          =   true,
            linkcolor           =   TealBlue,
           %anchorcolor         =   Black,
            citecolor           =   Black,
           %filecolor           =   Cyan,
           %menucolor           =   Red,
           %runcolor            =   filecolor,
            urlcolor            =   magenta,
            pdfinfo             =   {
                Title           =   {数学建模论文大作业模板},
                Author          =   {陈旭阳},
                Subject         =   {数学建模 论文模板}},
            bookmarksnumbered   =   true,
            pdfstartview        =   FitH,
            pdfpagelayout       =   OneColumn}
\usepackage[section]{placeins} % 使图像不会显示在别的部分 若过于严格则换成[below]
%\renewcommand{\tablename}{表}
%\renewcommand{\figurename}{图}
%\renewcommand{\contentsname}{目录}
%\renewcommand{\abstractname}{摘要}
\usepackage{graphicx}
\graphicspath{{figures/}} %图像文件目录
\usepackage[section]{placeins} % 使图像不会显示在别的部分 若过于严格则换成[below]
%\usepackage{fontspec} % 字体
\usepackage{titlesec} %Section标题格式
\usepackage{SUBSubsubsection}
\usepackage{authblk} %作者
\usepackage{stackrel} %上下写
%\usepackage{enumitem} 用enumerate包代替
\usepackage{listings} %排版程序语言
\usepackage{enumerate}
\usepackage{flafter} %不让float出现在定义之前的地方
\usepackage{float} %你们这帮float给我乖乖听话 HHHHHHHHHHH
\usepackage{pgfplots}
\pgfplotsset{width=7cm}
\usepackage{bigfoot} % to allow verbatim in footnote
\usepackage[numbered, framed]{matlab-prettifier}
\usepackage{filecontents}
\usepackage[all,cmtip]{xy} % Commutive diagram.
\usepackage{lineno} % Line numbers.

%===============TITLE===============
\title{个人作业2}
\author{1753763 陈旭阳}
\date{\today}

\begin{document}
    \pagestyle{empty}
    \pagenumbering{roman}
    \maketitle
    \newpage
%===============BODY===============
    \pagestyle{fancy}
    \pagenumbering{arabic}
    \linenumbers

    \begin{solution}
        分为两个步骤证明本题.
        \begin{enumerate}[Step 1.]
        \item 证明存在邻域$N(0)$, 存在$\alpha >0, k\in \mathbb{Z^+}$
            使得\[f(x)=x-\alpha x^{k+1}+O(x^{k+2}).\]
            
            因为$x_n\downarrow 0$, 即$f(x_n)\le x_n$, 又由$f$的连续性, 有
            \[f(0) = \lim_{x\to 0}{f(x)=0}.\]
            
            因为$x_{n+1}/x_n \to 1$, 即
            \[\lim_{x\to 0}{\frac{f(x)}{x}}=1,\] 所以$f'(0)=1$.

            因为$f$在原点解析, 所以存在邻域$N(0)$,
            存在$\alpha \in \mathbb{R}-\{0\}$, $k\in \mathbb{Z^+}$
            使得\[f(x)=x-\alpha x^{k+1}+O(x^{k+2}).\]
            因为有一列$\{x_n\}\to 0$使得$f(x_n) \le x_n$,
            故$\alpha <0$.
        \item 证明$x_n \sim (\alpha k n)^{1/k}.$\\
            设$y_n=x_n^{-k}.$ 有
            \begin{equation}
                \begin{aligned}
                    x_{n+1}^{-k} &= f(x_n)^{-k}\\
                    &= (x_n-\alpha x_n^{k+1}+O(x_n^{k+2}))^{-k}\\
                    &= x_n^{-k}(1-\alpha x_n^k + O(x_n^{k+1}))^{-k}\\
                    &= x_n^{-k}(1+\alpha kx_n^{k}+O(x_n^{2k}))\\
                    &= x_n^{-k}+\alpha k + O(x_n^k).
                \end{aligned}
            \end{equation}
            即
            \begin{equation}
                \begin{aligned}
                    y_{n+1} &= y_n + \alpha k + O(y_n^{-1})\\
                    y_{n+1} &= y_1 + n\alpha k + O(\sum_{j=1}^{n}{y_n^{-1}})\\
                    \frac{y_{n+1}}{n} &= \frac{y_1}{n} + \alpha k
                        + O((\sum_{j=1}^{n}{y_n^{-1}})/n)\\
                \end{aligned}
            \end{equation}
            因为$\lim_{n}{y_n^{-1}}$存在, 所以
            \[\lim_{n}{(\sum_{j=1}^{n}{y_n^{-1}})/n}\]存在, 即
            \[\lim_{n}{\frac{y_{n+1}}{n}} = \alpha k + O(1).\]
            所以$x_n \sim (\alpha k n)^{1/k}.$
        \end{enumerate}
    \end{solution}
\end{document}
