% !Mode::"TeX:UTF-8"

% -------------------- Information --------------------

\newcommand{\TITLE}{个人作业5}
\newcommand{\AUTHOR}{Jason}
\newcommand{\SUBJECT}{经典数学专题选讲}
\newcommand{\KEYWORDS}{}

% -------------------- Packages --------------------

\documentclass[a4paper,12pt]{ctexart}
\usepackage{amsmath}
\usepackage{amssymb}
%\usepackage{amsthm} % 定理格式 由ntheorem代替.
\usepackage{authblk} % 作者 (见校赛论文).
\usepackage{array}
\usepackage{bigfoot} % to allow verbatim in footnote.
\usepackage{bm} % \bm for bold symbols.
\usepackage{boldline} % 长表格表格线加粗.
\usepackage{caption} % 题注.
\usepackage{commath} % abs, norm
\usepackage{enumerate}
%\usepackage{enumitem} 用enumerate包代替.
\usepackage{fancyhdr} % 脚注.
\usepackage{filecontents}
\usepackage{flafter} % 不让float出现在定义之前的地方.
\usepackage{float} % 你们这帮float给我乖乖听话 HHHHHHHHHHH.
\usepackage[T1]{fontenc} % Bera Mono Font
\usepackage{fontspec} % 字体.
\usepackage{graphicx}
\usepackage{hyperref}
\usepackage{lastpage}
\usepackage{lipsum}
\usepackage{listings} % 排版程序语言.
\usepackage{longtable} % 长表格.
\usepackage{makecell} % 表格线加粗 \Xhline{1.2pt}.
\usepackage{mathtools} % \xleftrightarrow.
\usepackage{mathrsfs} % \mathscr
\usepackage{multirow} % 合并单元格.
\usepackage[square, numbers, sort&compress]{natbib} % 引用.
\usepackage[thmmarks, amsmath, thref]{ntheorem} % 定理格式.
\usepackage[section]{placeins} % 使图像不会显示在别的部分 若过于严格则换成[below].
\usepackage{stackrel} % 上下写 见校赛论文.
\usepackage{subcaption} % subcaption and subfigure
%\usepackage{SUBSubsubsection}
\usepackage{titlesec} % Section标题格式.
\usepackage{varioref} % For Cross References.
\usepackage[dvipsnames]{xcolor} % 颜色声明.
\usepackage[all, cmtip]{xy} % Commutive diagram.

% Require `ntheorem'

\usepackage[mathlines, edtable]{lineno} % Line numbers.
    %\begin{edtable}{tabular}[<args>] <entries> \end{edtable}

% Require `xcolor'

\usepackage[numbered, framed]{matlab-prettifier}
\usepackage{pgfplots}
\usepackage{tikz}

% Incompatible with `matlab-prettifier'

\usepackage[printwatermark]{xwatermark} % Foreground Watermarks.

% -------------------- Settings --------------------

% Title

\title{\TITLE}
\author{\AUTHOR}
\date{\today}

% Package: caption

\captionsetup{
    margin    =   6pt,
    font      =   small,
    labelfont =   bf
}

% Package: ctex

\setCJKfamilyfont{fzstk}{FZShuTi} % 方正舒体
\newcommand{\fzstk}{\CJKfamily{fzstk}}

% Package: fancyhdr

\setlength{\headheight}{15pt}
\lhead{Copyright \copyright\ \AUTHOR}
\rhead{Page \thepage\ of \pageref{LastPage}}

% Package: graphicx

\graphicspath{{resources/}} % 图像文件目录

% Package: hyperref

\hypersetup{
    linktoc             =   all,
    colorlinks          =   true,
    linkcolor           =   cyan,
    anchorcolor         =   black,
    citecolor           =   green,
    filecolor           =   cyan,
    menucolor           =   red,
    runcolor            =   filecolor,
    urlcolor            =   magenta,
	pdftitle           	=   {\TITLE},
	pdfauthor          	=   {\AUTHOR},
	pdfsubject         	=   {\SUBJECT},
	pdfcreator			=	{Visual Studio Code},
	pdfproducer			=	{XeLaTeX with documentclass ctexart},
	pdfkeywords        	=   {\KEYWORDS},
    bookmarksnumbered   =   true,
    pdfstartview        =   FitH,
    pdfpagelayout       =   OneColumn
}

% Package: lineno

\renewcommand{\linenumberfont}{\normalfont\scriptsize\sffamily}

\let\oldlstinputlisting\lstinputlisting
\renewcommand{\lstinputlisting}[2][\empty]{
    \par\nolinenumbers\oldlstinputlisting[#1]{#2}\linenumbers\par
}

\let\oldlstlisting\lstlisting
\let\oldendlstlisting\endlstlisting
\renewenvironment{lstlisting}
    {\par\nolinenumbers\oldlstlisting}
    {\oldendlstlisting\endnolinenumbers\par}

\let\oldtable\table
\let\oldendtable\endtable
\renewenvironment{table}
    {\par\nolinenumbers\oldtable}
    {\oldendtable\endnolinenumbers\par}

% Package: listings

\lstMakeShortInline[style=MATLAB-editor, basicstyle=\mlttfamily]&

\lstset{
    breaklines=true,
    backgroundcolor=\color{lightgray},
    basicstyle=\scriptsize,
    numbers=left,
    numberstyle={\color{black!33}\scriptsize\sffamily},
    xleftmargin=2em,
    xrightmargin=2em
}

% Package: ntheorem

%% Theorem
\newtheorem{theorem}{Theorem}[section]
\newtheorem{lemma}[theorem]{Lemma}
\newtheorem{corollary}[theorem]{Corollary}
%% Problem
\theoremstyle{plain}
\newtheorem{problem}{Problem}[section]
%% Definition
\theoremstyle{plain}
\theoremheaderfont{\bfseries}
\theorembodyfont{\rmfamily}
\newtheorem{definition}{Definition}[section]
%% Note
\theoremstyle{plain}
\theoremheaderfont{\itshape}
\theorembodyfont{\itshape}
\newtheorem{note}{Note}[section]
%% Proof
\theoremstyle{nonumberplain}
\theoremheaderfont{\itshape}
\theorembodyfont{\upshape}
\theoremseparator{.}
\theoremsymbol{\ensuremath{\square}}
\newtheorem{proof}{Proof}
%% Solution
\theoremsymbol{\ensuremath{\blacksquare}}
\newtheorem{solution}{Solution}

% Package: pgfplot

\pgfplotsset{width=7cm, compat=1.16}

% Package: varioref

\renewcommand{\reftextbefore}
    {on the \reftextvario{preceding page}{page before}}
\renewcommand{\reftextafter}
    {on the \reftextvario{following}{next} page}
\renewcommand{\reftextfacebefore}
    {on the \reftextvario{facing}{preceding} page}
\renewcommand{\reftextfaceafter}
    {on the \reftextvario{facing}{next}{page}}
\renewcommand{\reftextfaraway}[1]
    {on page \pageref{#1}}

% Package: xwatermark

\newsavebox\mybox
\savebox\mybox{\tikz[color=cyan, opacity=0.2]\node{\fzstk\SUBJECT};}
\newwatermark*[
    allpages,
    angle=45,
    scale=6,
    xpos=-20,
    ypos=15
]{\usebox\mybox}

% -------------------- General new commands --------------------

\DeclareMathAlphabet{\mathsfsl}{OT1}{cmss}{m}{sl}

\DeclareMathOperator{\arcosh}{arcosh}
\DeclareMathOperator{\Arcosh}{Arcosh}
\DeclareMathOperator{\Beta}{B}
\DeclareMathOperator{\diff}{d}
\DeclareMathOperator{\Log}{Log}

% Expectation

\newcommand{\expect}{\operatorname{E}\expectarg}
\DeclarePairedDelimiterX{\expectarg}[1]{(}{)}{
    \ifnum\currentgrouptype=16 \else\begingroup\fi
    \activatebar#1
    \ifnum\currentgrouptype=16 \else\endgroup\fi
}

\newcommand{\innermid}{\nonscript\;\delimsize\vert\nonscript\;}
\newcommand{\activatebar}{
    \begingroup\lccode`\~=`\|
    \lowercase{\endgroup\let~}\innermid
    \mathcode`|=\string"8000
}

\newcommand{\br}{\mathbb{R}}
\newcommand{\matr}[1]{\ensuremath{\mathsfsl{#1}}} % italic sans serif
\newcommand{\me}{\mathrm{e}}
\newcommand{\mi}{\mathrm{i}}
\newcommand{\restrict}[1]{\raisebox{-.5ex}{$|$}_{#1}}
\newcommand{\vect}[1]{\bm{#1}}

% -------------------- Specific new commands --------------------

\DeclareMathOperator{\rank}{R}
\newcommand{\lntr}[1]{\mathscr{#1}}
\DeclareMathOperator{\image}{Im}
\DeclareMathOperator{\mO}{O}

% -------------------- Document --------------------

\begin{document}

    % -------------------- Title Page --------------------

    \maketitle
    \thispagestyle{empty}
    \pagenumbering{roman}

    % -------------------- Abstract Page --------------------

    % -------------------- Contents --------------------

    %\newpage
    %\tableofcontents

    % -------------------- Body --------------------

    \newpage
    \pagestyle{fancy}
    \pagenumbering{arabic}
    \linenumbers

    \begin{problem}
        设$\matr{A}$是三阶实方阵, 并且对任意的$\vect{v}\in\br$
        有$\matr{A}\vect{v}\perp\vect{v}$.
        证明: 存在$\vect{a}\in\br^{3}$使得
        $\matr{A}\vect{a}=\vect{a}\times\vect{v}$恒成立.
    \end{problem}

    {\huge{在写解答之前先说一句: 我好讨厌这个做法啊!}}

    \begin{theorem}
        \label{theorem_antisymmetric}
        $\matr{A}$为反对称阵.
    \end{theorem}

    \begin{proof}
        由$\matr{A}\vect{v}\perp\vect{v}$,
        有
        \begin{equation}
        \begin{aligned}
            \vect{v}^{\top}\matr{A}^{\top}\vect{v} &= 0\\
            \vect{v}^{\top}\matr{A}\vect{v} &= 0.
        \end{aligned}
        \end{equation}
        故$\vect{v}^{\top}(\matr{A}^{\top}+\matr{A})\vect{v}=0,
        \forall\vect{v}\in\br^{3}$.
        又因为$\matr{A}^{\top}+\matr{A}$是对称阵, 所以其合同于零矩阵,
        故$\matr{A}^{\top}+\matr{A}=\matr{O}$, 即$\matr{A}$为反对称阵.
    \end{proof}

    \begin{theorem}
        设
        \begin{equation}
            \label{equation_defineA}
            \matr{A}=
            \begin{pmatrix}
                0 & m & n\\
                -m & 0 & p\\
                -n & -p & 0
            \end{pmatrix},
        \end{equation}
        那么$\vect{a}=(-p, n, -m)^{\top}$即为所求.
    \end{theorem}

    \begin{proof}
        没什么好证的吧, 代入验算一下就行了.
    \end{proof}

    % -------------------- Bibliography --------------------

    %\newpage
    %\bibliography{Principles_of_Mathematical_Analysis}
    %\bibliographystyle{plain}

    % -------------------- Appendix --------------------

    \newpage
    \appendix

    \section{一个过于复杂的证明}

    首先$\matr{A}$为反对称阵.

    \begin{theorem}
        关于矩阵$\matr{A}$的秩我们有
        \begin{equation}
            \label{equation_rank}
            \rank(\matr{A}) = 0\ \text{or}\ 2
        \end{equation}
    \end{theorem}

    \begin{proof}
        首先因为$\abs{\matr{A}}=\abs{-\matr{A}^{\top}} = 0$,
        所以$\rank(\matr{A})\leq 2$.

        其次假设$\rank(\matr{A})=1$,
        那么$\abs{\matr{A}}$的二阶子式均为零.
        我们考察$\matr{A}_{11}, \matr{A}_{22}, \matr{A}_{33}$, 可得
        \begin{equation}
            \label{equation_subdeterminant}
            -a_{12}a_{21}=-a_{13}a_{31}=-a_{23}a_{32}=0,
        \end{equation}
        即$\matr{A}=\matr{O}$, 与$\rank(\matr{A})=1$矛盾.
    \end{proof}

    因为$\rank(\matr{A})=0$的情况平凡,
    所以之后我们仅考虑$\rank(\matr{A})=2$的情况.

    \begin{definition}
        定义线性变换$\lntr{T}$使其在标准正交基下的矩阵为$\matr{A}$,
        并且记$S=\image\lntr{T}$为$\lntr{T}$的像空间.
    \end{definition}

    我们有$S$是$\lntr{T}$的不变子空间, 并且$\dim{S}=2$.
    
    \begin{theorem}
        对于任意向量$\vect{v}\in\br^{3}$, 都存在向量$\vect{u}\in S$
        使得$\matr{A}\vect{v}=\matr{A}\vect{u},
        \vect{a}\times\vect{u}=\vect{a}\times\vect{v}$成立.
    \end{theorem}

    \begin{proof}
        令$\displaystyle\vect{u}=\vect{v}-
        \frac{\vect{v}\cdot\vect{a}}{\abs{\vect{a}}}
        \frac{\vect{a}}{\abs{\vect{a}}}$即可.
    \end{proof}

    由此定理我们可以只用考虑$S$中的向量.

    \begin{theorem}
        若将线性变换$\lntr{T}$限制在$S$上,
        则它在$S$的一组正交基下的矩阵为反对称矩阵,
        即$\lntr{T}\restrict{S}$为$k$倍的旋转$\pi/2$的变换.
    \end{theorem}

    \begin{proof}
        为反对称矩阵的理由与$A$为反对称阵的理由相同.
    \end{proof}

    于是$\vect{a}$在$S$的法线上, 模长为$k$,
    方向为$\lntr{T}\restrict{S}$旋转$\pi/2$的方向.
\end{document}
