% !Mode::"TeX:UTF-8"
\documentclass[a4paper,12pt]{ctexart}
\usepackage{Constants}
\usepackage{amsmath}
%\usepackage{amsthm} %定理格式 由ntheorem代替
\usepackage{amssymb}
\usepackage[thmmarks, amsmath, thref]{ntheorem}
\usepackage{DefaultTheoremStyle}
\usepackage{lastpage}
\usepackage{makecell} %表格线加粗 \Xhline{1.2pt}
\usepackage{boldline} %长表格表格线加粗
\usepackage{multirow} %合并单元格
\usepackage{array}
\usepackage{longtable} %长表格
\usepackage[dvipsnames]{xcolor} %颜色声明
\usepackage{varioref} %For Cross References
\renewcommand{\reftextbefore}
    {on the \reftextvario{preceding page}{page before}}
\renewcommand{\reftextafter}
    {on the \reftextvario{following}{next} page}
\renewcommand{\reftextfacebefore}
    {on the \reftextvario{facing}{preceding} page}
\renewcommand{\reftextfaceafter}
    {on the \reftextvario{facing}{next}{page}}
\renewcommand{\reftextfaraway}[1]
    {在第\pageref{#1}页}
\usepackage{caption} %题注
\captionsetup{margin    =   6pt,
              font      =   small,
              labelfont =   bf}
\usepackage{fancyhdr} %脚注
\setlength{\headheight}{15pt}
\lhead{Copyright \copyright\ \AUTHOR}
\rhead{Page \thepage\ of \pageref{LastPage}}
\usepackage[square, numbers, sort&compress]{natbib} %引用
\renewcommand{\bibsection}{} %不显示"Reference"
\usepackage{hyperref}
\hypersetup{linktoc             =   all,
            colorlinks          =   true,
            linkcolor           =   TealBlue,
           %anchorcolor         =   Black,
            citecolor           =   Black,
           %filecolor           =   Cyan,
           %menucolor           =   Red,
           %runcolor            =   filecolor,
            urlcolor            =   magenta,
            pdfinfo             =   {
                Title           =   {\TITLE},
                Author          =   {\AUTHOR},
                Subject         =   {\SUBJECT}},
            bookmarksnumbered   =   true,
            pdfstartview        =   FitH,
            pdfpagelayout       =   OneColumn}
\usepackage[section]{placeins} % 使图像不会显示在别的部分 若过于严格则换成[below]
%\renewcommand{\tablename}{表}
%\renewcommand{\figurename}{图}
%\renewcommand{\contentsname}{目录}
%\renewcommand{\abstractname}{摘要}
\usepackage{graphicx}
\graphicspath{{figures/}} %图像文件目录
\usepackage[section]{placeins} % 使图像不会显示在别的部分 若过于严格则换成[below]
%\usepackage{fontspec} % 字体
\usepackage{titlesec} %Section标题格式
\usepackage{SUBSubsubsection}
\usepackage{authblk} %作者
\usepackage{stackrel} %上下写
\usepackage{mathtools} %\xleftrightarrow
%\usepackage{enumitem} 用enumerate包代替
\usepackage{listings} %排版程序语言
\usepackage{enumerate}
\usepackage{flafter} %不让float出现在定义之前的地方
\usepackage{float} %你们这帮float给我乖乖听话 HHHHHHHHHHH
\usepackage{pgfplots}
\pgfplotsset{width=7cm}
\usepackage{bigfoot} % to allow verbatim in footnote
\usepackage[numbered, framed]{matlab-prettifier}
\usepackage{filecontents}
\usepackage[all,cmtip]{xy} % Commutive diagram.
\usepackage{commath} % abs, norm
\newcommand\restrict[1]{\raisebox{-.5ex}{$|$}_{#1}}

\title{\TITLE}
\author{\AUTHOR}
\date{\today}

\begin{document}
    %\maketitle
    %\thispagestyle{empty}
    %\newpage
    %\pagestyle{fancy}
    \pagenumbering{arabic}

    \begin{problem}
        设$G$是$\mathbb{C}$中不含原点的开区域, 请研究满足$f(z)\in \mathrm{Arg}(z)$
        的连续函数$f:G\to \mathbb{C}$的存在性.
    \end{problem}

    我们猜测: 存在满足所要求的连续函数$f$当且仅当对于$G$中的任意一条简单闭曲线
    $\Gamma$都有$0\notin I(\Gamma)$,
    其中$I(\Gamma)$表示简单闭曲线$\Gamma$的内部.

    记$\mathfrak{R}_{-} \subseteq \mathbb{C}$为负实轴.

    证明思路:
    \begin{itemize}
    \item '$\Rightarrow$': 如果存在这样的简单闭曲线$\Gamma$,
        我们需要证明$f$不可能在$\Gamma$上连续.
        但由于映射$\mathrm{arg}\restrict{\Gamma}$不是单射, 证明过程中遇到了瓶颈.
    \item '$\Leftarrow$': 构造函数$f$. 固定$p \in G$并令$f(p)=\mathrm{arg}(p)$.
        若$z\in G$且$z \neq p$, 设$L_z$为从$p$到$z$的有向折线段,
        且任意一段都不与实轴平行.
        定义$n(z)$为$L_z$从上往下穿过(或碰到)$\mathfrak{R}_{-}$的次数减去
        $L_z$从下往上穿过(或碰到)$\mathfrak{R}_{-}$的次数.
        定义$f(z) = \mathrm{arg}(z)+2\pi n(z)$.

        当然我们首先要讨论$z\in \mathfrak{R}_{-}$
        和$L_z$与$\mathfrak{R}_{-}$相切的情况, 把定义说清楚.
        再证明$n(z)$与$L_z$的选取无关, 证明定义的良定性.
        这之后我们只需要证明$f$的连续性, 也就是说$f$在$\mathfrak{R}_{-}$上的连续性,
        由我们对$f$的构造还是比较显然的吧.
    \end{itemize}
    \begin{problem}
        设$A$是$n$阶可逆实方阵, 如果对任意的非零向量
        $\pmb{u}, \pmb{v} \in \mathbb{R}^n$, 夹角$\angle(\pmb{u}, \pmb{v})$与
        $\angle(A\pmb{u}, A\pmb{v})$恒等. 试问$A$的结构.
    \end{problem}
    
    $A$满足题中要求当且仅当$A$可表示成$A=Q(kI), Q\in \mathrm{O}(n), k\in \mathbb{R}-\{0\}$.

    \begin{itemize}
    \item '$\Rightarrow$': 首先考虑$\pmb{u}, \pmb{v}$为各个单位向量,
        可得$A$的列向量两两正交, 即$A$可表示为
        \begin{equation}
            A = QD, Q\in \mathrm{O}(n), D=\mathrm{diag}(x_1, x_2, \dotsc, x_n),
            \prod_{i=1}^n{x_i}\neq 0
        \end{equation}

        假设$x_i\neq x_j$, 那么令$\pmb{u}=\pmb{e}_i, \pmb{v}=\pmb{e}_i+\pmb{e}_j$.
        我们有$\angle(\pmb{u}, \pmb{v}) = \pi/4$, 以及
        \begin{equation}
            \tan{\angle(A\pmb{u}, A\pmb{v})} = \frac{x_j}{x_i} \neq 1.
        \end{equation}
        与题设矛盾! 故$x_1 = x_2 = \dotsb = x_n$.
    \item '$\Leftarrow$': 设$A=Q(kI), Q\in \mathrm{O}(n), k\in \mathbb{R}-\{0\}$
        于是
        \begin{align}
            \cos(\angle(A\pmb{u}, A\pmb{v}))
            &= \frac{(A\pmb{u})\cdot(A\pmb{v})}{\norm{A\pmb{u}}\norm{A\pmb{v}}}\\
            &= \frac{(A\pmb{u})^{T}(A\pmb{v})}{\norm{A}^2\norm{\pmb{u}}\norm{\pmb{v}}}\\
            &= \frac{\pmb{u}^{T}(kI)Q^{T}Q(kI)\pmb{v}}{k^2\norm{\pmb{u}}\norm{\pmb{v}}}\\
            &= \frac{\pmb{u}\cdot\pmb{v}}{\norm{\pmb{u}}\norm{\pmb{v}}}\\
            &= cos(\angle(\pmb{u}, \pmb{v})).
        \end{align}
        得证.
    \end{itemize}
\end{document}
