% !Mode::"TeX:UTF-8"

% -------------------- Information --------------------

\newcommand{\TITLE}{补充题9}
\newcommand{\AUTHOR}{Jason}
\newcommand{\SUBJECT}{复变函数补充题}
\newcommand{\KEYWORDS}{}

% -------------------- Packages --------------------

\documentclass[a4paper, 12pt]{ctexart}
\usepackage{amsmath}
\usepackage{amssymb}
% \usepackage{amsthm} % 定理格式 由ntheorem代替.
\usepackage{authblk} % 作者 (见校赛论文).
\usepackage{array}
\usepackage{bigfoot} % to allow verbatim in footnote.
\usepackage{bm} % \bm for bold symbols.
\usepackage{boldline} % 长表格表格线加粗.
\usepackage{caption} % 题注.
\usepackage{commath} % abs, norm
\usepackage{enumerate}
% \usepackage{enumitem} 用enumerate包代替.
\usepackage{fancyhdr} % 脚注.
\usepackage{filecontents}
\usepackage{flafter} % 不让float出现在定义之前的地方.
\usepackage{float} % 你们这帮float给我乖乖听话 HHHHHHHHHHH.
\usepackage[T1]{fontenc} % Bera Mono Font
\usepackage{fontspec} % 字体.
\usepackage{graphicx}
\usepackage{hyperref}
\usepackage{lastpage}
\usepackage{letltxmacro} % \let
\usepackage{lipsum}
\usepackage{listings} % 排版程序语言.
\usepackage{longtable} % 长表格.
\usepackage{makecell} % 表格线加粗 \Xhline{1.2pt}.
\usepackage{mathtools} % \xleftrightarrow.
\usepackage{mathrsfs} % \mathscr
\usepackage{multirow} % 合并单元格.
\usepackage[square, numbers, sort&compress]{natbib} % 引用.
\usepackage[thmmarks, amsmath, thref]{ntheorem} % 定理格式.
\usepackage[section]{placeins} % 使图像不会显示在别的部分 若过于严格则换成[below].
\usepackage{stackrel} % 上下写 见校赛论文.
\usepackage{subcaption} % subcaption and subfigure
% \usepackage{SUBSubsubsection}
\usepackage{titlesec} % Section标题格式.
\usepackage{varioref} % For Cross References.
\usepackage[dvipsnames]{xcolor} % 颜色声明.
\usepackage{xfrac} %\sfrac{}{}
\usepackage[all, cmtip]{xy} % Commutive diagram.

% Require `ntheorem'

\usepackage[mathlines, edtable]{lineno} % Line numbers.
    %\begin{edtable}{tabular}[<args>] <entries> \end{edtable}

% Require `xcolor'

\usepackage[numbered, framed]{matlab-prettifier}
\usepackage{pgfplots}
\usepackage{tikz}

% Incompatible with `matlab-prettifier'

\usepackage[printwatermark]{xwatermark} % Foreground Watermarks.

% -------------------- Settings --------------------

% Title

\title{\TITLE}
\author{\AUTHOR}
\date{\today}

% Package: caption

\captionsetup{
    margin    =   6pt,
    font      =   small,
    labelfont =   bf
}

% Package: ctex

\setCJKfamilyfont{fzstk}{FZShuTi} % 方正舒体
\newcommand{\fzstk}{\CJKfamily{fzstk}}

% Package: fancyhdr

\setlength{\headheight}{15pt}
\lhead{Copyright \copyright\ \AUTHOR}
\rhead{Page \thepage\ of \pageref{LastPage}}

% Package: graphicx

\graphicspath{{resources/}} % 图像文件目录

% Package: hyperref

\hypersetup{
    linktoc             =   all,
    colorlinks          =   true,
    linkcolor           =   cyan,
    anchorcolor         =   black,
    citecolor           =   green,
    filecolor           =   cyan,
    menucolor           =   red,
    runcolor            =   filecolor,
    urlcolor            =   magenta,
	pdftitle           	=   {\TITLE},
	pdfauthor          	=   {\AUTHOR},
	pdfsubject         	=   {\SUBJECT},
	pdfcreator			=	{Visual Studio Code},
	pdfproducer			=	{XeLaTeX with documentclass ctexart},
	pdfkeywords        	=   {\KEYWORDS},
    bookmarksnumbered   =   true,
    pdfstartview        =   FitH,
    pdfpagelayout       =   OneColumn
}

% Package: lineno

\renewcommand{\linenumberfont}{\normalfont\scriptsize\sffamily}

\let\oldlstinputlisting\lstinputlisting
\renewcommand{\lstinputlisting}[2][\empty]{
    \par\nolinenumbers\oldlstinputlisting[#1]{#2}\linenumbers\par
}

\let\oldlstlisting\lstlisting
\let\oldendlstlisting\endlstlisting
\renewenvironment{lstlisting}
    {\par\nolinenumbers\oldlstlisting}
    {\oldendlstlisting\endnolinenumbers\par}

\let\oldtable\table
\let\oldendtable\endtable
\renewenvironment{table}
    {\par\nolinenumbers\oldtable}
    {\oldendtable\endnolinenumbers\par}

% Package: listings

%% Title

\renewcommand\lstlistingname{代码}
\renewcommand\lstlistlistingname{代码}

%% Lstinline with color box

\LetLtxMacro{\oldlstinline}{\lstinline}
\renewcommand{\lstinline}[2][]{\colorbox{lightgray}{\oldlstinline[#1]{#2}}}
\newcommand{\matlabinline}[1]{
    \lstinline[style=MATLAB-editor, basicstyle=\mlttfamily]{#1}}

\lstset{
    breaklines=true,
    backgroundcolor=\color{lightgray},
    basicstyle=\scriptsize,
    inputpath=resources/,
    numbers=left,
    numberstyle={\color{black!33}\scriptsize\sffamily},
    xleftmargin=2em,
    xrightmargin=2em
}

% Package: ntheorem

%% Theorem
\newtheorem{theorem}{Theorem}[section]
\newtheorem{lemma}[theorem]{Lemma}
\newtheorem{corollary}[theorem]{Corollary}
%% Problem
\theoremstyle{plain}
\newtheorem{problem}{Problem}[section]
%% Proposition
\newtheorem{proposition}{Proposition}[section]
%% Conjecture
\newtheorem{conjecture}[proposition]{Conjecture}
%% Definition
\theoremstyle{plain}
\theoremheaderfont{\bfseries}
\theorembodyfont{\rmfamily}
\newtheorem{definition}{Definition}[section]
%% Note
\theoremstyle{plain}
\theoremheaderfont{\itshape}
\theorembodyfont{\itshape}
\newtheorem{note}{Note}[section]
%% Proof
\theoremstyle{nonumberplain}
\theoremheaderfont{\itshape}
\theorembodyfont{\upshape}
\theoremseparator{.}
\theoremsymbol{\ensuremath{\square}}
\newtheorem{proof}{Proof}
%% Solution
\theoremsymbol{\ensuremath{\blacksquare}}
\newtheorem{solution}{Solution}

% Package: pgfplot

\pgfplotsset{width=7cm, compat=1.16}

% Package: varioref

\renewcommand{\reftextbefore}
    {on the \reftextvario{preceding page}{page before}}
\renewcommand{\reftextafter}
    {on the \reftextvario{following}{next} page}
\renewcommand{\reftextfacebefore}
    {on the \reftextvario{facing}{preceding} page}
\renewcommand{\reftextfaceafter}
    {on the \reftextvario{facing}{next} page}
\renewcommand{\reftextfaraway}[1]
    {on page \pageref{#1}}

%% Label formats

\labelformat{lstlisting}{代码#1}
\labelformat{equation}{式(#1)}
\labelformat{figure}{图#1}
\labelformat{table}{表#1}

% Package: xwatermark

\newsavebox\mybox
\savebox\mybox{\tikz[color=cyan, opacity=0.2]\node{\fzstk\SUBJECT};}
\newwatermark*[
    allpages,
    angle=45,
    scale=6,
    xpos=-20,
    ypos=15
]{\usebox\mybox}

% -------------------- General new commands --------------------

\DeclareMathAlphabet{\mathsfsl}{OT1}{cmss}{m}{sl}

\DeclareMathOperator{\arcosh}{arcosh}
\DeclareMathOperator{\Arcosh}{Arcosh}
\DeclareMathOperator*{\Beta}{B}
\DeclareMathOperator{\Log}{Log}
\DeclareMathOperator*{\real}{Re}
\DeclareMathOperator*{\image}{Im}

% Expectation

\newcommand{\expect}{\operatorname{E}\expectarg}
\DeclarePairedDelimiterX{\expectarg}[1]{(}{)}{
    \ifnum\currentgrouptype=16 \else\begingroup\fi
    \activatebar#1
    \ifnum\currentgrouptype=16 \else\endgroup\fi
}

\newcommand{\innermid}{\nonscript\;\delimsize\vert\nonscript\;}
\newcommand{\activatebar}{
    \begingroup\lccode`\~=`\|
    \lowercase{\endgroup\let~}\innermid
    \mathcode`|=\string"8000
}

\newcommand*{\BC}{\mathbb{C}}
\newcommand*{\BR}{\mathbb{R}}
\newcommand*{\diff}{\mathop{}\!\mathrm{d}}
\newcommand*{\matr}[1]{\ensuremath{\mathsfsl{#1}}} % italic sans serif
\newcommand*{\me}{\mathrm{e}}
\newcommand*{\mi}{\mathrm{i}}
\newcommand*{\restrict}[1]{\raisebox{-.5ex}{$\vert$}_{#1}}
\newcommand*{\vect}[1]{\bm{#1}}

% -------------------- Specific new commands --------------------



% -------------------- Document --------------------

\begin{document}

    % -------------------- Title Page --------------------

    \maketitle
    \thispagestyle{empty}
    \pagenumbering{roman}

    % -------------------- Abstract Page --------------------

    % -------------------- Contents --------------------

    % \newpage
    % \tableofcontents

    % -------------------- Body --------------------

    \newpage
    \pagestyle{fancy}
    \pagenumbering{arabic}
    \linenumbers

    \begin{problem}
        设$f$是整函数, 并且对于任意的$a\in\BC$, $f$在$a$点的Taylor展开
        $\sum{c_{n}(z-a)^{n}}$至少有一项系数为零, 证明: $f$是多项式.
        [提示: 将不可数个点放到可数个抽屉, 至少有一个抽屉有不可数的点.]
    \end{problem}

    \begin{proof}
        令$B_{1}$为单位闭圆盘, 由题意知存在$n$使得对于无限个(可以做到不可数个)
        $a\in B_{1}$使得$f^{(n)}(a)=0$, 故由Weierstrass定理知
        $f^{(n)}$的零点集在$B_{1}$中有聚点, 故$f^{(n)}=0$, 即$f$为多项式.
    \end{proof}

    \begin{problem}
        设$G$是开区域, 请陈述并证明$G$上的调和函数的最大值原理, 唯一性定理.
        注意, 零点孤立性并不成立(如$u(x,y)=xy$), 所以调和函数的唯一性定理
        要比全纯函数的弱.
    \end{problem}

    \begin{theorem}
        设$G$是开区域, $u$为$G$上的调和函数. 如果$\max_{G}{\abs{u}}$存在,
        那么$u$为常数.
    \end{theorem}

    \begin{proof}
        设$v$为$G$上的调和函数使得$g(x+\mi y):=u(x,y)+\mi v(x,y)$
        为$G$上的全纯函数,
        令$f=\me^{g}$, 则$\abs{f}=\me^{u}$和
        $\abs{\sfrac{1}{f}}=\me^{-u}$中总有一个函数能在$G$中取到最大值.
        所以$\me^{g}$为常数, 故$u$为常数.
    \end{proof}

    \begin{theorem}
        设$G$是开区域, $u$为$G$上的调和函数. 如果存在开集$O\subseteq G$使得
        $u\restrict{O}=0$, 则$u=0$.
    \end{theorem}

    \begin{proof}
        由题$u_{x}\restrict{O}=u_{y}\restrict{O}=0$.
        由C-R方程, 若设调和函数$v$满足$u+\mi v$在$G$上全纯,
        则$v_{x}\restrict{O}=v_{y}\restrict{O}=0$, 即$v$在$O$上为常数.
        所以我们可以找到$G$上的全纯函数$f$使得$\real{f}=u$且$f\restrict{O}=0$.
        由全纯函数零点孤立性知$f=0$, 故$u=0$.
    \end{proof}

    \begin{theorem}
        设$G$是有界开区域, $u$为$\overline{G}$上的连续函数并且在$G$调和.
        如果$u\restrict{\partial G}=0$, 则$u=0$.
    \end{theorem}

    \begin{proof}
        设$g$为$\overline{G}$上的连续函数且在$G$全纯, 并满足
        $real{g}=u$, 设$f=\me^{g}$. 因为$f(z)\neq 0$, $\forall z$,
        又$f\restrict{\partial G}=1$, 由最大模原理和最小模原理知
        $f=1$, 故$g=0$.
    \end{proof}

    \begin{conjecture}
        若定义在开区域上的调和函数$u$的零点集有正Lebesgue测度, 那么$u=0$.
    \end{conjecture}

    \begin{problem}
        设$\{f_{n}\}$是区域$G$上的一列单叶解析函数, 并且在$G$上内闭一致收敛于$F$.
        假如$F$不是常值函数, 下面用反证法来证明$F$必是单叶解析函数.
        若$F$不是单叶函数, 则存在$a\neq b$使得$F(a)=F(b)$.
        对于充分小的$\delta>0$, 考察积分
        \begin{equation}
            I=\int_{\abs{z-b}=\delta}{\frac{F'(z)}{F(z)-F(b)}\diff{z}}.
        \end{equation}
        \begin{enumerate}[\hspace{2em}(1)]
            \item 说明被积函数的分母不为零, 从而积分有意义;
            \item 利用分解$F(z)-F(b)=(z-b)^{m}\varphi(z)$说明$I\neq 0$;
            \item 利用$f_{n}(z)-f_{n}(a)\to F(z)-F(b)$说明$I=0$, 导出矛盾.
        \end{enumerate}
    \end{problem}

    \begin{proof}
        因为$F$是区域$G$上解析函数列一致收敛的极限函数, 所以$F$解析.
        记$g(z)=F(z)-F(b)$是关于$z$的解析函数, 因为$g$的零点具有孤立性,
        所以存在$\delta>0$使得在去心闭邻域
        $N_{\delta}(b)$上有$F(z)-F(b)=g(z)\neq 0$, 故积分有意义.

        设$b$是$g$的$m>0$阶零点, 有因式分解
        \begin{equation}
            F(z)-F(b)=g(z)=(z-b)^{m}\varphi(z),
        \end{equation}
        且$\varphi(b)\neq 0$. 又
        $F'(z)=g'(z)=m(z-b)^{m-1}\varphi(z)+(z-b)^{m}\varphi'(z)$,
        于是
        \begin{equation}
        \begin{aligned}
            I &= \int_{\abs{z-b}=\delta}{\frac{g'(z)}{g(z)}\diff{z}}\\
            &= \int_{\abs{z-b}=\delta}{\left(
                \frac{m(z-b)^{m-1}\varphi(z)}{(z-b)^{m}\varphi(z)}
                + \frac{(z-b)^{m}\varphi'(z)}{(z-b)^{m}\varphi(z)}\right)
                \diff z}\\
            &= m\int_{\abs{z-b}=\delta}{(z-b)^{-1}\diff z}
              + \int_{\abs{z-b}=\delta}{\frac{\varphi'(z)}{\varphi(z)}\diff z}\\
            &= 2\pi\mi m \neq 0.
        \end{aligned}
        \end{equation}

        因为$\norm{F(z)-f_{n}(z)}\to 0$,
        所以
        \begin{equation}
        \begin{aligned}
            0 &\leq \norm{(F(z)-F(a))-(f_{n}(z)-f_{n}(a))}\\
            &\leq
            \norm{F(z)-f_{n}(z)}+\norm{F(a)-f_{n}(a)} \to 0,
        \end{aligned}
        \end{equation}
        又$F(a)=F(b)$, 所以$f_{n}(z)-f_{n}(a)$一致收敛到$F(z)-F(b)$,
        所以
        \begin{equation}
            0
            =
            \int_{\abs{z-b}=\delta}{\frac{F'(z)}{f_{n}(z)-f_{n}(a)}\diff{z}}
            = I_{n}
            \to I,
        \end{equation}
        即$I=0$, 矛盾! 所以$F$是单叶解析函数.
    \end{proof}

    \begin{problem}
        设$f(\theta)$是$\BR$上以$2\pi$为周期的$C^{1}$实函数,
        可以视$f$为定义在$\abs{z}=1$上的一个函数$g$,
        即$g(\me^{\mi\theta})=f(\theta)$.
        根据Dirichlet收敛定理它可展开为Fourier级数
        \begin{equation}
            f(\theta)=\sum_{n=-\infty}^{\infty}{c_{n}\me^{\mi n\theta}};
            \quad
            g(z)=\sum_{n=-\infty}^{\infty}{c_{n}z^{n}},\ \abs{z}=1.
        \end{equation}
        试问$g$是否能延拓为定义在某个包含$\abs{z}=1$上的开区域上的全纯函数?
    \end{problem}

    \begin{solution}
        根据Fourier系数的计算公式, 我们有$\abs{c_{n}}=\abs{c_{-n}}$.
        因此如果$\displaystyle\limsup_{n\to +\infty}{\sqrt[n]{\abs{c_{n}}}} < 1$
        则能延拓,
        如果$\displaystyle\limsup_{n\to +\infty}{\sqrt[n]{\abs{c_{n}}}} = 1$
        则不能延拓.

        事实上这两种情况都有可能取得到. 当$f=0$时, 只要$n\neq 0$, 都有$c_{n}=0$; 而当
        $f(x)=x$时,
        $\displaystyle\limsup_{n\to +\infty}{\sqrt[n]{\abs{c_{n}}}}=
        \limsup_{n\to +\infty}{\sqrt[n]{\sfrac{1}{n}}} = 1$
    \end{solution}

    % -------------------- Bibliography --------------------

    % \newpage
    % \bibliography{Principles_of_Mathematical_Analysis}
    % \bibliographystyle{plain}

\end{document}
