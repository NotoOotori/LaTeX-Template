经过长期努力, 中国特色社会主义进入了新时代. 习总书记在十九大报告中对中国特色社会主义新时代的本质内涵作了高度凝练和科学概括:"这个新时代, 是承前启后, 继往开来, 在新的历史条件下继续夺取中国特色社会主义伟大胜利的时代, 是决胜全面建成小康社会, 进而全面建设社会主义现代化强国的时代, 是全国各族人民团结奋斗, 不断创造美好生活, 逐步实现全体人民共同富裕的时代, 是全体中华儿女勠力同心, 奋力实现中华民族伟大复兴中国梦的时代, 是我国日益走近世界舞台中央, 不断为人类作出更大贡献的时代."\citep{Xi:19da}

新时代中国的发展不会一帆风顺. 在国内, 我国正处于社会矛盾凸显期. 经济方面, 中国依然面临供给侧结构性问题;\citep{Cheng:Revolution}
民生方面, 城乡发展不平衡, 贫富差距大的问题仍然十分严重;
生态方面, 仍有许多企业违规排放污染物质. 比如中国东北部一些工厂违规使用CFC-11(一种氟利昂)进行聚氨酯发泡, 造成全球氟利昂排放量的减少远不达预期.\citep{Nature:2018,Nature:2019}
因此在全面发展深化改革的同时保持社会稳定至关重要. 中央全面深化改革委员会第七次会议中强调"今年改革发展面临的风险挑战较多, 要把握形势发展变化, 化解突出矛盾和问题, 稳妥有序推进改革".\citep{Xi:Revolution}
国际上美国对中国发动了贸易战, 旨在对中国包括贸易和高新制造业等的各方面进行打压. 美国对大量进口中国商品大幅加征关税, 又先后对中兴, 华为两家企业进行制裁, 对中国经济带来了不小的冲击.

新青年是实现中华民族伟大复兴的中坚力量. 他们现在正在上学, 而中国建设社会主义现代化强国的时期便是他们历经社会的磨炼, 从象牙塔中的青年蜕变为能独当一面的骨干的过程. 正因为新时代中国发展将会面临坎坷, 新青年们也将承担更大的使命. 习总书记在纪念五四运动100周年大会上的讲话中对新时代中国青年做出了以下六点要求: 新时代中国青年要树立远大理想, 要担当时代责任, 要勇于砥砺奋斗, 要练就过硬本领, 要增强学习紧迫感.\citep{Xi:Teenagers}
总之, 新青年人要承担建设祖国的使命, 为祖国的强盛贡献出自己的一份力.
