由于交易数目, 交易成功率, 交易响应时间之间有一定的相关关系, 一种故障可能引起多种数据的异常, 同一种数据异常也有可能是由不同的故障引起的, 因此并不能直接根据现有的观测数据准确的鉴别故障的类别. 为对各种故障进行鉴别, 需要更细致的采集数据. 常见的故障场景包括分行侧网络传输节点故障, 分行侧参数数据变更或者配置错误, 数据中心后端处理系统异常, 数据中心后端处理系统应用进程异常等等.

针对分行侧网络传输节点故障. 可以在各个分行侧网络传输节点设置测速装置, 收集交易请求数据成功上传的信息速率, 或者直接对前端交易请求数据进行收集, 如果出现故障, 那么数据成功上传的数据量和上传速度都会骤降, 同时交易请求的成功率也会骤降以此鉴别此类故障.

针对分行侧参数数据变更或者配置错误. 可以直接在数据中心后端对处理信息的成功与失败数据进行收集或者对分行侧参数数据进行收集, 如果出现故障, 数据中心后端处理失败率回增加, 同时分行侧数据会出现异常以此鉴别此类故障.

针对数据中心后端处理系统异常. 可以监测CPU的运行数据, 如主频, 外频, 工作电压, 超线程, 缓存数据集等等, 或者单笔交易的处理时间, 如果出现故障, 那么CPU的运行数据会出现异常, 同时单笔交易的处理时间会大大增加, 可与平均交易时间比较来判别.

针对数据中心后端处理系统应用进程异常. 可以监测应用进程的CPU使用率, 内存使用率, 磁盘使用率, 网络使用率等等, 可以很方便地监测各个应用的运行状况:是否在运行, 运行占用资源的程度. 同时可对应用进程处理的交易数据进行收集, 与应用进程占用资源的数据进行比对来鉴别故障, 同时也可以配合CPU监测数据进行判别.

针对不同的故障可能发生的位置, 在一笔交易完成所需要经过的路径上进行特征数据的采集, 可以更加准确地找出故障发生的位置并进行报警, 极大地提高了故障排查的效率.
