% -------------------- Packages --------------------

\documentclass[chinese]{assignment}[2021/03/08]
\usepackage[biblatex, lineno]{packages}[2021/03/08]

% -------------------- Settings --------------------

% Title

\title{\LaTeX 作业模板}
\author{陈旭阳}
\date{\today}
\institute{同济大学数学科学学院}
\professor{张赫}
\course{数理逻辑(最菜菜鸡班)}
\subject{作业模板}
\keywords{}

% -------------------- New commands --------------------



% -------------------- Document --------------------

\begin{document}
    \maketitle

    \begin{problem}
        请问郑博文是不是大傻逼\footfullcite{rudin1976principles}. [如果是, 请举出具体事例; 如果不是, 请说明理由]\footfullcite{rudin1976principleschinese2}.
        \begin{equation}
            a\in[0, 1]
        \end{equation}
    \end{problem}

    \begin{solution}
        是, 不需要理由.

        \begin{equation}
            \nabla{H}=\frac{\partial{z}}{\partial{t}}=g(\vect{\alpha}) = \symcal{A} = \symscr{A} = \int_{0}^{1}f(x)\diff x.
        \end{equation}

        你好.

        \matlabinputlisting[caption={MATLAB}, label={MATLAB}]{matlab.m}

    \end{solution}

    \clearpage

    您好.\footfullcite{rudin1976principles}

    \begin{problem}
        再您妈的见.\footfullcite{rudin1976principleschinese3}
    \end{problem}

    我们来对比下数学模式和非数学模式的字体$01234$01234.

\end{document}
